% Options for packages loaded elsewhere
\PassOptionsToPackage{unicode}{hyperref}
\PassOptionsToPackage{hyphens}{url}
\PassOptionsToPackage{dvipsnames,svgnames,x11names}{xcolor}
%
\documentclass[
  singlecolumn]{report}

\usepackage{amsmath,amssymb}
\usepackage{iftex}
\ifPDFTeX
  \usepackage[T1]{fontenc}
  \usepackage[utf8]{inputenc}
  \usepackage{textcomp} % provide euro and other symbols
\else % if luatex or xetex
  \usepackage{unicode-math}
  \defaultfontfeatures{Scale=MatchLowercase}
  \defaultfontfeatures[\rmfamily]{Ligatures=TeX,Scale=1}
\fi
\usepackage[]{libertinus}
\ifPDFTeX\else  
    % xetex/luatex font selection
\fi
% Use upquote if available, for straight quotes in verbatim environments
\IfFileExists{upquote.sty}{\usepackage{upquote}}{}
\IfFileExists{microtype.sty}{% use microtype if available
  \usepackage[]{microtype}
  \UseMicrotypeSet[protrusion]{basicmath} % disable protrusion for tt fonts
}{}
\makeatletter
\@ifundefined{KOMAClassName}{% if non-KOMA class
  \IfFileExists{parskip.sty}{%
    \usepackage{parskip}
  }{% else
    \setlength{\parindent}{0pt}
    \setlength{\parskip}{6pt plus 2pt minus 1pt}}
}{% if KOMA class
  \KOMAoptions{parskip=half}}
\makeatother
\usepackage{xcolor}
\usepackage[top=30mm,left=20mm,heightrounded]{geometry}
\setlength{\emergencystretch}{3em} % prevent overfull lines
\setcounter{secnumdepth}{-\maxdimen} % remove section numbering
% Make \paragraph and \subparagraph free-standing
\ifx\paragraph\undefined\else
  \let\oldparagraph\paragraph
  \renewcommand{\paragraph}[1]{\oldparagraph{#1}\mbox{}}
\fi
\ifx\subparagraph\undefined\else
  \let\oldsubparagraph\subparagraph
  \renewcommand{\subparagraph}[1]{\oldsubparagraph{#1}\mbox{}}
\fi


\providecommand{\tightlist}{%
  \setlength{\itemsep}{0pt}\setlength{\parskip}{0pt}}\usepackage{longtable,booktabs,array}
\usepackage{calc} % for calculating minipage widths
% Correct order of tables after \paragraph or \subparagraph
\usepackage{etoolbox}
\makeatletter
\patchcmd\longtable{\par}{\if@noskipsec\mbox{}\fi\par}{}{}
\makeatother
% Allow footnotes in longtable head/foot
\IfFileExists{footnotehyper.sty}{\usepackage{footnotehyper}}{\usepackage{footnote}}
\makesavenoteenv{longtable}
\usepackage{graphicx}
\makeatletter
\def\maxwidth{\ifdim\Gin@nat@width>\linewidth\linewidth\else\Gin@nat@width\fi}
\def\maxheight{\ifdim\Gin@nat@height>\textheight\textheight\else\Gin@nat@height\fi}
\makeatother
% Scale images if necessary, so that they will not overflow the page
% margins by default, and it is still possible to overwrite the defaults
% using explicit options in \includegraphics[width, height, ...]{}
\setkeys{Gin}{width=\maxwidth,height=\maxheight,keepaspectratio}
% Set default figure placement to htbp
\makeatletter
\def\fps@figure{htbp}
\makeatother
\newlength{\cslhangindent}
\setlength{\cslhangindent}{1.5em}
\newlength{\csllabelwidth}
\setlength{\csllabelwidth}{3em}
\newlength{\cslentryspacingunit} % times entry-spacing
\setlength{\cslentryspacingunit}{\parskip}
\newenvironment{CSLReferences}[2] % #1 hanging-ident, #2 entry spacing
 {% don't indent paragraphs
  \setlength{\parindent}{0pt}
  % turn on hanging indent if param 1 is 1
  \ifodd #1
  \let\oldpar\par
  \def\par{\hangindent=\cslhangindent\oldpar}
  \fi
  % set entry spacing
  \setlength{\parskip}{#2\cslentryspacingunit}
 }%
 {}
\usepackage{calc}
\newcommand{\CSLBlock}[1]{#1\hfill\break}
\newcommand{\CSLLeftMargin}[1]{\parbox[t]{\csllabelwidth}{#1}}
\newcommand{\CSLRightInline}[1]{\parbox[t]{\linewidth - \csllabelwidth}{#1}\break}
\newcommand{\CSLIndent}[1]{\hspace{\cslhangindent}#1}

\usepackage{cancel}
\makeatletter
\makeatother
\makeatletter
\makeatother
\makeatletter
\@ifpackageloaded{caption}{}{\usepackage{caption}}
\AtBeginDocument{%
\ifdefined\contentsname
  \renewcommand*\contentsname{Table of contents}
\else
  \newcommand\contentsname{Table of contents}
\fi
\ifdefined\listfigurename
  \renewcommand*\listfigurename{List of Figures}
\else
  \newcommand\listfigurename{List of Figures}
\fi
\ifdefined\listtablename
  \renewcommand*\listtablename{List of Tables}
\else
  \newcommand\listtablename{List of Tables}
\fi
\ifdefined\figurename
  \renewcommand*\figurename{Figure}
\else
  \newcommand\figurename{Figure}
\fi
\ifdefined\tablename
  \renewcommand*\tablename{Table}
\else
  \newcommand\tablename{Table}
\fi
}
\@ifpackageloaded{float}{}{\usepackage{float}}
\floatstyle{ruled}
\@ifundefined{c@chapter}{\newfloat{codelisting}{h}{lop}}{\newfloat{codelisting}{h}{lop}[chapter]}
\floatname{codelisting}{Listing}
\newcommand*\listoflistings{\listof{codelisting}{List of Listings}}
\makeatother
\makeatletter
\@ifpackageloaded{caption}{}{\usepackage{caption}}
\@ifpackageloaded{subcaption}{}{\usepackage{subcaption}}
\makeatother
\makeatletter
\@ifpackageloaded{tcolorbox}{}{\usepackage[skins,breakable]{tcolorbox}}
\makeatother
\makeatletter
\@ifundefined{shadecolor}{\definecolor{shadecolor}{rgb}{.97, .97, .97}}
\makeatother
\makeatletter
\makeatother
\makeatletter
\makeatother
\ifLuaTeX
  \usepackage{selnolig}  % disable illegal ligatures
\fi
\IfFileExists{bookmark.sty}{\usepackage{bookmark}}{\usepackage{hyperref}}
\IfFileExists{xurl.sty}{\usepackage{xurl}}{} % add URL line breaks if available
\urlstyle{same} % disable monospaced font for URLs
\hypersetup{
  pdftitle={scale descriptions},
  colorlinks=true,
  linkcolor={blue},
  filecolor={Maroon},
  citecolor={Blue},
  urlcolor={Blue},
  pdfcreator={LaTeX via pandoc}}

\title{scale descriptions}
\author{}
\date{}

\begin{document}
\maketitle
\ifdefined\Shaded\renewenvironment{Shaded}{\begin{tcolorbox}[boxrule=0pt, frame hidden, interior hidden, borderline west={3pt}{0pt}{shadecolor}, enhanced, breakable, sharp corners]}{\end{tcolorbox}}\fi

\listoffigures
\listoftables
\hypertarget{to-do}{%
\chapter{TO DO}\label{to-do}}

\hypertarget{rumination}{%
\subsubsection{Rumination}\label{rumination}}

``During the last 30 days, how often did\ldots. you have negative
thoughts that repeated over and over?''

Ordinal response options for the Kessler-6 are: ``None of the time'';
``A little of the time''; ``Some of the time''; ``Most of the time'';
``All of the time.''

\hypertarget{mini-ipip-6-waves-1-34-15}{%
\subsubsection{Mini-IPIP 6 (waves:
1-3,4-15)}\label{mini-ipip-6-waves-1-34-15}}

We measured participants personality with the Mini International
Personality Item Pool 6 (Mini-IPIP6)
(\protect\hyperlink{ref-sibley2011}{C. G. Sibley et al. 2011}) which
consists of six dimensions and each dimensions is measured with four
items:

\begin{enumerate}
\def\labelenumi{\arabic{enumi}.}
\item
  agreeableness,

  \begin{enumerate}
  \def\labelenumii{\roman{enumii}.}
  \item
    I sympathize with others' feelings.
  \item
    I am not interested in other people's problems. (r)
  \item
    I feel others' emotions.
  \item
    I am not really interested in others. (r)
  \end{enumerate}
\item
  conscientiousness,

  \begin{enumerate}
  \def\labelenumii{\roman{enumii}.}
  \item
    I get chores done right away.
  \item
    I like order.
  \item
    I make a mess of things. (r)
  \item
    I ften forget to put things back in their proper place. (r)
  \end{enumerate}
\item
  extraversion,

  \begin{enumerate}
  \def\labelenumii{\roman{enumii}.}
  \item
    I am the life of the party.
  \item
    I don't talk a lot. (r)
  \item
    I keep in the background. (r)
  \item
    I talk to a lot of different people at parties.
  \end{enumerate}
\item
  honesty-humility,

  \begin{enumerate}
  \def\labelenumii{\roman{enumii}.}
  \item
    I feel entitled to more of everything. (r)
  \item
    I deserve more things in life. (r)
  \item
    I would like to be seen driving around in a very expensive car. (r)
  \item
    I would get a lot of pleasure from owning expensive luxury goods.
    (r)
  \end{enumerate}
\item
  neuroticism, and

  \begin{enumerate}
  \def\labelenumii{\roman{enumii}.}
  \item
    I have frequent mood swings.
  \item
    I am relaxed most of the time. (r)
  \item
    I get upset easily.
  \item
    I seldom feel blue. (r)
  \end{enumerate}
\item
  openness to experience

  \begin{enumerate}
  \def\labelenumii{\roman{enumii}.}
  \item
    I have a vivid imagination.
  \item
    I have difficulty understanding abstract ideas. (r)
  \item
    I do not have a good imagination. (r)
  \item
    I am not interested in abstract ideas. (r)
  \end{enumerate}
\end{enumerate}

Each dimension was assessed with four items and participants rated the
accuracy of each item as it applies to them from 1 (Very Inaccurate) to
7 (Very Accurate). Items marked with (r) are reverse coded.

\hypertarget{honesty-humility-modesty-facet-waves-10-14}{%
\subsubsection{Honesty-Humility-Modesty Facet (waves:
10-14)}\label{honesty-humility-modesty-facet-waves-10-14}}

Participants indicated the extent to which they agree with the following
four statements from
(\protect\hyperlink{ref-ashton2008}{\textbf{ashton2008?}}),
(\protect\hyperlink{ref-campbell2004}{\textbf{campbell2004?}}), and C.
G. Sibley et al. (\protect\hyperlink{ref-sibley2011}{2011}) (1 =
Strongly Disagree to 7 = Strongly Agree)

\begin{verbatim}
i.  I want people to know that I am an important person of high status, (Waves: 1, 10-14)

ii. I am an ordinary person who is no better than others.

iii. I wouldn't want people to treat me as though I were superior to them.

iv. I think that I am entitled to more respect than the average person is.
\end{verbatim}

\hypertarget{age-waves-1-15}{%
\subsubsection{Age (waves: 1-15)}\label{age-waves-1-15}}

We asked participants' age in an open-ended question (``What is your
age?'' or ``What is your date of birth'').

\hypertarget{alcohol-frequency-waves-6-15}{%
\subsubsection{Alcohol Frequency (waves:
6-15)}\label{alcohol-frequency-waves-6-15}}

We measured participants' frequency of drinking alcohol using one item
adapted from
(\protect\hyperlink{ref-ministryofhealth2013}{\textbf{ministryofhealth2013?}})
. Participants were asked ``How often do you have a drink containing
alcohol?'' (1 = Never - I don't drink, 2 = Monthly or less, 3 = Up to 4
times a month, 4 = Up to 3 times a week, 5 = 4 or more times a week, 6 =
Don't know).

\hypertarget{alcohol-intensity-waves-6-15}{%
\subsubsection{Alcohol Intensity (waves:
6-15)}\label{alcohol-intensity-waves-6-15}}

We measured participants' intensity of drinking alcohol using one item
adapted from
(\protect\hyperlink{ref-ministryofhealth2013}{\textbf{ministryofhealth2013?}}).
Participants were asked ``How many drinks containing alcohol do you have
on a typical day when drinking alcohol? (number of drinks on a typical
day when drinking)''

\hypertarget{body-satisfaction-waves-2-3-4-15}{%
\subsubsection{Body Satisfaction (waves: 2-3,
4-15)}\label{body-satisfaction-waves-2-3-4-15}}

We measured body satisfaction with one item from Stronge et al.
(\protect\hyperlink{ref-stronge_facebook_2015}{2015}): ``I am satisfied
with the appearance, size and shape of my body'', which participants
rated from 1 (very inaccurate) to 7 (very accurate).

\hypertarget{nz-born-waves-1-24-15}{%
\subsubsection{NZ-Born (waves: 1-2,4-15)}\label{nz-born-waves-1-24-15}}

We asked participants ``Which country were you born in?'' or ``Where
were you born? (please be specific, e.g., which town/city?)'' (waves:
6-15).

\hypertarget{felt-belongingness-waves-1-3-4-15}{%
\subsubsection{Felt Belongingness (waves: 1-3,
4-15)}\label{felt-belongingness-waves-1-3-4-15}}

We assessed felt belongingness with three items adapted from the Sense
of Belonging Instrument (\protect\hyperlink{ref-hagerty1995}{Hagerty and
Patusky 1995}): (1) ``Know that people in my life accept and value me'';
(2) ``Feel like an outsider''; (3) ``Know that people around me share my
attitudes and beliefs''. Participants responded on a scale from 1 (Very
Inaccurate) to 7 (Very Accurate). The second item was reversely coded.

\hypertarget{charity-donation-waves-1-3-4-15}{%
\subsubsection{Charity Donation (waves: 1-3,
4-15)}\label{charity-donation-waves-1-3-4-15}}

Using one item from
(\protect\hyperlink{ref-hoverd2010}{\textbf{hoverd2010?}}), we asked
participants ``How much money have you donated to charity in the last
year?''. To stablise this indicator, we first took the natural log of
the response + 1, and then centred and standardised the log-transformed
indicator.

\hypertarget{number-of-children-waves-1-3-4-15}{%
\subsubsection{Number of Children (waves: 1-3,
4-15)}\label{number-of-children-waves-1-3-4-15}}

We measured number of children using one item from
(\protect\hyperlink{ref-bulbulia2015}{\textbf{bulbulia2015?}}). We asked
participants ``How many children have you given birth to, fathered, or
adopted. How many children have you given birth to, fathered, or
adopted?'' or ````How many children have you given birth to, fathered,
or adopted. How many children have you given birth to, fathered, and/or
parented?'' (waves: 12-15).

\hypertarget{sense-of-community-waves-6-15}{%
\subsubsection{Sense of Community (waves:
6-15)}\label{sense-of-community-waves-6-15}}

We measured sense of community with a single item from Sengupta et al.
(\protect\hyperlink{ref-sengupta2013}{2013}): ``I feel a sense of
community with others in my local neighbourhood.'' Participants answered
on a scale of 1 (strongly disagree) to 7 (strongly agree).

\hypertarget{education-attainment-waves-1-4-15}{%
\subsubsection{Education Attainment (waves: 1,
4-15)}\label{education-attainment-waves-1-4-15}}

Participants were asked ``What is your highest level of
qualification?''. We coded participans highest finished degree according
to the New Zealand Qualifications Authority. Ordinal-Rank 0-10 NZREG
codes (with overseas school quals coded as Level 3, and all other
ancillary categories coded as missing)
See:https://www.nzqa.govt.nz/assets/Studying-in-NZ/New-Zealand-Qualification-Framework/requirements-nzqf.pdf

\hypertarget{employment-waves-1-3-4-11}{%
\subsubsection{Employment (waves: 1-3,
4-11)}\label{employment-waves-1-3-4-11}}

We asked participants ``Are you currently employed? (This includes
self-employed or casual work)''. * note: This question disappeared in
the updated NZAVS Technical documents (Data Dictionary).

\hypertarget{job-security-waves-1-34-79-15}{%
\subsubsection{Job Security (waves:
1-3,4-7,9-15)}\label{job-security-waves-1-34-79-15}}

Participants indicated their feeling of job security by answering ``How
secure do you feel in your current job?'' on a scale from 1 (not secure)
to 7 (very secure).

\hypertarget{emotional-regulation-waves-10-13}{%
\subsubsection{Emotional Regulation (waves:
10-13)}\label{emotional-regulation-waves-10-13}}

We measured participants' levels of emotional regulation using three
items adpated from Gratz and Roemer
(\protect\hyperlink{ref-gratz_multidimensional_2004}{2004}) and Gross
and John (\protect\hyperlink{ref-gross_individual_2003}{2003}): (1)
``When I feel negative emotions, my emotions feel out of control.''; (2)
``When I feel negative emotions, I suppress or hide my emotions.''; (3)
``When I feel negative emotions, I change the way I think to help me
stay calm.''. Participants were asked to indicate the extent to which
they agree with these items (1 = Strongly Disagree to 7 = Strongly
Agree).

\hypertarget{european-waves-1-15}{%
\subsubsection{European (waves: 1-15)}\label{european-waves-1-15}}

Participants were asked ``Which ethnic group do you belong to (NZ census
question)?'' or ``Which ethnic group(s) do you belong to? (Open-ended)''
(wave: 3). Europeans were coded as 1, whereas other ethnicities were
coded as 0.

\hypertarget{ethnicity-waves-3}{%
\subsubsection{Ethnicity (waves: 3)}\label{ethnicity-waves-3}}

Based on the New Zealand Cencus, we asked participants ``Which ethnic
group(s) do you belong to?''. The responses were: (1) New Zealand
European; (2) Māori; (3) Samoan; (4) Cook Island Māori; (5) Tongan; (6)
Niuean; (7) Chinese; (8) Indian; (9) Other such as DUTCH, JAPANESE,
TOKELAUAN. Please state:. We coded their answers into four groups:
Maori, Pacific, Asian, and Euro (except for Time 3, which used an
open-ended measure).

\hypertarget{gender-waves-1-15}{%
\subsubsection{Gender (waves: 1-15)}\label{gender-waves-1-15}}

We asked participants' gender in an open-ended question: ``what is your
gender?'' or ``Are you male or female?'' (waves: 1-5). Female was coded
as 0, Male was coded as 1, and gender diverse coded as 3
(\protect\hyperlink{ref-fraser_coding_2020}{Fraser et al. 2020}). (or
0.5 = neither female nor male)

\hypertarget{gratitude-waves-10-15}{%
\subsubsection{Gratitude (waves: 10-15)}\label{gratitude-waves-10-15}}

We assessed the extent to which participants have gratitude, using three
items from McCullough, Emmons, and Tsang
(\protect\hyperlink{ref-mccullough_grateful_2002}{2002}): (1) ``I have
much in my life to be thankful for.''; (2) ``When I look at the world, I
don't see much to be grateful for.''; (3) ``I am grateful to a wide
variety of people.''. Participants indicated their agreement with these
items (1 = Strongly Disagree to 7 = Strongly Agree). The second item was
reversely coded.

\hypertarget{home-owner-waves-4}{%
\subsubsection{Home Owner (waves: 4)}\label{home-owner-waves-4}}

Using one item from Houkamau and Sibley
(\protect\hyperlink{ref-houkamau_looking_2015}{2015}), we asked
participants ``(Māori) Do you own your own home?(either partly or fully
owned)'' (1 = Yes, 0 = No)

\hypertarget{hours-of-exercise-waves-1-4-15}{%
\subsubsection{Hours of Exercise (waves: 1,
4-15)}\label{hours-of-exercise-waves-1-4-15}}

We measured hours of exercising using one item from C. G. Sibley et al.
(\protect\hyperlink{ref-sibley2011}{2011}). We asked participants to
estimate and report how many hours they spend in exercise/physical
activity last week. To stablise this indicator, we first took the
natural log of the response + 1, and then centred and standardised the
log-transformed indicator.

\hypertarget{hours-of-work-waves-1-4-15}{%
\subsubsection{Hours of Work (waves: 1,
4-15)}\label{hours-of-work-waves-1-4-15}}

We measured hours of Work using one item from C. G. Sibley et al.
(\protect\hyperlink{ref-sibley2011}{2011}). We asked participants to
estimate and report how many hours they spend in working in paid
employment last week.

\hypertarget{body-mass-index-waves-2-3-4-15}{%
\subsubsection{Body Mass Index (waves: 2-3,
4-15)}\label{body-mass-index-waves-2-3-4-15}}

Participants were asked ``What is your height? (metres)'' and ``What is
your weight? (kg)''. Based on participants indication of their height
and weight we calculated the BMI by dividing the weight in kilograms by
the square of the height in meters.

\hypertarget{disability-waves-5-15}{%
\subsubsection{Disability (waves: 5-15)}\label{disability-waves-5-15}}

We assessed disability with a one item indicator adapted from Verbrugge
(\protect\hyperlink{ref-verbrugge1997}{1997}), that asks ``Do you have a
health condition or disability that limits you, and that has lasted for
6+ months?'' (1 = Yes, 0 = No).

\hypertarget{fatigue-waves-5-15}{%
\subsubsection{Fatigue (waves: 5-15)}\label{fatigue-waves-5-15}}

We assessed subjective fatigue by asking participants, ``During the last
30 days, how often did \ldots{} you feel exhausted?'' Responses were
collected on an ordinal scale (0 = None of The Time, 1 = A little of The
Time, 2 = Some of The Time, 3 = Most of The Time, 4 = All of The Time).

\hypertarget{hours-of-sleep-waves-5-15}{%
\subsubsection{Hours of Sleep (waves:
5-15)}\label{hours-of-sleep-waves-5-15}}

Participants were asked ``During the past month, on average, how many
hours of \emph{actual sleep} did you get per night''.

\hypertarget{ethnic-group-impermeability-waves-9-13}{%
\subsubsection{Ethnic group impermeability (waves:
9-13)}\label{ethnic-group-impermeability-waves-9-13}}

We assessed ethnic group impermeability using one item from Mummendey
and Wenzel (\protect\hyperlink{ref-mummendey_social_1999}{1999}).
Participants were asked to indicate the extent to which they agree with
the statement (``The current income gap between New Zealand Europeans
and other ethnic groups would be very hard to change.''; 1 = Strongly
Disagree to 7 = Strongly Agree).

\hypertarget{income-waves-1-3-4-15}{%
\subsubsection{Income (waves: 1-3, 4-15)}\label{income-waves-1-3-4-15}}

Participants were asked ``Please estimate your total household income
(before tax) for the year XXXX''. To stablise this indicator, we first
took the natural log of the response + 1, and then centred and
standardised the log-transformed indicator.

\hypertarget{kessler-6-waves-2-34-15}{%
\subsubsection{Kessler-6 (waves:
2-3,4-15)}\label{kessler-6-waves-2-34-15}}

We measured psychological distress using the Kessler-6 scale
(\protect\hyperlink{ref-kessler2002}{R. ~C. Kessler et al. 2002}), which
exhibits strong diagnostic concordance for moderate and severe
psychological distress in large, cross-cultural samples
(\protect\hyperlink{ref-kessler2010}{R. C. Kessler et al. 2010};
\protect\hyperlink{ref-prochaska2012}{Prochaska et al. 2012}).
Participants rated during the past 30 days, how often did\ldots{} (1)
``\ldots{} you feel hopeless''; (2) ``\ldots{} you feel so depressed
that nothing could cheer you up''; (3) ``\ldots{} you feel restless or
fidgety''; (4)``\ldots{} you feel that everything was an effort''; (5)
``\ldots{} you feel worthless''; (6) '' you feel nervous?'' Ordinal
response alternatives for the Kessler-6 are: ``None of the time''; ``A
little of the time''; ``Some of the time''; ``Most of the time''; ``All
of the time.''

\hypertarget{life-meaning-waves-10-15}{%
\subsubsection{Life Meaning (waves:
10-15)}\label{life-meaning-waves-10-15}}

We assessed participants' levels of life meaning using two items from
Steger et al. (\protect\hyperlink{ref-steger_meaning_2006}{2006}): (1)
My life has a clear sense of purpose; (2) I have a good sense of what
makes my life meaningful. Participants indicated their agreement with
these items (1 = Strongly Disagree to 7 = Strongly Agree).

\hypertarget{satisfaction-with-life-waves-1-34-15}{%
\subsubsection{Satisfaction with Life (waves:
1-3,4-15)}\label{satisfaction-with-life-waves-1-34-15}}

We measured life satisfaction with two items adapted from the
Satisfaction with Life Scale (\protect\hyperlink{ref-diener1985}{Diener
et al. 1985}): ``I am satisfied with my life'' and ``In most ways my
life is close to ideal''. Participants responded on a scale from 1
(Strongly Disagree) to 7 (Strongly Agree).

\hypertarget{lost-job-waves-10-15}{%
\subsubsection{Lost Job (waves: 10-15)}\label{lost-job-waves-10-15}}

Participants were asked to indicate whether they lost their job or had
the principal earner in their household lose job in the last year (1 =
Yes, 0 = No).

\hypertarget{nz-deprivation-index-waves}{%
\subsubsection{NZ Deprivation Index (waves:
??)}\label{nz-deprivation-index-waves}}

We used the NZ Deprivation Index to assign each participant a score
based on where they live (\protect\hyperlink{ref-atkinson2019}{Atkinson,
Salmond, and Crampton 2019}). This score combines data such as income,
home ownership, employment, qualifications, family structure, housing,
and access to transport and communication for an area into one
deprivation score.

\hypertarget{national-well-being-index-waves-1-3-4-15}{%
\subsubsection{National Well-being Index (waves: 1-3,
4-15)}\label{national-well-being-index-waves-1-3-4-15}}

We measured participant's sense of national well-being by asking them to
``Please rate your level of satisfaction with the following aspects of
your life and New Zealand: (1) The economic situation in New Zealand,
(2) The social conditions in New Zealand, and (3) Business in New
Zealand'' on a scale from 0 (Completely Dissatisfied) to 10 (Completely
Satisfied) (\protect\hyperlink{ref-tiliouine2006}{Tiliouine, Cummins,
and Davern 2006}).

\hypertarget{nzsei-13-waves-8-15}{%
\subsubsection{NZSEI-13 (waves: 8-15)}\label{nzsei-13-waves-8-15}}

We assessed occupational prestige and status using the New Zealand
Socio-economic Index 13 (NZSEI-13)
(\protect\hyperlink{ref-fahy2017}{Fahy, Lee, and Milne 2017}). This
index uses the income, age, and education of a reference group, in this
case the 2013 New Zealand census, to calculate an score for each
occupational group. Scores range from 10 (Lowest) to 90 (Highest). This
list of index scores for occupational groups was used to assign each
participant a NZSEI-13 score based on their occupation.

\hypertarget{parent-waves-5-15}{%
\subsubsection{Parent (waves: 5-15)}\label{parent-waves-5-15}}

Participants were asked ``If you are a parent, what is the birth date of
your eldest child?'' or ``If you are a parent, in which year was your
eldest child born?'' (waves: 10-15). Parents were coded as 1, while the
others were coded as 0. .

\hypertarget{living-with-partner-waves-4-7-12}{%
\subsubsection{Living with Partner (waves: 4-7,
12)}\label{living-with-partner-waves-4-7-12}}

Participants were asked ``Do you live with your partner?'' (1 = Yes, 0 =
No).

\hypertarget{perfectionism-waves-10-15}{%
\subsubsection{Perfectionism (waves:
10-15)}\label{perfectionism-waves-10-15}}

We assessed participants' perfectionism using three items from Rice,
Richardson, and Tueller (\protect\hyperlink{ref-rice_short_2014}{2014}):
(1) Doing my best never seems to be enough; (2) My performance rarely
measures up to my standards; (3) I am hardly ever satisfied with my
performance. Participants indicated the extent to which they agree with
these items (1 = Strongly Disagree to 7 = Strongly Agree).

\hypertarget{political-wing-waves-34-15}{%
\subsubsection{Political Wing (waves:
3,4-15)}\label{political-wing-waves-34-15}}

We measured participants' political orientation using a single item
adapted from (\protect\hyperlink{ref-jost2006}{\textbf{jost2006?}}) .
Participants were asked to rate how politically left-wing versus
right-wing they see themselves as being (1 = Extremely Left-wing to 7 =
Extremely Right-wing)

\hypertarget{powerlessness-waves-10-13}{%
\subsubsection{Powerlessness (waves:
10-13)}\label{powerlessness-waves-10-13}}

Participants' Powerlessness was measured using two items: (1) I do not
have enough power or control over important parts of my life; (2) Other
people have too much power or control over important parts of my life.
Participants indicated their agreement with these items (1 = Strongly
Disagree to 7 = Strongly Agree).

\hypertarget{self-respect-waves-3-4-11-15}{%
\subsubsection{Self-Respect (waves: 3, 4-11,
15)}\label{self-respect-waves-3-4-11-15}}

We assessed participants' levels of self-respect using an item adapted
from Tyler, Degoey, and Smith
(\protect\hyperlink{ref-tyler_understanding_1996}{1996}). Participant
indicated the extent to which they agree with the statement (``If they
knew me, most NZers would respect what I have accomplished in life'') on
a likert scale (1 = Strongly Disagree to 7 = Strongly Agree)

\hypertarget{retired-waves-10-15}{%
\subsubsection{Retired (waves: 10-15)}\label{retired-waves-10-15}}

Participants were asked to indicate whether they were retired or not in
the last year (1 = Yes, 0 = No).

\hypertarget{rumination-waves-2-34-15}{%
\subsubsection{Rumination (waves:
2-3,4-15)}\label{rumination-waves-2-34-15}}

We used a single item adapted from Nolen-hoeksema and Morrow
(\protect\hyperlink{ref-nolen-hoeksema_effects_1993}{1993}) to measure
participants' level of rumination. Participants were asked, ``During the
last 30 days, how often did\ldots you have negative thoughts that
repeated over and over?'' and indicated their frequency of rumination on
an ordinal scale (0 = None of The time, 1 = A little of the time, 2 =
some of the time, 3 = most of the time, 4 = all of the time)

\hypertarget{self-control-waves-5-15}{%
\subsubsection{Self-Control (waves:
5-15)}\label{self-control-waves-5-15}}

Participants were asked to indicate the extent to which they endorse the
two items (``In general, I have a lot of self-control'', ``I wish I had
more self-discipline'') from Tangney, Baumeister, and Boone
(\protect\hyperlink{ref-tangney_high_2004}{2004}). The responses to the
items ranged from 1 (Strongly Disagree) to 7 (Strongly Agree).

\hypertarget{self-esteem-waves-1-3-4-15}{%
\subsubsection{Self-Esteem (waves: 1-3,
4-15)}\label{self-esteem-waves-1-3-4-15}}

We measured participants' self-esteem using three items adapted from
(\protect\hyperlink{ref-rosenberg1965}{\textbf{rosenberg1965?}}).
Participants were instructed to circle the number that best represents
how accurately each statement describes them. Participants responded to
the items (``On the whole am satisfied with myself.'', ``Take a positive
attitude toward myself'', ``Am inclined to feel that I am a failure'')
on a likert-type scale (1 = Very inaccurate to 7 = Very accurate).

\hypertarget{semiretired-waves-10-15}{%
\subsubsection{Semiretired (waves:
10-15)}\label{semiretired-waves-10-15}}

SAME AS RETIRED

\hypertarget{sexual-orientation-waves-5-15}{%
\subsubsection{Sexual Orientation (waves:
5-15)}\label{sexual-orientation-waves-5-15}}

Participants were asked to report their sexual orientation: ``How would
you describe your sexual orientation? (e.g., heterosexual, homosexual,
straight, gay, lesbian, bisexual, etc.)''.

\hypertarget{sexual-satisfaction-waves-10-15}{%
\subsubsection{Sexual Satisfaction (waves:
10-15)}\label{sexual-satisfaction-waves-10-15}}

Participants were asked ``How satisfied are you with your sex life?'' (1
= Not satisfied to 7 = Very satisfied).

\hypertarget{health-condition-or-disability-waves-5-15}{%
\subsubsection{Health Condition or Disability (waves:
5-15)}\label{health-condition-or-disability-waves-5-15}}

Participants' subjective health was measured using one item (``Do you
have a health condition or disability that limits you, and that has
lasted for 6+ months?''; 1 = Yes, 0 = No) adapted from Verbrugge
(\protect\hyperlink{ref-verbrugge1997}{1997}).

\hypertarget{smoker-waves-4-15}{%
\subsubsection{Smoker (waves: 4-15)}\label{smoker-waves-4-15}}

We asked participants whether they are currently smoking or not (1 = Yes
or 0 = No), using a single item: ``Do you currently smoke?'' or ``Do you
currently smoke tobacco cigarettes?'' (waves: 10-15) from Muriwai,
Houkamau, and Sibley
(\protect\hyperlink{ref-muriwai_looking_2018}{2018}).

\hypertarget{standard-living-waves-1-3-4-15}{%
\subsubsection{Standard Living (waves: 1-3,
4-15)}\label{standard-living-waves-1-3-4-15}}

We measured participants' satisfaction with their standard of living
using an item from the Australian Unity Wellbeing Index
(\protect\hyperlink{ref-cummins_developing_2003}{Cummins et al. 2003}).
Participants read an instruction (``Please rate your level of
satisfaction with the following aspects of your life and New Zealand.'')
and responded to an item (``Your standard of living'') on a 10-point
scale (0 = completely dissatisfied to 10 = completely satisfied).

\hypertarget{social-support-waves-1-3-4-15}{%
\subsubsection{Social Support (waves: 1-3,
4-15)}\label{social-support-waves-1-3-4-15}}

Participants' perceived social support was measured using three items
from Cutrona and Russell (\protect\hyperlink{ref-cutrona1987}{1987}) and
Williams, Cheung, and Choi
(\protect\hyperlink{ref-williams_cyberostracism_2000}{2000}): (1)
``There are people I can depend on to help me if I really need it''; (2)
``There is no one I can turn to for guidance in times of stress''; (3)
``I know there are people I can turn to when I need help.'' Participants
indicated the extent to which they agree with those items (1 = Strongly
Disagree to 7 = Strongly Agree). The second item was negatively-worded,
so we reversely recorded the responses to this item.

\hypertarget{living-in-an-urban-area-waves-1-15}{%
\subsubsection{Living in an Urban Area (waves:
1-15)}\label{living-in-an-urban-area-waves-1-15}}

We coded whether they are living in an urban or rural area (1 = Urban, 0
= Rural) based on the addresses provided.

\hypertarget{vengeful-rumination-waves-10-15}{%
\subsubsection{Vengeful Rumination (waves:
10-15)}\label{vengeful-rumination-waves-10-15}}

We assessed participants' vengeful rumination using three items,
respectively adapted from Caprara
(\protect\hyperlink{ref-caprara_indicators_1986}{1986}) and Berry et al.
(\protect\hyperlink{ref-berry_forgivingness_2005}{2005}), and developed
for NZAVS: (1) Sometimes I can't sleep because of thinking about past
wrongs I have suffered; (2) I can usually forgive and forget when
someone does me wrong; (3) I find myself regularly thinking about past
times that I have been wronged. Participants indicated their agreement
with these items (1 = Strongly Disagree to 7 = Strongly Agree). The
values for the second item were reversely coded.

\hypertarget{volunteers-waves-1-4-15}{%
\subsubsection{Volunteers (waves: 1,
4-15)}\label{volunteers-waves-1-4-15}}

Participants were asked,``Please estimate how many hours you spent doing
each of the following things last week'' and responded to an item
(``voluntary/charitable work''), from
(\protect\hyperlink{ref-sibley2011}{C. G. Sibley et al. 2011}).

\hypertarget{personal-wellbeing-waves-1-3-4-15}{%
\subsubsection{Personal Wellbeing (waves: 1-3,
4-15)}\label{personal-wellbeing-waves-1-3-4-15}}

We measured participants' subjective wellbeing using three items from
the Australian Unity Wellbeing Index
(\protect\hyperlink{ref-cummins_developing_2003}{Cummins et al. 2003}):
(1) your health; (2) Your standard of living; (3) your future security;
(4) Your personal relationships. Participants read an instruction (``The
following items assess your current satisfaction with different aspects
of your life and aspects of New Zealand more generally'') and indicated
their satisfaction with those items (0 = Completely Dissatisfied to 10 =
Completely Satisfied).

\hypertarget{satisfaction-with-nz-environment-waves-1-3-4-15}{%
\subsubsection{Satisfaction with NZ Environment (waves: 1-3,
4-15)}\label{satisfaction-with-nz-environment-waves-1-3-4-15}}

Using one item Adapted from the International Wellbeing Index
(\protect\hyperlink{ref-tiliouine2006}{Tiliouine, Cummins, and Davern
2006}), we asked participants to assess their current satisfaction with
the quality of New Zealand's natural environment (1 = Completely
dissatisfied to 10 = Completely satisfied).

\hypertarget{opinions-about-government-spending-on-motoways-waves-1-4-5-10-11}{%
\subsubsection{Opinions about Government Spending on Motoways (waves: 1,
4-5,
10-11)}\label{opinions-about-government-spending-on-motoways-waves-1-4-5-10-11}}

Participants indicated the extent to which they oppose or support an
item (``Increased government spending on new motorways'') from
(\protect\hyperlink{ref-milfont2022}{\textbf{milfont2022?}}) on a likert
scale (1 = Strongly Oppose to 7 = Strongly Support).

\hypertarget{opinions-about-government-subsidy-of-public-transport-waves-1-4-5-10-11}{%
\subsubsection{Opinions about Government Subsidy of Public Transport
(waves: 1, 4-5,
10-11)}\label{opinions-about-government-subsidy-of-public-transport-waves-1-4-5-10-11}}

Participants indicated the extent to which they oppose or support an
item (``Government subsidy of public transport'') from
(\protect\hyperlink{ref-milfont2020}{\textbf{milfont2020?}}) on a likert
scale (1 = Strongly Oppose to 7 = Strongly Support).

\hypertarget{protecting-native-species-waves-2-5-8-9-11-14-15}{%
\subsubsection{Protecting Native Species (waves: 2, 5, 8-9, 11,
14-15)}\label{protecting-native-species-waves-2-5-8-9-11-14-15}}

Participants indicated the extent to which they agree with an item
(``Protecting New Zealand's native species should be a national
priority''), 1 = Strongly Disagree to 7 = Strongly Agree. (Developed for
NZAVS)

\hypertarget{climate-change-belief-waves-1-15}{%
\subsubsection{Climate Change Belief (waves:
1-15)}\label{climate-change-belief-waves-1-15}}

We measured participants' climate change belief using three items from
(\protect\hyperlink{ref-sibley2013}{\textbf{sibley2013?}}) and
(\protect\hyperlink{ref-milfont2022}{\textbf{milfont2022?}}): (1)
``Climate change is real''; (2) ``Climate change is caused by humans'';
(3) ``I am deeply concerned about climate change.'' (waves: 5-15).
Participants indicated the extent to which they agree with these items
(1 = Strongly Disagree to 7 = Strongly Agree).

\hypertarget{valuing-environmental-protection-waves-1-3-4-7-8}{%
\subsubsection{Valuing Environmental Protection (waves: 1-3, 4,
7-8)}\label{valuing-environmental-protection-waves-1-3-4-7-8}}

Using one item from
(\protect\hyperlink{ref-schwartz1992}{\textbf{schwartz1992?}}), we asked
participants to indicate how important protecting the environment
(preserving nature) is for them as a guiding principle in your life (-1
= Opposed to my values to 7 = Of supreme importance).

\hypertarget{personal-environmental-sacrifice-waves-1-3-4-6-9}{%
\subsubsection{Personal Environmental Sacrifice (waves: 1-3, 4-6,
9)}\label{personal-environmental-sacrifice-waves-1-3-4-6-9}}

We measured participants' sacrifices for the national environment using
four items, adapted from
(\protect\hyperlink{ref-liu2012}{\textbf{liu2012?}}): (1) Environmental
sacrifices made: ``Have you made sacrifices to your standard of living
(e.g., accepted higher prices, driven less, conserved energy) in order
to protect the environment?''; (2) Willingness to make environmental
sacrifices: ``Are you willing to make sacrifices to your standard of
living (e.g., accept higher prices, drive less, conserve energy) in
order to protect the environment?'' (3) Daily routine changed: ``Have
you made changes to your daily routine in order to protect the
environment?'' (waves: 1-6); (4) Willingness to change daily routine:
``Are you willing to change your daily routine in order to protect the
environment?'' (waves: 1-6). Responses to these items ranged from
Definitely No (1) to Definitely Yes (7).

\hypertarget{environmental-sacrifice-norm-perception-waves-1-3-4-6-9}{%
\subsubsection{Environmental Sacrifice Norm Perception (waves: 1-3, 4-6,
9)}\label{environmental-sacrifice-norm-perception-waves-1-3-4-6-9}}

Environmental sacrifice norm perceptions was measured with one single
question, adapted from
(\protect\hyperlink{ref-liu2012}{\textbf{liu2012?}}): ``Do you think
most New Zealanders are willing to make sacrifices to their standard of
living in order to protect the environment?''. Responses to these items
ranged from Definitely No (1) to Definitely Yes (7).

\hypertarget{opinion-about-the-governments-carbon-emissions-regulation-waves}{%
\subsubsection{Opinion about the Government's Carbon Emissions
Regulation (waves:
)}\label{opinion-about-the-governments-carbon-emissions-regulation-waves}}

We asked participants' opinion about the government's carbon emissions
regulation using one item from
(\protect\hyperlink{ref-satherley2020}{\textbf{satherley2020?}}).
Participants were asked to indicate their agreement with the statement
(``The New Zealand government should be involved in regulating carbon
emissions.'') on a scale ranging from Strongly Disagree (1) to Strongly
Agree (7).

\hypertarget{quality-of-waterways-waves-9-11-14-15}{%
\subsubsection{Quality of Waterways (waves: 9-11,
14-15)}\label{quality-of-waterways-waves-9-11-14-15}}

Participants rated their level of satisfaction with one item (``the
quality and health of the waterways in your local region'') from
(\protect\hyperlink{ref-milfont2022}{\textbf{milfont2022?}}) (0 =
Completely Dissatisfied to 10 = Completely Satisfied)

\hypertarget{valuing-unity-with-nature-waves-7}{%
\subsubsection{Valuing Unity with Nature (waves:
7)}\label{valuing-unity-with-nature-waves-7}}

Using one item from
(\protect\hyperlink{ref-schwartz1992}{\textbf{schwartz1992?}}), We asked
participants to indicate how important UNITY WITH NATURE (fitting into
nature) is for them as a guiding principle in your life (-1 = Opposed to
my values to 7 = Of supreme importance).

\hypertarget{possum-control-waves-3-5-8-9-11}{%
\subsubsection{Possum Control (waves: 3, 5, 8-9,
11)}\label{possum-control-waves-3-5-8-9-11}}

Participants were asked, ``Do you support the use of 1080 poison for
possum control in New Zealand?''. Responses to this item ranged from
Definitely No (1) to Definitely Yes (7).

\hypertarget{pro-environmental-efficacy-waves-1-3-4-6-8-9-12-15}{%
\subsubsection{Pro-Environmental Efficacy (waves: 1-3, 4-6, 8-9,
12-15)}\label{pro-environmental-efficacy-waves-1-3-4-6-8-9-12-15}}

We measured participants' pro-environmental efficacy using two items
from (\protect\hyperlink{ref-sharma}{\textbf{sharma?}}): (1) belief:
``By taking personal action I believe I can make a positive difference
to environmental problems.''; (2) feeling:``I feel I can make a
difference to the state of the environment.'' Participants indicated the
extent to which they agree with these items (1 = Strongly Disagree to 7
= Strongly Agree).

\hypertarget{earthquake-anxiety-waves-4-14}{%
\subsubsection{Earthquake Anxiety (waves:
4-14)}\label{earthquake-anxiety-waves-4-14}}

Using one item from
(\protect\hyperlink{ref-rowney2014}{\textbf{rowney2014?}}), we asked
participatns ``During the last 30 days, how often did you worry that a
big earthquake might hit your region'' (0 = None of the time to 4 = All
of the time)?

\hypertarget{affected-by-christchurch-earthquakes-waves-3-6}{%
\subsubsection{Affected by Christchurch Earthquakes (waves: 3,
6)}\label{affected-by-christchurch-earthquakes-waves-3-6}}

Participants were asked ``Where you personally affected by the
Christchurch earthquakes?'' (1 = Yes or 0 = No)

\hypertarget{belief-in-god-waves-2-5-7-15}{%
\subsubsection{Belief in God (waves: 2-5,
7-15)}\label{belief-in-god-waves-2-5-7-15}}

Using one item from
(\protect\hyperlink{ref-eurobarometer2005}{\textbf{eurobarometer2005?}}),
we asked participants ``Do you believe in a God'' (1 = Yes, 0 = No).

\hypertarget{belief-in-sprituality-waves-2-5-7-15}{%
\subsubsection{Belief in Sprituality (waves: 2-5,
7-15)}\label{belief-in-sprituality-waves-2-5-7-15}}

Using one item from
(\protect\hyperlink{ref-eurobarometer2005}{\textbf{eurobarometer2005?}}),
we asked participants ``Do you believe in some form of spirit or
lifeforce? (1 = Yes, 0 = No).

\hypertarget{religion-affiliation-waves-1-15}{%
\subsubsection{Religion Affiliation (waves:
1-15)}\label{religion-affiliation-waves-1-15}}

Using one item from
(\protect\hyperlink{ref-hoverd2010}{\textbf{hoverd2010?}}), we asked
participants to indicate their religion identification (``Do you
identify with a religion and/or spiritual group?'') on a binary response
(1 = Yes, 0 = No). We then asked ``What religion or spiritual group?''
in a open-ended question. These questions are used in the New Zealand
Census.

\hypertarget{religious-identification-waves-1-15}{%
\subsubsection{Religious Identification (waves:
1-15)}\label{religious-identification-waves-1-15}}

If participants answered \emph{yes} to ``Do you identify with a religion
and/or spiritual group? we asked''How important is your religion to how
you see yourself?'' (1 = Not important, 7 = Very important), using one
item from (\protect\hyperlink{ref-hoverd2010}{\textbf{hoverd2010?}}).
Those participants who were not religious were imputed a score of ``1''.

\hypertarget{frequency-of-church-attendance-waves-2-11-13-15}{%
\subsubsection{Frequency of Church Attendance (waves: 2-11,
13-15)}\label{frequency-of-church-attendance-waves-2-11-13-15}}

If participants answered \emph{yes} to ``Do you identify with a religion
and/or spiritual group?'' we measured their frequency of church
attendance using one item from \&. B. Sibley C. G.
(\protect\hyperlink{ref-sibley2012}{2012}): ``how many times did you
attend a church or place of worship in the last month?''. Those
participants who were not religious were imputed a score of ``0''.

\hypertarget{spiritual-identification-waves-8-10-12-15}{%
\subsubsection{Spiritual Identification (waves: 8, 10,
12-15)}\label{spiritual-identification-waves-8-10-12-15}}

Spiritual identification was measured using one item (``I identify as a
spiritual person.'') from Postmes, Haslam, and Jans
(\protect\hyperlink{ref-postmes_single-item_2013}{2013}). Participants
indicated their agreement with this item (1 = Strongly Disagree to 7 =
Strongly Agree).

\hypertarget{frequency-of-prayer-waves-3-11-13-15}{%
\subsubsection{Frequency of Prayer (waves: 3-11,
13-15)}\label{frequency-of-prayer-waves-3-11-13-15}}

If participants answered \emph{yes} to ``Do you identify with a religion
and/or spiritual group?'' we measured their frequency of prayer by
asking the question (``how many times did you pray in the last week?'')
from (\protect\hyperlink{ref-bulbulia2015}{\textbf{bulbulia2015?}}).
Those participants who were not religious were imputed a score of ``0''.

\hypertarget{frequency-of-scripture-reading-waves-3.5-6-11-13-15}{%
\subsubsection{Frequency of Scripture Reading (waves: 3.5, 6-11,
13-15)}\label{frequency-of-scripture-reading-waves-3.5-6-11-13-15}}

If participants answered \emph{yes} to ``Do you identify with a religion
and/or spiritual group?'' we measured their frequency of scripture
reading by asking the question (``how many times did you read religious
scripture in the last week?'') from Bulbulia et al.
(\protect\hyperlink{ref-bulbulia2016}{2016}). Those participants who
were not religious were imputed a score of ``0''

\hypertarget{perceived-discrimination-religion-waves-7-15}{%
\subsubsection{Perceived Discrimination -- Religion (waves:
7-15)}\label{perceived-discrimination-religion-waves-7-15}}

Participants were asked to indicate the extent to which they agree with
the statement, ``I feel that I am often discriminated against because of
my religious/spiritual beliefs.'' (1 = Strongly Disagree to 7 = Strongly
Agree).

\hypertarget{religion-church-waves-2-11-13-15}{%
\subsubsection{Religion -- Church (waves: 2-11,
13-15)}\label{religion-church-waves-2-11-13-15}}

Based on participants' frequency of church attendance, we coded whether
participants attend a church at least one time per month or not (1 =
Yes, 0 = No).

\hypertarget{sample-frame-waves-1-15}{%
\subsubsection{Sample Frame (waves:
1-15)}\label{sample-frame-waves-1-15}}

see NZAVS

\hypertarget{generation-cohort-waves-1-15}{%
\subsubsection{Generation Cohort (waves:
1-15)}\label{generation-cohort-waves-1-15}}

We use the Pew Research Centre birth cohort categories
(\protect\hyperlink{ref-michaeldimock2019}{\textbf{michaeldimock2019?}}):
Generation Silent (1928-1945), Generation Baby Boomer (1946-1964),
Generation X (1965-1980), Generation Y (1981-1996) and Generation Z
(1997-2012).

\hypertarget{political-orientation-waves-1-3-4-15}{%
\subsubsection{Political Orientation (waves: 1-3,
4-15)}\label{political-orientation-waves-1-3-4-15}}

We measured participants' political orientation using a single item
adapted from (\protect\hyperlink{ref-jost2006}{\textbf{jost2006?}}).
Participants were asked to rate how politically liberal versus
conservative they see themselves as being (1 = Extremely liberal to 7 =
Extremely conservative)

\hypertarget{rural_gch2018-waves}{%
\subsubsection{``rural\_gch2018'', (waves:
)}\label{rural_gch2018-waves}}

see NZAVS, report here

\hypertarget{bigger_doms-waves}{%
\subsubsection{``bigger\_doms'' (waves: )}\label{bigger_doms-waves}}

Developed by combining large census categories (Note Inkuk there will
not be a reference for this variable -- so no citation)

\hypertarget{sample_weights-waves}{%
\subsubsection{sample\_weights (waves: )}\label{sample_weights-waves}}

(note inkuk we use ``w\_gend\_age\_euro'' )

\hypertarget{short-form-subjective-health-waves-5-15}{%
\subsubsection{Short-Form Subjective Health (waves:
5-15)}\label{short-form-subjective-health-waves-5-15}}

Participants' subjective health was assessed by three items selected
from the MOS 36-item short-form health survey
(\protect\hyperlink{ref-warejr1992}{\textbf{warejr1992?}}). The items
were (1) ``In general, would you say your health is\ldots{}''; (2) ``I
seem to get sick a little easier than most people.''; (3) ``I expect my
health to get worse.'' Participants responded to those items on a scale
(1 = Poor to 7 = Excellent). The second and third items were
negatively-worded, so we reversed the responses.

\hypertarget{work-life-balance-waves-12-13}{%
\subsubsection{Work-Life Balance (waves:
12-13)}\label{work-life-balance-waves-12-13}}

Participants indicated the extent to which they agree with the
statement, ``I have a good balance between work and other important
things in my life.'' (1 = Strongly Disagree to 7 = Strongly Agree).

\hypertarget{individual-permeability-waves-9-13}{%
\subsubsection{Individual Permeability (waves:
9-13)}\label{individual-permeability-waves-9-13}}

Participants indicated the extent to which they agree with the
statement, ``I believe I am capable, as an individual of improving my
status in society.'', from
(\protect\hyperlink{ref-tausch2015}{\textbf{tausch2015?}}) (1 = Strongly
Disagree to 7 = Strongly Agree).

\pagebreak

\hypertarget{references}{%
\subsubsection{References}\label{references}}

\hypertarget{refs}{}
\begin{CSLReferences}{1}{0}
\leavevmode\vadjust pre{\hypertarget{ref-atkinson2019}{}}%
Atkinson, J, C Salmond, and P Crampton. 2019. {``NZDep2018 Index of
Deprivation, User{'}s Manual.''} Wellington.

\leavevmode\vadjust pre{\hypertarget{ref-berry_forgivingness_2005}{}}%
Berry, Jack W., Everett L. Worthington Jr., Lynn E. O'Connor, Les
Parrott III, and Nathaniel G. Wade. 2005. {``Forgivingness, Vengeful
Rumination, and Affective Traits.''} \emph{Journal of Personality} 73
(1): 183--226. \url{https://doi.org/10.1111/j.1467-6494.2004.00308.x}.

\leavevmode\vadjust pre{\hypertarget{ref-bulbulia2016}{}}%
Bulbulia, Joseph, Geoffrey Troughton, Lara M. Greaves, Taciano L.
Milfont, and Chris G. Sibley. 2016. {``To Burn or to Save? The Opposing
Functions of Reading Scripture on Environmental Intentions.''}
\emph{Religion, Brain \& Behavior} 6 (4): 278--89.
\url{https://doi.org/10.1080/2153599X.2015.1026926}.

\leavevmode\vadjust pre{\hypertarget{ref-caprara_indicators_1986}{}}%
Caprara, Gian Vittorio. 1986. {``Indicators of Aggression: The
Dissipation-Rumination Scale.''} \emph{Personality and Individual
Differences} 7 (6): 763--69.
\url{https://doi.org/10.1016/0191-8869(86)90074-7}.

\leavevmode\vadjust pre{\hypertarget{ref-cummins_developing_2003}{}}%
Cummins, Robert A., Richard Eckersley, Julie Pallant, Jackie van Vugt,
and RoseAnne Misajon. 2003. {``Developing a National Index of Subjective
Wellbeing: The Australian Unity Wellbeing Index.''} \emph{Social
Indicators Research} 64 (2): 159--90.
\url{https://doi.org/10.1023/A:1024704320683}.

\leavevmode\vadjust pre{\hypertarget{ref-cutrona1987}{}}%
Cutrona, Carolyn E, and Daniel Wayne Russell. 1987. {``The Provisions of
Social Relationships and Adaptation to Stress.''} \emph{Advances in
Personal Relationships} 1: 37--67.

\leavevmode\vadjust pre{\hypertarget{ref-diener1985}{}}%
Diener, Ed, Robert A Emmons, Randy J Larsen, and Sharon Griffin. 1985.
{``The Satisfaction With Life Scale.''} \emph{Journal of Personality
Assessment} 49 (1): 71--75.

\leavevmode\vadjust pre{\hypertarget{ref-fahy2017}{}}%
Fahy, Katie M., Alan Lee, and Barry J. Milne. 2017. \emph{New Zealand
Socio-Economic Index 2013}. Wellington, New Zealand: Statistics New
Zealand-Tatauranga Aotearoa.

\leavevmode\vadjust pre{\hypertarget{ref-fraser_coding_2020}{}}%
Fraser, Gloria, Joseph Bulbulia, Lara M. Greaves, Marc S. Wilson, and
Chris G. Sibley. 2020. {``Coding Responses to an Open-Ended Gender
Measure in a New Zealand National Sample.''} \emph{The Journal of Sex
Research} 57 (8): 979--86.
\url{https://doi.org/10.1080/00224499.2019.1687640}.

\leavevmode\vadjust pre{\hypertarget{ref-gratz_multidimensional_2004}{}}%
Gratz, Kim L., and Lizabeth Roemer. 2004. {``Multidimensional Assessment
of Emotion Regulation and Dysregulation: Development, Factor Structure,
and Initial Validation of the Difficulties in Emotion Regulation
Scale.''} \emph{Journal of Psychopathology and Behavioral Assessment} 26
(1): 41--54. \url{https://doi.org/10.1023/B:JOBA.0000007455.08539.94}.

\leavevmode\vadjust pre{\hypertarget{ref-gross_individual_2003}{}}%
Gross, James J., and Oliver P. John. 2003. {``Individual Differences in
Two Emotion Regulation Processes: Implications for Affect,
Relationships, and Well-Being.''} \emph{Journal of Personality and
Social Psychology} 85 (2): 348--62.
\url{https://doi.org/10.1037/0022-3514.85.2.348}.

\leavevmode\vadjust pre{\hypertarget{ref-hagerty1995}{}}%
Hagerty, Bonnie M. K., and Kathleen Patusky. 1995. {``Developing a
Measure Of Sense of Belonging:''} \emph{Nursing Research} 44 (1): 9--13.
\url{https://doi.org/10.1097/00006199-199501000-00003}.

\leavevmode\vadjust pre{\hypertarget{ref-houkamau_looking_2015}{}}%
Houkamau, Carla A., and Chris G. Sibley. 2015. {``Looking Māori Predicts
Decreased Rates of Home Wwnership: Institutional Racism in Housing Based
on Perceived Appearance.''} \emph{{PLoS} {ONE}} 10 (3): e0118540.
\url{https://doi.org/10.1371/journal.pone.0118540}.

\leavevmode\vadjust pre{\hypertarget{ref-kessler2002}{}}%
Kessler, R.~C., G. Andrews, L.~J. Colpe, E. Hiripi, D.~K. Mroczek,
S.-L.~T. Normand, E.~E. Walters, and A.~M. Zaslavsky. 2002. {``Short
Screening Scales to Monitor Population Prevalences and Trends in
Non-Specific Psychological Distress.''} \emph{Psychological Medicine} 32
(6): 959--76. \url{https://doi.org/10.1017/S0033291702006074}.

\leavevmode\vadjust pre{\hypertarget{ref-kessler2010}{}}%
Kessler, Ronald C., Jennifer Greif Green, Michael J. Gruber, Nancy A.
Sampson, Evelyn Bromet, Marius Cuitan, Toshi A. Furukawa, et al. 2010.
{``Screening for Serious Mental Illness in the General Population with
the K6 Screening Scale: Results from the WHO World Mental Health (WMH)
Survey Initiative.''} \emph{International Journal of Methods in
Psychiatric Research} 19 (S1): 4--22.
\url{https://doi.org/10.1002/mpr.310}.

\leavevmode\vadjust pre{\hypertarget{ref-mccullough_grateful_2002}{}}%
McCullough, Michael E., Robert A. Emmons, and Jo-Ann Tsang. 2002. {``The
Grateful Disposition: A Conceptual and Empirical Topography.''}
\emph{Journal of Personality and Social Psychology} 82 (1): 112--27.
\url{https://doi.org/10.1037/0022-3514.82.1.112}.

\leavevmode\vadjust pre{\hypertarget{ref-mummendey_social_1999}{}}%
Mummendey, A., and M. Wenzel. 1999. {``Social Discrimination and
Tolerance in Intergroup Relations: Reactions to Intergroup
Difference.''} \emph{Personality and Social Psychology Review: An
Official Journal of the Society for Personality and Social Psychology,
Inc} 3 (2): 158--74. \url{https://doi.org/10.1207/s15327957pspr0302_4}.

\leavevmode\vadjust pre{\hypertarget{ref-muriwai_looking_2018}{}}%
Muriwai, E., C. A. Houkamau, and C. G. Sibley. 2018. {``Looking Like a
Smoker, a Smokescreen to Racism? Māori Perceived Appearance Linked to
Smoking Status.''} \emph{Ethnicity \& Health} 23 (4): 353--66.
\url{https://doi.org/10.1080/13557858.2016.1263288}.

\leavevmode\vadjust pre{\hypertarget{ref-nolen-hoeksema_effects_1993}{}}%
Nolen-hoeksema, Susan, and Jannay Morrow. 1993. {``Effects of Rumination
and Distraction on Naturally Occurring Depressed Mood.''}
\emph{Cognition and Emotion} 7 (6): 561--70.
\url{https://doi.org/10.1080/02699939308409206}.

\leavevmode\vadjust pre{\hypertarget{ref-postmes_single-item_2013}{}}%
Postmes, Tom, S. Alexander Haslam, and Lise Jans. 2013. {``A Single-Item
Measure of Social Identification: Reliability, Validity, and Utility.''}
\emph{The British Journal of Social Psychology} 52 (4): 597--617.
\url{https://doi.org/10.1111/bjso.12006}.

\leavevmode\vadjust pre{\hypertarget{ref-prochaska2012}{}}%
Prochaska, Judith J., Hai-Yen Sung, Wendy Max, Yanling Shi, and Michael
Ong. 2012. {``Validity Study of the K6 Scale as a Measure of Moderate
Mental Distress Based on Mental Health Treatment Need and Utilization:
The K6 as a Measure of Moderate Mental Distress.''} \emph{International
Journal of Methods in Psychiatric Research} 21 (2): 88--97.
\url{https://doi.org/10.1002/mpr.1349}.

\leavevmode\vadjust pre{\hypertarget{ref-rice_short_2014}{}}%
Rice, Kenneth G., Clarissa M. E. Richardson, and Stephen Tueller. 2014.
{``The Short Form of the Revised Almost Perfect Scale.''} \emph{Journal
of Personality Assessment} 96 (3): 368--79.
\url{https://doi.org/10.1080/00223891.2013.838172}.

\leavevmode\vadjust pre{\hypertarget{ref-sengupta2013}{}}%
Sengupta, Nikhil K, Nils Luyten, Lara M Greaves, Danny Osborne, Andrew
Robertson, Colmar Brunton, Gavin Armstrong, and Chris G Sibley. 2013.
{``Sense of Community in New Zealand Neighbourhoods: A Multi-Level Model
Predicting Social Capital.''} \emph{New Zealand Journal of Psychology}
42 (1): 36--45.

\leavevmode\vadjust pre{\hypertarget{ref-sibley2012}{}}%
Sibley, \& Bulbulia, C. G. 2012. {``Healing Those Who Need Healing: How
Religious Practice Affects Social Belonging.''} \emph{Journal for the
Cognitive Science of Religion} 1: 29--45.

\leavevmode\vadjust pre{\hypertarget{ref-sibley2011}{}}%
Sibley, Chris G, Nils Luyten, Missy Purnomo, Annelise Mobberley, Liz W
Wootton, Matthew D Hammond, Nikhil Sengupta, et al. 2011. {``The
Mini-IPIP6: Validation and Extension of a Short Measure of the Big-Six
Factors of Personality in New Zealand.''} \emph{New Zealand Journal of
Psychology} 40 (3): 142--59.

\leavevmode\vadjust pre{\hypertarget{ref-steger_meaning_2006}{}}%
Steger, Michael F., Patricia Frazier, Shigehiro Oishi, and Matthew
Kaler. 2006. {``The Meaning in Life Questionnaire: Assessing the
Presence of and Search for Meaning in Life.''} \emph{Journal of
Counseling Psychology} 53 (1): 80--93.
\url{https://doi.org/10.1037/0022-0167.53.1.80}.

\leavevmode\vadjust pre{\hypertarget{ref-stronge_facebook_2015}{}}%
Stronge, Samantha, Lara M. Greaves, Petar Milojev, Tim West-Newman,
Fiona Kate Barlow, and Chris G. Sibley. 2015. {``Facebook Is Linked to
Body Dissatisfaction: Comparing Users and Non-Users.''} \emph{Sex Roles:
A Journal of Research} 73 (5): 200--213.
\url{https://doi.org/10.1007/s11199-015-0517-6}.

\leavevmode\vadjust pre{\hypertarget{ref-tangney_high_2004}{}}%
Tangney, June P., Roy F. Baumeister, and Angie Luzio Boone. 2004.
{``High Self-Control Predicts Good Adjustment, Less Pathology, Better
Grades, and Interpersonal Success.''} \emph{Journal of Personality} 72
(2): 271--324. \url{https://doi.org/10.1111/j.0022-3506.2004.00263.x}.

\leavevmode\vadjust pre{\hypertarget{ref-tiliouine2006}{}}%
Tiliouine, Habib, Robert A. Cummins, and Melanie Davern. 2006.
{``Measuring Wellbeing in Developing Countries: The Case of Algeria.''}
\emph{Social Indicators Research} 75 (1): 1--30.
\url{https://doi.org/10.1007/s11205-004-2012-2}.

\leavevmode\vadjust pre{\hypertarget{ref-tyler_understanding_1996}{}}%
Tyler, Tom, Peter Degoey, and Heather Smith. 1996. {``Understanding Why
the Justice of Group Procedures Matters: A Test of the Psychological
Dynamics of the Group-Value Model.''} \emph{Journal of Personality and
Social Psychology} 70 (5): 913--30.
\url{https://doi.org/10.1037/0022-3514.70.5.913}.

\leavevmode\vadjust pre{\hypertarget{ref-verbrugge1997}{}}%
Verbrugge, Lois M. 1997. {``A Global Disability Indicator.''}
\emph{Journal of Aging Studies} 11 (4): 337--62.
\url{https://doi.org/10.1016/S0890-4065(97)90026-8}.

\leavevmode\vadjust pre{\hypertarget{ref-williams_cyberostracism_2000}{}}%
Williams, Kipling D., Christopher K. T. Cheung, and Wilma Choi. 2000.
{``Cyberostracism: Effects of Being Ignored over the Internet.''}
\emph{Journal of Personality and Social Psychology} 79 (5): 748--62.
\url{https://doi.org/10.1037/0022-3514.79.5.748}.

\end{CSLReferences}



\end{document}
