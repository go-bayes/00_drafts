% Options for packages loaded elsewhere
\PassOptionsToPackage{unicode}{hyperref}
\PassOptionsToPackage{hyphens}{url}
\PassOptionsToPackage{dvipsnames,svgnames,x11names}{xcolor}
%
\documentclass[
  singlecolumn,
  9pt]{article}

\usepackage{amsmath,amssymb}
\usepackage{iftex}
\ifPDFTeX
  \usepackage[T1]{fontenc}
  \usepackage[utf8]{inputenc}
  \usepackage{textcomp} % provide euro and other symbols
\else % if luatex or xetex
  \usepackage{unicode-math}
  \defaultfontfeatures{Scale=MatchLowercase}
  \defaultfontfeatures[\rmfamily]{Ligatures=TeX,Scale=1}
\fi
\usepackage[]{libertinus}
\ifPDFTeX\else  
    % xetex/luatex font selection
\fi
% Use upquote if available, for straight quotes in verbatim environments
\IfFileExists{upquote.sty}{\usepackage{upquote}}{}
\IfFileExists{microtype.sty}{% use microtype if available
  \usepackage[]{microtype}
  \UseMicrotypeSet[protrusion]{basicmath} % disable protrusion for tt fonts
}{}
\makeatletter
\@ifundefined{KOMAClassName}{% if non-KOMA class
  \IfFileExists{parskip.sty}{%
    \usepackage{parskip}
  }{% else
    \setlength{\parindent}{0pt}
    \setlength{\parskip}{6pt plus 2pt minus 1pt}}
}{% if KOMA class
  \KOMAoptions{parskip=half}}
\makeatother
\usepackage{xcolor}
\usepackage[top=30mm,bottom=30mm,left=20mm,heightrounded]{geometry}
\setlength{\emergencystretch}{3em} % prevent overfull lines
\setcounter{secnumdepth}{-\maxdimen} % remove section numbering
% Make \paragraph and \subparagraph free-standing
\ifx\paragraph\undefined\else
  \let\oldparagraph\paragraph
  \renewcommand{\paragraph}[1]{\oldparagraph{#1}\mbox{}}
\fi
\ifx\subparagraph\undefined\else
  \let\oldsubparagraph\subparagraph
  \renewcommand{\subparagraph}[1]{\oldsubparagraph{#1}\mbox{}}
\fi


\providecommand{\tightlist}{%
  \setlength{\itemsep}{0pt}\setlength{\parskip}{0pt}}\usepackage{longtable,booktabs,array}
\usepackage{calc} % for calculating minipage widths
% Correct order of tables after \paragraph or \subparagraph
\usepackage{etoolbox}
\makeatletter
\patchcmd\longtable{\par}{\if@noskipsec\mbox{}\fi\par}{}{}
\makeatother
% Allow footnotes in longtable head/foot
\IfFileExists{footnotehyper.sty}{\usepackage{footnotehyper}}{\usepackage{footnote}}
\makesavenoteenv{longtable}
\usepackage{graphicx}
\makeatletter
\def\maxwidth{\ifdim\Gin@nat@width>\linewidth\linewidth\else\Gin@nat@width\fi}
\def\maxheight{\ifdim\Gin@nat@height>\textheight\textheight\else\Gin@nat@height\fi}
\makeatother
% Scale images if necessary, so that they will not overflow the page
% margins by default, and it is still possible to overwrite the defaults
% using explicit options in \includegraphics[width, height, ...]{}
\setkeys{Gin}{width=\maxwidth,height=\maxheight,keepaspectratio}
% Set default figure placement to htbp
\makeatletter
\def\fps@figure{htbp}
\makeatother
% definitions for citeproc citations
\NewDocumentCommand\citeproctext{}{}
\NewDocumentCommand\citeproc{mm}{%
  \begingroup\def\citeproctext{#2}\cite{#1}\endgroup}
\makeatletter
 % allow citations to break across lines
 \let\@cite@ofmt\@firstofone
 % avoid brackets around text for \cite:
 \def\@biblabel#1{}
 \def\@cite#1#2{{#1\if@tempswa , #2\fi}}
\makeatother
\newlength{\cslhangindent}
\setlength{\cslhangindent}{1.5em}
\newlength{\csllabelwidth}
\setlength{\csllabelwidth}{3em}
\newenvironment{CSLReferences}[2] % #1 hanging-indent, #2 entry-spacing
 {\begin{list}{}{%
  \setlength{\itemindent}{0pt}
  \setlength{\leftmargin}{0pt}
  \setlength{\parsep}{0pt}
  % turn on hanging indent if param 1 is 1
  \ifodd #1
   \setlength{\leftmargin}{\cslhangindent}
   \setlength{\itemindent}{-1\cslhangindent}
  \fi
  % set entry spacing
  \setlength{\itemsep}{#2\baselineskip}}}
 {\end{list}}
\usepackage{calc}
\newcommand{\CSLBlock}[1]{\hfill\break#1\hfill\break}
\newcommand{\CSLLeftMargin}[1]{\parbox[t]{\csllabelwidth}{\strut#1\strut}}
\newcommand{\CSLRightInline}[1]{\parbox[t]{\linewidth - \csllabelwidth}{\strut#1\strut}}
\newcommand{\CSLIndent}[1]{\hspace{\cslhangindent}#1}

\usepackage{cancel}
\usepackage[noblocks]{authblk}
\renewcommand*{\Authsep}{, }
\renewcommand*{\Authand}{, }
\renewcommand*{\Authands}{, }
\renewcommand\Affilfont{\small}
\usepackage{cancel}
\makeatletter
\@ifpackageloaded{caption}{}{\usepackage{caption}}
\AtBeginDocument{%
\ifdefined\contentsname
  \renewcommand*\contentsname{Table of contents}
\else
  \newcommand\contentsname{Table of contents}
\fi
\ifdefined\listfigurename
  \renewcommand*\listfigurename{List of Figures}
\else
  \newcommand\listfigurename{List of Figures}
\fi
\ifdefined\listtablename
  \renewcommand*\listtablename{List of Tables}
\else
  \newcommand\listtablename{List of Tables}
\fi
\ifdefined\figurename
  \renewcommand*\figurename{Figure}
\else
  \newcommand\figurename{Figure}
\fi
\ifdefined\tablename
  \renewcommand*\tablename{Table}
\else
  \newcommand\tablename{Table}
\fi
}
\@ifpackageloaded{float}{}{\usepackage{float}}
\floatstyle{ruled}
\@ifundefined{c@chapter}{\newfloat{codelisting}{h}{lop}}{\newfloat{codelisting}{h}{lop}[chapter]}
\floatname{codelisting}{Listing}
\newcommand*\listoflistings{\listof{codelisting}{List of Listings}}
\makeatother
\makeatletter
\makeatother
\makeatletter
\@ifpackageloaded{caption}{}{\usepackage{caption}}
\@ifpackageloaded{subcaption}{}{\usepackage{subcaption}}
\makeatother
\ifLuaTeX
  \usepackage{selnolig}  % disable illegal ligatures
\fi
\IfFileExists{bookmark.sty}{\usepackage{bookmark}}{\usepackage{hyperref}}
\IfFileExists{xurl.sty}{\usepackage{xurl}}{} % add URL line breaks if available
\urlstyle{same} % disable monospaced font for URLs
\hypersetup{
  pdftitle={Causal Diagrams for the Evolutionary Human Sciences: A Practical Guide},
  pdfauthor={Joseph A. Bulbulia},
  pdfkeywords={Directed Acyclic Graph, Causal
Inference, Confounding, Feedback, Interaction, Mediation, Moderation, Panel},
  colorlinks=true,
  linkcolor={blue},
  filecolor={Maroon},
  citecolor={Blue},
  urlcolor={Blue},
  pdfcreator={LaTeX via pandoc}}

\title{Causal Diagrams for the Evolutionary Human Sciences: A Practical
Guide}


  \author{Joseph A. Bulbulia}
            \affil{%
                  Victoria University of Wellington, New Zealand, School
                  of Psychology, Centre for Applied Cross-Cultural
                  Research
              }
      
\date{2023-11-24}
\begin{document}
\maketitle
\begin{abstract}
This article offers practical advice for creating causal diagrams. It
recommends aligning a graph's spatial layout with causation's temporal
order. Because causal graphs only have utility as part of the framework
of theory and assumptions that define causal data science, I begin by
reviewing this framework. I then consider how, within this framework,
causal diagrams may be used to uncover structural sources of bias,
focussing on confounding bias, and illustrating the benefits of
chronological hygiene in one's graph. I conclude by using causal
diagrams to elucidate the widely misunderstood concepts causal
interaction, mediation, and dynamic longitudinal feedback, again
focussing on the benefits of chronological ordering. Overall, this guide
hopes to better equip evolutionary human scientists with understanding
at skills to enhance the rigour and clarity of their causal inferences.
\end{abstract}
\subsection{Introduction}\label{introduction}

Correlation does not imply causation. This adage is widely known.
Nevertheless, many human scientists report manifest correlations and use
hedging language that implies causation. I have been guilty. However,
such reporting typically lacks justification. Making matters worse,
widely adopted strategies for confounding control such as indiscriminate
co-variate adjustment will often enhance bias
(\citeproc{ref-mcelreath2020}{McElreath 2020}). Across many human
sciences, including the evolutionary human sciences, persistent
confusion in the analysis and reporting of correlations has greatly
impeded scientific progress (\citeproc{ref-bulbulia2022}{Bulbulia
2022}).

There is hope. First, the advances of the open science movement have
demonstrated that greater attention to the `replication crisis' across
the experimental human sciences can bring considerable improvements to
the quality and integrity of experimental research. Many corrective
practices in that movement have become normative. Second, several
decades of intensive research in the health sciences, computer science,
and economics have yielded conceptual clarifications and rigorous
analytic toolkits for causal inference in observational settings.
Although these fields are still evolving, a substantial foundation for
causal data science already exists.\footnote{foundations} Given these
precedents, we should be optimistic that rapid progress in causal
inference across all the human sciences is a feasible and achievable
goal. The articles in this special issue of \emph{Evolutionary Human
Sciences} are offer testimony for this hope.

{[}foundations{]}: This foundational work, initiated by Neyman
(\citeproc{ref-neyman1923}{\textbf{neyman1923?}}), was further developed
by Rubin, Robins, Pearl, and their students, and provides a robust basis
for future advancements (\citeproc{ref-hernuxe1n2023}{Hernán and Robins
2023}; \citeproc{ref-pearl1995}{Pearl 1995};
\citeproc{ref-robins1986}{Robins 1986}; \citeproc{ref-rubin1976}{Rubin
1976}).

Causal diagrams, known as `directed acyclic graphs' or `DAGs,' are
indispensable inferential tools in the methodologies of causal
inference. These tools rest on a robust system of formal mathematical
proofs, instilling confidence in their use. However, they do not require
mathematical training. The tool is broadly accessible (for the sighted).
They serve several purposes: first, by making causal assumptions
explicit, they enhance the transparency of the analysis; second, they
help in identifying structural sources of bias that can undermine
inference; third, they provide valuable guidance on data collection
requirements.

Causal diagrams only acquire their significance when integrated within
the broader theoretical frameworks and workflows of Causal Data Science.
This discipline distinguishes itself from traditional data science by
focusing the tasks of estimating of pre-specified counterfactual
contrasts, or `estimands.' In this approach, counterfactual scenarios
are simulated from data under explicit assumptions and then
quantitatively compared. Therefore, Causal Data Science can be viewed as
a form of `counterfactual data science' or `full data science' ---
`full' in the sense that the data we observe provide only partial
insights into the targeted causal quantities and uncertainties
researchers hope to consistently quantify (reference:
(\citeproc{ref-bulbulia2023}{Bulbulia \emph{et al.} 2023}); see also:
(\citeproc{ref-ogburn2021}{\textbf{ogburn2021?}})). Using causal
diagrams without a thorough understanding of their role in Causal Data
Science could inadvertently worsen the causality crisis by fostering
misguided confidence where none is due.

In this work, I aim to offer readers of \emph{Evolutionary Human
Science} practical guidance on creating causal diagrams in ways that
mitigate the risk of overreaching.

\textbf{Part 1} introduces the core elements of Causal Data Science,
emphasising the fundamental assumptions necessary for obtaining valid
causal inferences from observational data. Although this overview is
brief, it is vital for researchers using causal diagrams to familiarise
themselves with these foundational concepts (see also other references
in this issue).

\textbf{Part 2} introduces \textbf{chronologically ordered causal
diagrams} and their applications in addressing confounding bias. We
discover that maintaining chronological order in the spatial layout of
these diagrams clarifies structural biases, enhances strategies for
identifying causal effect estimates, and indicates where causal
inferences may remain elusive. Chronologically ordered causal diagrams
also underscore the value of collecting repeated measures over time,
thereby improving research planning. Although, chronological ordering is
not strictly essential for the utility of a causal diagram, the examples
we consider demonstrate their advantages in common scenarios.

\textbf{Part 3} uses chronologically ordered causal diagrams to
demystify complex concepts such as causal interaction, mediation, and
longitudinal data analysis. Here, attention to chronological sequencing
in the diagram's layout, coupled with a clear understanding of the
targeted fcausal estimands, may greatly advances scientific
comprehension. This approach enables us to more effectively formulate
and address causal questions in areas where confusing statistical
traditions like structural equation modelling presently hold sway.

There are numerous outstanding resources on causal diagrams available,
which I highly recommend to readers (\citeproc{ref-barrett2021}{Barrett
2021}; \citeproc{ref-cinelli2022}{Cinelli \emph{et al.} 2022};
\citeproc{ref-greenland1999}{Greenland \emph{et al.} 1999};
\citeproc{ref-hernuxe1n2023}{Hernán and Robins 2023};
\citeproc{ref-mcelreath2020}{McElreath 2020};
\citeproc{ref-pearl2009}{Pearl 2009a}; \citeproc{ref-rohrer2018}{Rohrer
2018}; \citeproc{ref-suzuki2020}{Suzuki \emph{et al.} 2020}).\footnote{An
  excellent resource is Miguel Hernán's free online course, here:
  \url{https://pll.harvard.edu/course/causal-diagrams-draw-your-assumptions-your-conclusions}.}
This article aims to build upon these previous works by providing
additional conceptual context and underscoring the importance of
chronological hygiene in pinpointing structural sources of confounding
bias. It also seeks to clarify common misunderstandings within the
evolutionary sciences regarding `interaction', `mediation', and the
analysis of complex longitudinal data.

\subsection{Part 1. Overview of Causal Data
Science}\label{part-1.-overview-of-causal-data-science}

The critical first step in causal inference is formulating a
well-defined causal question(\citeproc{ref-hernuxe1n2016}{Hernán
\emph{et al.} 2016}). Causal diagrams appear later in our analytic
workflow, when we consider whether and how the data enable inference
about the pre-specified causal question. This section introduces
fundamental concepts in Causal Data Science, and locates the place of
causal diagrams within a larger workflow that moves from stating a
causal question to answering it with data.

\subsubsection{The Fundamental Problem of Causal
Inference}\label{the-fundamental-problem-of-causal-inference}

Consider an intervention, \(A\), and its effect, \(Y\). We say that
\(A\) causes \(Y\) if altering \(A\) would lead to a change in \(Y\)
(\citeproc{ref-hume1902}{Hume 1902}; \citeproc{ref-lewis1973}{Lewis
1973}). If altering \(A\) would not lead to a change in \(Y\), then we
say that \(A\) has no causal effect on \(Y\).

The objective in causal inference is to measure the difference in a
specifically defined outcome \(Y\) when subjected to different levels of
a clearly defined intervention \(A\). Commonly, we refer to these
interventions as `exposures' or `treatments;' we refer to the resulting
effects as `potential outcomes.'

Let us assume that \(A\) can exist in only two states: \(A = 0\) or
\(A = 1\). We denote the potential outcome when \(A\) is set to 0 as
\(Y(0)\) and when \(A\) is set to 1 as \(Y(1)\).\footnote{There are
  various conventions for representing potential outcomes, such as
  \(Y^a\) and \(Y_a\). To simplify, we omit subscripts, using
  \(Y|A = 0\) or \(Y|A = 1\) instead of \(Y_i|A_i = 0\) or
  \(Y_i|A_i = 1\). Additionally, we omit an implicit time subscript; for
  a time index \(t\), the notation would be \(Y_{i,t+1}|A_{i,t} = 0\) or
  \(Y_{i,t+1}|A_{i,t} = 1\). We employ this simplified notation for
  legibility, opting for more precise notation when the context demands
  clarity.}

To quantitatively evaluate whether the altering \(A\) makes a difference
to the outcome \(Y\), we must compute contrasts for the potential
outcomes under different exposures. For instance, \(Y(1) - Y(0)\)
calculates this contrast under a binary exposure on the difference
scale, while \(\frac{Y(1)}{Y(0)}\) does so on the ratio scale. To
quantitatively evaluate evidence for causality requires specifying an
intervention, here \(A = \{0,1\}\); specifying an the potential outcome
under different realisations of the intervention, here: \(Y(0)\) and
\(Y(1)\); and specifying a scale of contrast, such as the difference
scale or the ratio scale. Importantly, we must specify some unit or set
of units on which the interventions to be evaluated occur, and are to be
measured. Doing so reveals that causal data science cannot rely on
ordinary data science.

History is characterised by its unidirectional progression, a
fundamental aspect of physics that presents a significant challenge in
causal data science. At any given moment, for any unit under
consideration, only one level of an exposure can be realised. Consider
hypothetical questions such as `What if Isaac Newton had not observed
the falling apple?' or `What if Leonardo da Vinci had never pursued
art?' or `What if Archduke Ferdinand had not been assassinated?' These
questions underscore our inability to access alternate realities where
these events unfolded differently. This limitation applies to all
individual units experiencing any level of an exposure; each unit can
either experience \(Y|A = 1\) or \(Y|A = 0\), but never both
simultaneously. As a result, we cannot directly calculate the difference
between \(Y(1)\) and \(Y(0)\), or their ratio, from our available data.
In every case, at least one of the outcomes necessary to determine a
causal effect at the individual level remains counterfactual. This
situation gives rise to what is known as the `fundamental problem of
causal inference': our constraint to observing only one treatment state
for each individual at a time (\citeproc{ref-holland1986}{Holland 1986};
\citeproc{ref-rubin1976}{Rubin 1976}). Consequently, causal data science
faces a unique type of missing data problem, where the `full data'
needed for causal contrasts are inherently incomplete, missing at least
half of their values (\citeproc{ref-edwards2015}{Edwards \emph{et al.}
2015}; \citeproc{ref-westreich2015}{Westreich \emph{et al.} 2015}). This
challenge is distinct from typical missing data scenarios where the data
could have been recorded but were not. The missing information crucial
for computing causal contrasts is intrinsically linked to the
irreversible nature of time.

\subsubsection{Specifying Causal
Effects}\label{specifying-causal-effects}

In typical scenarios, computing individual causal effects is not
feasible. However, under certain assumptions, it is possible to credibly
calculate average causal effects. We may obtain average treatment
effects by contrasting groups that have received different levels of a
treatment. The average treatment effect (ATE) on a difference scale is
represented as:

\[
ATE  = \mathbb{E}[Y(1)] - \mathbb{E}[Y(0)]
\]

Here, \(\mathbb{E}\) denotes the average response of all individuals
within an exposure group, and \(Y(1)\) and \(Y(0)\) represent the
potential outcomes under interventions \(A = 1\) and \(A = 0\),
respectively.

It is important to note that simply aggregating across groups that
received different interventions and calculating the difference in their
average outcomes does not fully address the fundamental problem of
causal inference.The fundamental missing data challenge persists. We can
expand our calculation of the average treatment effect (ATE) to reveal
the missing data challenge:

\[
ATE = \underbrace{\big(\mathbb{E}[Y(1)|A = 1]\big)}_{\text{observed}} - \underbrace{\big(\mathbb{E}[Y(0)|A = 0]\big)}_{\text{observed}} + \underbrace{\big(\mathbb{E}[Y(1)|A = 0]\big)}_{\text{unobserved}} - \underbrace{\big(\mathbb{E}[Y(0)|A = 1]\big)}_{\text{unobserved}}
\]

The ATE must combine observable outcomes with hypothetical
outcomes---those that would have occurred under a different exposure,
and these exposures are never directly observed.

To understand how causal data science can derive valid inferences
without directly observing counterfactuals, consider the benefits of
randomised experiments. When treatments are randomly assigned, and
randomisation is effective, the outcomes under different treatment
conditions should, in theory, be identical. If there are differences in
average outcomes between treated and untreated groups in a randomized
setup, these differences can be attributed to the treatment itself. That
is, randomisation allows us to infer that the treatment averages by
group would have been identical. Although randomisation can fail, it
provides a means to identify group-level causal effects by eliminating
other potential explanations for the observed differences. For this
reason, we should prefer experiments for addressing scientific questions
that can be addressed by them.

Regrettably, many scientific questions, particularly those in the
evolutionary human sciences, cannot be addressed through experimental
means. This limitation is acutely felt when researchers confront `what
if?' scenarios rooted in the unidirectional nature of human history. In
observational settings, where the random assignment of individuals to
groups is not feasible, achieving a balance across variables that might
account for treatment-level differences presents a significant
challenge. We next consider fundamental assumptions essential for
deriving valid group-level causal contrasts from data, noting where
causal diagrams appear within a workflow that proceeds from asking a
pre-specified causal question to answering it. We will assume that we
have stated a clear causal question, for which there is a well-defined
exposure and outcome, and a clearly identified population for which the
question is targeted. Before attempting any statistical models, we must
satisfy ourselves that the following assumptions are not violated.

\subsubsection{Fundamental Identification
Assumptions}\label{fundamental-identification-assumptions}

There are three fundamental identification assumptions that must be
satisfied to estimate causal effects with data.

\paragraph{Fundamental Identification Assumption 1: Causal Consistency
and Treatment Effect
Heterogeneity}\label{fundamental-identification-assumption-1-causal-consistency-and-treatment-effect-heterogeneity}

The causal consistency assumption posits that for any given level of
exposure, \(A=a\), the observed outcome, \(Y|A=a\), is interchangeable
with the counterfactual outcome. To illustrate,let \(i\) represent an
individual. The observed outcome when treatment is \(A_i = a\) is
denoted as \(Y_i^{observed}|A_i = a\). Under causal consistency, this
observed outcome corresponds to one of the counterfactual outcomes
necessary for causal analysis:

\[
Y_i^{observed}|A_i = 
\begin{cases} 
Y_i(a^*) & \text{if } A_i = a^* \\
Y_i(a) & \text{if } A_i = a
\end{cases}
\]

This assumption implies that the observed outcome at a specific exposure
level mirrors the counterfactual outcome for that individual. Although
it might seem straightforward to equate an individual's observed outcome
with their counterfactual outcome, in observational studies, treatment
effects often vary, presenting challenges in satisfying this assumption.

Consider the question of whether a society's beliefs in Big Gods affects
its development of Social Complexity. Historians anthropologists report
that such beliefs vary over time and across cultures in their intensity,
interpretations, institutional management, and ritual embodiments. Such
variation in content and setting may significantly influence social
complexity. Moreover the treatments as they are realised in one society
might affect the treatments realised in other societies. However, to
apply the causal consistency assumption such treatments must be
independent. Such variation underscores the need for careful
consideration of treatment heterogeneity.

The theory of causal inference under multiple versions of treatment
addresses the challenge of treatment heterogeneity
(\citeproc{ref-vanderweele2009}{VanderWeele 2009a},
\citeproc{ref-vanderweele2018}{2018};
\citeproc{ref-vanderweele2013}{VanderWeele and Hernan 2013}). It
formally proves that if the treatment variations (\(K\)) are
conditionally independent of the outcome \(Y(k)\) given covariates
\(L\). Where \(\coprod\) denotes independence, if

\[
K \coprod Y(k) | L
\]

Then we may consistently estimate of causal effects even with varied
treatments. In such settings, we may think of \(K\) acts as a
``coarsened indicator'' for \(A\) such that we obtain an average effect
estimate for the multiple treatment versions \(K\) on \(Y(k)\).

While the theory of causal inference under multiple versions of
treatment provides a formal solution to the problem of treatment effect
heterogeneity, interpreting the resulting causal effect estimates under
this theory can be challenging. A common example in the health sciences
involves interpreting the causal effects of change in Body Mass Index
(BMI) on health. Notably, weight loss can occur through various methods,
each with different health implications. Some methods, such as regular
exercise or a calorie-reduced diet, are generally beneficial for health.
However, weight loss might also result from adverse conditions such as
infectious diseases, cancers, depression, famine, or even amputations,
which are not beneficial to health. Although causal effects can be
consistently estimated while adjusting for covariates \(L\), the true
nature and implications of the changes in BMI might remain unclear. This
uncertainty highlights the need for precise and well-defined causal
questions, such as weight loss achieved specifically through aerobic
exercise over a period of at least five years. This level of specificity
helps ensure that the causal estimates we obtain are not only
statistically sound but also meaningful and relevant to the research
question at hand (for discussion see:
(\citeproc{ref-hernuxe1n2008}{Hernán \emph{et al.} 2008};
\citeproc{ref-hernuxe1n2022a}{Hernán \emph{et al.} 2022};
\citeproc{ref-murray2021a}{Murray \emph{et al.} 2021}).

Beyond interpretation, there is the additional problem that we cannot
really know whether the measured covariates \(L\) are sufficient to
render the multiple-versions of treatment independent of the
counterfactual outcomes. This problem is especially acute when there are
spill-over effects, such when treatment-effects are relative to the
density and distribution of treatment-effects in a population
{[}CITE{]}. For this reason, causal data science must rely heavily on
sensitivity analyses (VanderWeele
(\citeproc{ref-vanderweele2019}{2019});).

In summary, what seemed initially to be a near truism -- that each units
observed outcome may be assumed to correspond to that unit's
counterfactual outcome -- turns out to be a strong assumption. In many
settings, causal consistency should be presumed unrealistic until proven
tenable.

For now, we note that the causal consistency assumption provides a
starting point for recovering the missing counterfactuals essential for
computing causal contrasts. It achieves this by identifying half of
these counterfactuals directly from observed data. The concept of
exchangeability, which we will explore next, offers a means to derive
the remaining half.

\subsubsection{Fundamental Identification Assumtion 2: Conditional
Exchangeability (No Unmeasured
Confounding)}\label{fundamental-identification-assumtion-2-conditional-exchangeability-no-unmeasured-confounding}

We satisfy conditional exchangeability when the treatment groups are
equivalent in variables that could affect potential outcomes. In
experimental designs, random assignment facilitates conditional
exchangeability. In observational studies, more effort is required. We
must control for covariates that could account for observed correlations
between \(A\) and \(Y\) in the absence of a causal effect of \(A\) on
\(Y\) for every observed.

Let \(L\) denote the set of covariates necessary to ensure this
conditional independence. Let \(\coprod\) again denote independence. We
satisfy conditional exchangeability when:

\[
Y(a) \coprod A | L \quad \text{or equivalently} \quad A \coprod Y(a) | L
\]

Where this, and the other fundamental assumptions hold, we may compute
the average treatment effect (ATE) on the difference scale:

\[
ATE = \mathbb{E}[Y(a^*) | L] - \mathbb{E}[Y(a) | L]
\]

In domains like cultural evolution, where experimental control is often
impractical, causal inferences hinge on the plausibility of satisfying
this `no unmeasured confounding' assumption. Lacking randomisation,
however, causal data science must turn to sensitivity analyses (Appendix
1 critiques such assumptions for estimating causal effects of beliefs in
Big Gods on social complexity).

Importantly, \emph{the primary purpose of a causal diagram within a
causal inference workflow is to evaluate conditional exchangeability.}
Causal diagrams represent crucial structural assumptions that are
necessary for achieving balance in the confounders across levels of the
exposure our pre-specified causal estimand contrasts. It is important to
recognise that in this setting, causal diagrams are designed to
\emph{highlight only those aspects pertinent to the assessment of
`no-unmeasured confounding.'} A common mistake is to over-elaborate a
causal diagram, risking loss of focus on the structural biases that are
essential to understand. Overly detailed causal diagrams may obscure
rather than clarify the underlying structural sources of bias, and for
this reason should be avoided.

\subsubsection{Fundamental Identification Assumption 3:
Positivity}\label{fundamental-identification-assumption-3-positivity}

The positivity assumption is met when there exists a non-zero
probability for each level of exposure within every level of covariates
needed to ensure conditional exchangeability. This implies that for each
stratification of every covariate, the likelihood of each exposure value
must exceed zero. Where \(A\) is the exposure and \(L\) a vector of
covariates, positivity is achieved only if:

\[
0 < \Pr(A = a | L = l) < 1, ~ \forall a \in A, ~ \forall l \in L
\]

Two forms of positivity violations exist:

\begin{enumerate}
\def\labelenumi{\arabic{enumi}.}
\item
  \textbf{Random non-positivity:} which occurs when an exposure is
  theoretically possible, but specific exposure levels are not
  represented in the data. Notably, random non-positivity is the only
  identifiability assumption verifiable with data.
\item
  \textbf{Deterministic non-positivity:} which occurs when the exposure
  is implausible by nature. For instance, a hysterectomy in biological
  males would appear biologically implausible.
\end{enumerate}

Ensuring positivity often presents practical challenges. Consider
estimating the effects of church attendance on charity. The objective is
to assess the one-year impact on charitable donations following a shift
from no church attendance to weekly attendance. Assume access to
extensive panel data, tracking 20,000 individuals over three years. If
the natural transition rate from no attendance to weekly attendance is
one in a thousand annually, the effective sample for the treatment
condition dwindles to 20. Despite abundant data, random non-positivity
can significantly hinder valid inference.

Note that where positivity is violated, causal diagrams offer limited
assistance to causal inference because causal inference is not supported
by the data.

\subsubsection{Data and Model Considerations in Causal
Inference}\label{data-and-model-considerations-in-causal-inference}

Beyond the three fundamental identification assumptions that must be
satisfied to estimate causal effects with data, there are numerous
practical consideration that enter into any causal inference workflow.

\paragraph{Measurement error}\label{measurement-error}

Measurement error refers to the discrepancy between the true value of a
variable and the value that is observed or recorded. This error can
arise from a variety of sources, including instrument calibration
issues, respondent misreporting, or coding errors. It is essential to
understand and address measurement error as it can significantly distort
the analysis and lead to misleading conclusions.

Measurement error can be classified into two main types: systematic (or
biased) and random (or unbiased).

\textbf{Random measurement error:} occurs when fluctuations in the
measurement process and do not consistently bias the data in one
direction. While random measurement errors can increase the variability
of data and reduce statistical power, they do not typically introduce
bias in estimates of causal effects when there are no such effects.
However, random measurement error can attenuate bias when there are true
causal effects, where the estimated effect of an exposure on an outcome
is systematically weakened.

\textbf{Systematic measurement error:} occurs when the measurements
deviate from the true value in a consistent direction. Systematic errors
can lead to biased estimates of causal effects, as they consistently
overestimate or underestimate true causal magnitudes. Here again causal
diagrams can be useful {[}CITE OTHER PAPER{]}

The best approach to handling measurement error is to improving data
quality but this is not always possible, and we must perform sensitivity
analyses {[}CITE{]}

In other work, I describe how to use causal diagrams to examine the
structural sources of bias that may arise from different forms of
measurement error {[}CITE{]}. We will not consider this use here {[}CITE
OTHER PAPER{]}. For now, it is important to emphasise that the simple
causal diagrams with arrows between variables typically abstract away
from biases that arise from measurement error, and such simplicity can
be a source of false confidence.\footnote{The careful reader will note a
  tension. Addressing structural sources of bias requires simple causal
  diagrams. Such diagrams do not capture the threats to inference
  arising from measurement, which requires more complicated causal
  diagrams. I follow Hernán and Robins in advising a two step approach
  in which authors draft separate diagrams to handle separate threats to
  valid causal inference (\citeproc{ref-hernuxe1n2023}{Hernán and Robins
  2023}).}

\paragraph{Considerations of Selection
Bias}\label{considerations-of-selection-bias}

Selection bias arises when the sample that is observed is not
representative of the population for which causal inference is intended.
There are two primary forms: bias arising to initial selection and bias
resulting from attrition or non-response.

\textbf{Selection prior to observation:} occurs when the process of
choosing individuals or units for the study leads to a sample that is
not representative of the target population. It may occur due to
specific inclusion or exclusion criteria or through non-random selection
methods. This form of bias can introduce systematic differences between
the treatment and control groups, affecting generalisability. In this
setting, the quantities we obtain from causal data science might not not
apply as we think.

\textbf{Attrition/non-response bias:} occurs post-selection, often
during the course of a study. Attrition bias arises when participants or
unit drop out of the study, and their dropout is related to both the
treatment and the outcome. Non-response bias, similarly, occurs when
certain subjects do not respond to surveys or follow-up, and this
non-response is correlated with the treatment and outcome. Both forms of
bias can lead to skewed results, as the final sample may differ
significantly from the initial sample in crucial aspects related to the
study's focus. This bias cannot be addressed by conditioning directly on
\(L\).

Causal diagrams can be useful in diagnosing sources of selection bias
{[}CITE OTHER WORK{]}. Here we limit the application of causal diagrams
for understanding confounding bias. However, it is imperative to
recognise that, similar to measurement error, selection bias can
substantially distort causal inference.

\paragraph{Modelling Assumptions}\label{modelling-assumptions}

After we have satisfied the fundamental and practical assumptions
required for valid causal inference, we must eventually derive a
estimate of our pre-specified causal contrasts from data. In statistical
analysis, human scientists predominantly use parametric models,
characterised by predetermined functional forms and distributional
assumptions. This reliance creates susceptibility to model
misspecification, which can manifest in various detrimental ways:

\begin{enumerate}
\def\labelenumi{\arabic{enumi}.}
\item
  \textbf{Introduction of bias:} inaccurate specifications in parametric
  models can lead to biased causal effect estimates. Such bias emerges
  when the actual inter-variable relationships are more intricate or
  divergent than the model's assumptions.
\item
  \textbf{Overstated precision:} a misaligned model might suggest
  unwarranted precision, typically by miscalculating parameter standard
  errors, leading to misplaced confidence in the findings.
\item
  \textbf{Concealment of Underlying Flaws:} model misspecification can
  deceptively align well with data, yet fail to truly represent the
  causal framework. This highlights the limitations of heavily relying
  on goodness-of-fit metrics and the importance of more comprehensive
  evaluations.
\end{enumerate}

To mitigate these issues, several strategies are advisable. Rigorous
diagnostic checks are essential to identify breaches in model
assumptions, encompassing tests for linearity, homoscedasticity, and
outlier presence. Furthermore, when data structures are ambiguous,
adopting non-parametric or semi-parametric methods can be more
effective, as these allow for greater adaptability to the complexities
of causal relationships. The ongoing evolution of machine learning
algorithms and doubly robust estimators, which model both the exposure
and outcome, offers promise. These methods can yield valid estimates
even if only one of the models is accurate. However, sensitivity
analyses remain critical for verifying inference robustness under
various model assumptions. Nonetheless, despite all efforts for
robustness, risks of invalidity persist (see discussions in Hoffman
\emph{et al.} (\citeproc{ref-hoffman2022}{2022});
(\citeproc{ref-vansteelandt2022}{\textbf{vansteelandt2022?}});
(\citeproc{ref-muuxf1oz2012}{\textbf{muñoz2012?}}); Díaz \emph{et al.}
(\citeproc{ref-duxedaz2021}{2021}); Williams and Díaz
(\citeproc{ref-williams2021}{2021});
(\citeproc{ref-wager2018}{\textbf{wager2018?}});
(\citeproc{ref-cui2020}{\textbf{cui2020?}})).

The critical takeaway is that even effective causal diagrams do not
guarantee immunity to model misspecification. Ensuring correct model
specification remains a significant challenge in robust causal
inference. or deeper scrutiny.

\subsection{Summary of Part 1: Causal Data Science in Causal
Inference}\label{summary-of-part-1-causal-data-science-in-causal-inference}

Causal inference in data science commences with a clearly defined causal
question. We use causal diagrams as part of an analytic workflow to
enable data-based inference afor a pre-specified causal question. This
workflow starts by posing a clearly defined causal question

When posing a causal question, we typically seek to measure the effect
of an intervention (exposure or treatment) \(A\) on an outcome \(Y\).
The central challenge, called the `fundamental problem of causal
inference,' arises from the inherent limitation of observing only one
treatment state per individual at any given time. It is thus typically
not possible to directly observe all potential outcomes needed to
estimate individual-level causal effects directly from data.

Although individual causal effects are challenging to compute, average
causal effect estimates within levels exposures may be obtained when
certain assumptions are satisfied. For examples, the average treatment
effect (ATE) is a measure used to contrast groups with different
treatment levels. However, this approach still encounters the
fundamental missing data challenge, necessitating the combination of
observed and hypothetical outcomes.

Three key assumptions must be met for credible causal inference: causal
consistency (outcomes at a given exposure level match counterfactual
outcomes), conditional exchangeability (no unmeasured confounding), and
positivity (non-zero probability of each exposure level across every
covariate stratification). Each assumption must be satisfied to ensure
valid causal inference.

We furthermore noted that numerous practical considerations influence
our confidence in our causal inference, including the prospects of
measurement error, selection bias, and model misspecification. These
factors can also dramatically afffect the accuracy and applicability of
our causal conclusions. Addressing these issues involves understanding
the nature of the errors, ensuring representativeness of the sample, and
employing robust statistical models and sensitivity analyses.

Within the workflows of causal data science, we noted that causal
diagrams find their primary utility in helping researchers to identify
structural sources of bias that may compromise the conditional
exchangeability assumption.

It should be clear that the demands of causal data science differ from
the familiar methodologies of standard data science. Because causal data
science must quantitatively estimate contrasts for unobserved
counterfactuals, it is not sufficient to commence with causal diagrams
and proceed directly to modelling patterns in the observed data.
Instead, causal inference demands a meticulous, multi-step workflow.
While causal diagrams are indeed valuable tools, their utility is fully
realised only when they are integrated into this comprehensive and
methodical workflow. Having described essential features of this
workflow, we are now ready to show how causal diagrams may be used to
clarify structural sources of bias.

\subsection{Part 2. Applications of Chronologically Ordered Causal
Diagrams for Understanding Confounding
Bias}\label{part-2.-applications-of-chronologically-ordered-causal-diagrams-for-understanding-confounding-bias}

In this section, I review the application of chronologically ordered
causal diagrams to assessing structural sources of confounding bias in
causal inference (\citeproc{ref-greenland1999}{Greenland \emph{et al.}
1999}; \citeproc{ref-pearl1995}{Pearl 1995},
\citeproc{ref-pearl2009}{2009a}). Causal diagrams are powerful tools for
identifying the conditions under which causal effects can be reliably
estimated from data. Moreover organising them chronologically helps to
avoid inferential over-reaching by making the sources of bias clearer,
and by providing clear directives for data collection. We begin with
essential terminology (Appendix 2 contains a longer list of terms).

\textbf{Nodes}: represent variables or events in a causal diagram. These
will be depicted using circles.

\textbf{Arrows}: represent assumed directions of influence or causation
between variables. An arrow pointing from one node to another implies
that the former influences or causes the latter.

\textbf{Red Arrows}: below, I use a red arrow to describe a path that
leading to bias in the exposure \(\to\) outcome relationship.

\textbf{Dashed Arrows}: below, I use a dashed black arrow to describe a
path whose bias has been reduced by a conditioning strategy but which
nevertheless remains open.

\textbf{Variable Naming Conventions}: In the context of this discussion,
we use the following notation to denote different elements of causal
diagrams:

\begin{itemize}
\tightlist
\item
  \(L\): denotes variables that may potentially lead to confounding
  bias.
\item
  \(A\): represents the treatment or intervention of interest studied.
\item
  \(Y\): signifies the outcome of interest.
\item
  \(U\): denotes an unmeasured confounder.
\end{itemize}

\textbf{Adjustment set}: a collection of variables that we either
condition upon or deliberately avoid conditioning upon to block all
paths between the exposure and the outcome in the causal diagram
(\citeproc{ref-pearl2009}{Pearl 2009a}).

\textbf{Confounder}: a member of an adjustment set. Importantly, a
variable is a ``confounder'' in relation to a specific adjustment set.

\textbf{Modified Disjunctive Cause Criterion}: I adopt a \emph{Modified
Disjunctive Cause Criterion} for controlling for confounding
(\citeproc{ref-vanderweele2019}{VanderWeele 2019}). According to this
criterion, a member of any set of variables that can reduce or remove
the bias caused by confounding is deemed a member of this confounder
set. The strategy is as follows:

\begin{enumerate}
\def\labelenumi{\alph{enumi}.}
\tightlist
\item
  Control for any variable that causes the exposure, the outcome, or
  both.
\item
  Control for any proxy for an unmeasured variable that is a shared
  cause of both exposure and outcome.
\item
  Define an instrumental variable as a variable associated with the
  exposure but does not influence the outcome independently, except
  through the exposure. Exclude any instrumental variable that is not a
  proxy for an unmeasured confounder from the confounder set.\footnote{Note
    that the concept of a ``confounder set'' is broader than the concept
    of an ``adjustment set.'' Every adjustment set is a member of a
    confounder set. So the Modified Disjunctive Cause Criterion will
    eliminate confounding when the data permit. However a confounder set
    includes variables that will reduce confounding in cases where
    confounding cannot be eliminated. Confounding can almost never be
    elimiated with certainty. For this reason we must perform
    sensitivity analyses to check the robustness of our results. These
    results will be less dependent on sensitivity analysis if we can
    reduce confounding. For this reason, I follow those who recommend
    using the Modified Disjunctive Cause Criterion for confounding
    control. Here, when focussing on strategies for attenuated
    confounding that cannot be fully controlled, I use dotted black
    directed edges to indicate attenuated confounding, and a blue
    directed edge to denote the association between the exposure and the
    outcome. Note that nearly every plausible scenario involving causal
    inference with observational data and non-random exposures presents
    a risk of unmeasured confounding. However, I refrain from
    universally applying this visualisation strategy to each graph to
    maintain focus on the specific issue each graph represents.}
\end{enumerate}

I adopt this strategy because in the workflows of causal inference, we
can only rarely confidently eliminate all confounding. Pragmatically,
our task is to reduce confounding where possible and perform sensitivity
analyses that clarify the magnitude of risks the nearly inevitable
failure of our models to confidently satisfy the ``no-unmeasured
confounding'' assumption.

\subsubsection{Chronological Ordering in Causal
Diagrams}\label{chronological-ordering-in-causal-diagrams}

Recall that the purpose of a causal diagram is to parsimoniously present
structural sources of bias. The structure of our graph rarely be
verified by data. For this reason, we say a causal diagram is
`structural' (\citeproc{ref-bulbulia2021}{Bulbulia \emph{et al.} 2021};
\citeproc{ref-hernuxe1n2023}{Hernán and Robins 2023};
\citeproc{ref-pearl2009a}{Pearl 2009b};
\citeproc{ref-greenland1999c}{\textbf{greenland1999c?}}). It is not a
statistical representation of the data, and thus differs from the graphs
commonly employed in the structural equation modelling tradition. The
failure to appreciate the distinction between structural and statistical
models has led to much confusion
(\citeproc{ref-vanderweele2015}{VanderWeele 2015};
\citeproc{ref-vanderweele2022}{VanderWeele 2022};
\citeproc{ref-vanderweele2022b}{VanderWeele and Vansteelandt 2022}).

A chronological ordered in causal diagram is mathematically equivalent
to one that lacks such order. However, what might be called
``chronological hygiene'' in the layout of a graph can often greatly
enhance understanding of the causal relationships among variables. A
graph is chronologically hygienic if the arrangement of nodes and arrows
in the diagram follow the assumed temporal sequence of causation. Here,
I adopt the following conventions for chronological hygiene.

\begin{enumerate}
\def\labelenumi{\arabic{enumi}.}
\item
  \textbf{Left-to-Right temporal flow}: this spatial arrangement mirrors
  the temporal progression of events or variables, with events appearing
  to the left occurring earlier and events occurring on the right
  appearing later.
\item
  \textbf{Time-indexing of nodes}: to further enhance clarity and
  precision, I index nodes on diagram by their relative occurrence in
  time. Such indexing explicitly denotes the relative time point at
  which each variable or event occurs or is measured relative to the
  others
\end{enumerate}

For example, suppose we are interested in the question of whether
beliefs in Big Gods (\$A\$) affect Social Complexity (\$Y\$). Suppose
further that we believe social factors (\$L\$) influence beliefs in Big
Gods. We would then arrange the nodes such that:

\[L_{t0} \to A_{t1} \to Y_{t2}\]

We are now ready to apply chronologically ordered diagrams to the tasks
of confounding control.

\subsubsection{Elemental Confounds and Their
Solutions}\label{elemental-confounds-and-their-solutions}

We begin by reviewing the ``four fundamental confounders'' described in
McElreath (\citeproc{ref-mcelreath2020}{2020}) p.185.

\subsubsection{1. The problem of confounding by a common
cause}\label{the-problem-of-confounding-by-a-common-cause}

Confounding arises from a common cause when a variable, denoted as
\(L\), influences both an exposure (\(A\)) and an outcome (\(Y\)). This
influence of \(L\) can create a statistical association between \(A\)
and \(Y\), which does not necessarily imply a causal relationship.

Consider an example where smoking is a common cause (\(L\)) that leads
to both yellow fingers (\(A\)) and cancer (\(Y\)). In such cases, \(A\)
and \(Y\) will show an association in the data, driven by the common
cause \(L\). For instance, changing the colour of a person's fingers
would not affect their cancer risk. The confounding effect is
represented in Figure~\ref{fig-dag-common-cause}, where the red arrow
signifies the bias from the open path connecting \(A\) and \(Y\), caused
by their common cause \(L\).

\begin{figure}

\centering{

\includegraphics[width=0.8\textwidth,height=\textheight]{causal-dags_files/figure-pdf/fig-dag-common-cause-1.pdf}

}

\caption{\label{fig-dag-common-cause}Counfounding by a common cause. The
red path indicates bias arising from the open backdoor path from A to
Y.}

\end{figure}%

\subsubsection{Advice: Consider the Temporal Sequence of All Measured
Variables}\label{advice-consider-the-temporal-sequence-of-all-measured-variables}

To address confounding by a common cause, one should adjust for it,
effectively blocking the backdoor path from the exposure to the outcome.
Essentially, conditioning on \(L\) separates \(A\) and \(Y\) in terms of
dependency. Standard methods for this adjustment include regression,
matching, inverse probability of treatment weighting, and G-methods, as
detailed in (\citeproc{ref-hernuxe1n2023}{Hernán and Robins 2023}).
Figure~\ref{fig-dag-common-cause-solution} highlights the necessity for
any confounder, as a common cause of both \(A\) and \(Y\), to precede
\(A\) chronologically, as causes precede their effects.

After we have time-indexing the nodes on the graph it becomes evident
that \textbf{control of confounding generally requires time-series data
repeatedly measured on the units for which causal inferences apply.} Our
causal diagram is a circuit breaker that casts doubt on attempts for
causal inference in settings where researchers lack time series data.

\begin{figure}

\centering{

\includegraphics[width=0.8\textwidth,height=\textheight]{causal-dags_files/figure-pdf/fig-dag-common-cause-solution-1.pdf}

}

\caption{\label{fig-dag-common-cause-solution}Solution: adjust for
pre-exposure confounder. The implication: obtain time series data to
ensure the confounder occurs before the exposure.}

\end{figure}%

\subsubsection{2. Confounding by Collider Stratification (Conditioning
on a Common
Effect)}\label{confounding-by-collider-stratification-conditioning-on-a-common-effect}

When conditioning on a common effect, we consider a scenario where a
variable \(L\) is influenced by both a treatment \(A\) and an outcome
\(Y\).

Imagine \(A\) represents the level of belief in Big Gods, and \(Y\)
denotes social complexity, with \(L\) being economic trade. Initially,
suppose there is no causal link between \(A\) and \(Y\) --- altering
belief in Big Gods does not impact social complexity directly. However,
assume both \(A\) and \(Y\) independently affect economic trade (\(L\)).
If we analyze the data, ignoring the temporal sequence, particularly
when time series data are not available, or are measured with error, we
might mistakenly infer a causal relationship between \(A\) and \(Y\)
arising from their shared effect on \(L\).

In mathematical terms, when \(A\) and \(Y\) are independent, their joint
probability should equal the product of their individual probabilities:
\(P(A, Y) = P(A)P(Y)\). But, conditioning on \(L\) alters this
relationship. The joint probability of \(A\) and \(Y\) given \(L\),
\(P(A, Y | L)\), does not equal the product of \(P(A | L)\) and
\(P(Y | L)\). Thus, the common effect \(L\) creates an apparent
association between \(A\) and \(Y\), which is not causal. This spurious
association might lead to the false conclusion of a direct link between
beliefs in Big Gods and social complexity in cross-sectional data.

\begin{figure}

\centering{

\includegraphics[width=0.8\textwidth,height=\textheight]{causal-dags_files/figure-pdf/fig-dag-common-effect-1.pdf}

}

\caption{\label{fig-dag-common-effect}Confounding by conditioning on a
collider. The dashed red path indicates bias from the open backdoor path
from A to Y.}

\end{figure}%

\subsubsection{Advice: attend to the temporal order of all measured
variables}\label{advice-attend-to-the-temporal-order-of-all-measured-variables}

To manage the issue of conditioning on a common effect, it is crucial to
maintain the correct temporal order:

\begin{enumerate}
\def\labelenumi{\arabic{enumi}.}
\tightlist
\item
  Measure all confounders \(L_{t0}\), which are common causes of both
  the exposure \(A_{t1}\) and the outcome \(Y_{t2}\).
\item
  Ensure that \(A_{t1}\) is measured before \(Y_{t2}\).
\end{enumerate}

Adhering to this temporal sequence precludes \(L\) from being an effect
of either \(A\) or \(Y\).\footnote{7}: In our example of beliefs and
social complexity, this typically necessitates time-series data with
precise measurements. Additionally, a sufficiently large sample of
cultures undergoing transitions in religious beliefs, with pre- and
post-transition measurements of social complexity, is needed. The
cultures in the dataset should also be independent of each other. That
is, each element of our causal inferential workflow must
cohere.\footnote{The independence of cultural units was at the centre of
  the study of comparative urban archaeology from the late 19th
  (\citeproc{ref-decoulanges1903}{De Coulanges 1903}) through the late
  20th century (\citeproc{ref-wheatley1971}{Wheatley 1971}). Despite
  attention to this problem in recent work (e.g.
  (\citeproc{ref-watts2016}{Watts \emph{et al.} 2016})), there is
  arguably a greater head-room for understanding the need for
  conditional independence of cultures in recent cultural evolutionary
  studies. Again, attending to the temporal order of events is
  essential.}

\begin{figure}

\centering{

\includegraphics[width=0.8\textwidth,height=\textheight]{causal-dags_files/figure-pdf/fig-dag-common-effect-solution-1.pdf}

}

\caption{\label{fig-dag-common-effect-solution}Solution: time idexing of
confounders helps to avoid collider bias and maintain d-separation. The
graph makes the imperative clear: we must collect time series data with
confounders measured before the exposure, and that we must likewise
measure the exposure before the outcome, with data collected
repeatitively on the same units.}

\end{figure}%

\subsubsection{M-bias: Conditioning on a Collider Before the Exposure
May Introduce
Bias}\label{m-bias-conditioning-on-a-collider-before-the-exposure-may-introduce-bias}

While it is generally advisable to include indicators for confounders
that are measured before their corresponding exposures, there are
caveats. One must be cautious not to over-condition on pre-exposure
variables that are not associated with both the exposure and confounder.
Such over-conditioning may inadvertently induce confounding, known as
``M-bias.''

As illustrated in Figure~\ref{fig-m-bias}, M-bias can occur even if a
variable \(L\) occurs before the treatment \(A\). This happens when
\(L\) does not affect either \(A\) or \(Y\), but is a descendant of
unmeasured variables that influence both \(A\) and \(Y\) independently.
Conditioning on \(L\) creates a spurious association between \(A\) and
\(Y\). In such cases, \(A\) and \(Y\) might be unconditionally
independent (\(A \coprod Y(a)\)), but when stratified by \(L\),
independence is violated: (\(A \cancel{\coprod} Y(a)| L\)). This form of
bias is another manifestation of collider stratification bias.

Note that when the path is ordered chronologically from left to right,
the ``M'' shape, giving M-bias its name, changes to an ``E'' shape.
However, the term ``M-bias'' is retained.

\subsubsection{Advice: Adopt the Modified Disjunctive Cause Criterion
for Confounding
Control}\label{advice-adopt-the-modified-disjunctive-cause-criterion-for-confounding-control}

The modified disjunctive cause criterion offers a strategy to satisfy
the backdoor criterion and reduce bias:

\begin{enumerate}
\def\labelenumi{\alph{enumi}.}
\tightlist
\item
  Control for any variable that causes the exposure, the outcome, or
  both.
\item
  Control for any proxy for an unmeasured variable that is a shared
  cause of both exposure and outcome.
\item
  Define an instrumental variable as one associated with the exposure
  but not independently influencing the outcome, except through the
  exposure. Exclude any instrumental variable that is not a proxy for an
  unmeasured confounder from the confounder set.
\end{enumerate}

Determining which variables belong in the confounder set can be
challenging. Specialist knowledge often plays a key role here, as the
data alone may not provide clear guidance. This approach is supported by
various sources, including VanderWeele \emph{et al.}
(\citeproc{ref-vanderweele2020}{2020}) and VanderWeele
(\citeproc{ref-vanderweele2019}{2019}), with specific exceptions noted
in sources like bulbulia2021.

Typically, indicators for confounders should be included only if they
are known to be measured before their exposures - with notable
exceptions described below in fig-dag-descendent-solution-2 and .

However, researchers should also be cautious about over-conditioning on
pre-exposure variables that are not associated with both the exposure
and confounder, as doing so can induce confounding. As shown in
Figure~\ref{fig-m-bias}, collider stratification may arise even if \(L\)
occurs before \(A\). This happens when \(L\) does not affect \(A\) or
\(Y\), but may be the descendent of an unmeasured variable that affects
\(A\) and another unmeasured variable that also affects \(Y\).
Conditioning on \(L\) in this scenario evokes ``M-bias.'' If \(L\) is
not a common cause of \(A\) and \(Y\), or the effect of a shared common
cause, \(L\) should not be included in a causal model.
Figure~\ref{fig-m-bias} presents a case in which \(A \coprod Y(a)\) but
\(A \cancel{\coprod} Y(a)| L\). M-bias is another example of collider
stratification bias (see: (\citeproc{ref-cole2010}{Cole \emph{et al.}
2010})).\footnote{Note, when we draw a chronologically ordered path from
  left to right the M shape for which ``M-bias'' takes its name changes
  to an E shape We shall avoid proliferating jargon and retain the term
  ``M bias.''}

\begin{figure}

\centering{

\includegraphics[width=0.8\textwidth,height=\textheight]{causal-dags_files/figure-pdf/fig-m-bias-1.pdf}

}

\caption{\label{fig-m-bias}M-bias: Confounding control by including
previous outcome measures. The dashed red path indicates bias from the
open backdoor path from A to Y by conditioning on pre-exposure variable
L. The solution: do not condition on L. The graph shows that
conditioning on variables measured before the exposure is not sufficient
to prevent confounding.}

\end{figure}%

\subsubsection{Advice: Adopt the Modified Disjunctive Cause Criterion
for Confounding
Control}\label{advice-adopt-the-modified-disjunctive-cause-criterion-for-confounding-control-1}

The modified disjunctive cause criterion offers a strategy to satisfy
the backdoor criterion and reduce bias:

\begin{enumerate}
\def\labelenumi{\alph{enumi}.}
\tightlist
\item
  Control for any variable that causes the exposure, the outcome, or
  both.
\item
  Control for any proxy for an unmeasured variable that is a shared
  cause of both exposure and outcome.
\item
  Define an instrumental variable as one associated with the exposure
  but not independently influencing the outcome, except through the
  exposure. Exclude any instrumental variable that is not a proxy for an
  unmeasured confounder from the confounder set.
\end{enumerate}

Determining which variables belong in the confounder set can be
challenging. Specialist knowledge often plays a key role here, as the
data alone may not provide clear guidance. For more discussion of the
modified disjunctive cause criterion see: VanderWeele \emph{et al.}
(\citeproc{ref-vanderweele2020}{2020}) and VanderWeele
(\citeproc{ref-vanderweele2019}{2019}).

\subsubsection{3. Mediator Bias}\label{mediator-bias}

Mediator bias occurs when conditioning on a mediator, a variable that is
part of the causal pathway between the treatment and the outcome,
distorts the total effect of the treatment on the outcome.

Consider ``beliefs in Big Gods'' as the treatment \(A_{t0}\), ``social
complexity'' as the outcome \(Y_{t2}\), and ``economic trade'' as the
mediator \(L_{t1}\).

In this example, beliefs in Big Gods \(A_{t0}\) directly influence
economic trade \(L_{t1}\), which then affects social complexity
\(Y_{t2}\). Conditioning on economic trade \(L_{t1}\) can lead to biased
estimates of the overall effect of beliefs in Big Gods \(A\) on social
complexity \(Y_{t2}\). This bias arises because conditioning on \(L\)
might minimize the direct effect of \(A_{t0}\) on \(Y_{t1}\), blocking
the pathway through \(L_{t1}\). This is known as mediator bias,
illustrated in Figure~\ref{fig-dag-mediator}.

It might seem that conditioning on a mediator under a null hypothesis
(where \(A\) does not cause \(Y\)) would not introduce bias. However,
consider a situation where \(L_{t1}\) is affected by both the exposure
\(A_{t0}\) and an unmeasured variable \(U\) that is related to the
outcome \(Y_{t2}\). In this case, including \(L_{t1}\) in the analysis
might exaggerate the association between \(A_{t0}\) and \(Y_{t2}\), even
if there is no actual association between them and \(U\) does not cause
\(A_{t0}\). This scenario is depicted in
Figure~\ref{fig-dag-descendent}.

Therefore, unless specifically conducting mediation analysis, it is
generally inadvisable to condition on a post-treatment variable like
\(L_{t1}\). Attending to the temporal order in data collection is
crucial. If we cannot ensure that \(L\) is measured before \(A\), and if
\(A\) can influence \(L\), including \(L\) in our model could lead to
mediator bias. This scenario is represented in
Figure~\ref{fig-dag-descendent}.

You ask: ``What if \(L\) is a confounder of \(A\) and \(Y\)?'' We have
already considered this problem, and its solution. The problem was
presented in Figure~\ref{fig-dag-common-cause}. Our chronlogically
ordered causal diagram Figure~\ref{fig-dag-common-cause-solution}
directs us to the solution: \emph{If \(L\) is a common cause of \(A\)
and \(Y\) we must ensure that our measurement of \(L\) occurs before our
measurement of \(A\) and that our measurement of \(Y\) occurs after our
measurement of \(A\).} For we will only control for confounding if we
have obtained \(L_{t0}\), \(A_{t1}\), and \(Y_{t2}\), and not otherwise.

\begin{figure}

\centering{

\includegraphics[width=0.8\textwidth,height=\textheight]{causal-dags_files/figure-pdf/fig-dag-mediator-1.pdf}

}

\caption{\label{fig-dag-mediator}Confounding by conditioning on a
mediator. The dashed black arrow indicates bias arising from partially
blocking the path between A and Y. Here, the true effect of A on Y is
attenuated.}

\end{figure}%

\subsubsection{Advice: attend to the temporal order of all measured
variables}\label{advice-attend-to-the-temporal-order-of-all-measured-variables-1}

One should only condition on a mediator if our interest pre-specified
causal question requires a causal mediation model (The assumptions of
causal mediation are discussed in Section 3). Generally, we may avoid
mediation bias by ensuring that \(L\) is measured before the treatment
\(A\) and the outcome \(Y\). There are two exceptions to this rule. If
\(L\) were associated with \(Y\) and could not be caused by \(A\), then
conditioning on \(L\) would typically enhance the precision of the
causal effect estimate of \(A \to Y\). This precision enhancement holds
even if \(L\) occurs \emph{after} \(A\). A second a counter-example is
presented in Figure~\ref{fig-dag-descendent-solution-2} and developed in
the next section. However, when conditioning on a post-treatment
variable the onus is always on the researcher to explain why the
post-treatment variable cannot be affected by the exposure. Here again,
we discover the importance of explicitly stating the temporal ordering
of our variables in our graph.{[}\^{}10{]} Doing so directs us the data
we need to answer our causal question, and greatly diminishes the threat
of unwittingly introducing mediation bias.

\begin{figure}

\centering{

\includegraphics[width=0.8\textwidth,height=\textheight]{causal-dags_files/figure-pdf/fig-dag-mediator-solution-1.pdf}

}

\caption{\label{fig-dag-mediator-solution}Solution: do not condition on
a mediator. The implication: by ensuring temporal order in data
collection we diminish the probabilty of mistaking an effect of an
exposure for its confounder.}

\end{figure}%

\subsubsection{4. Conditioning on a
descendant}\label{conditioning-on-a-descendant}

\paragraph{Case when conditioning on a descendant augments
bias}\label{case-when-conditioning-on-a-descendant-augments-bias}

Suppose a variable \(L\) is a cause of another variable \(L^\prime\).
Here, \(L^\prime\) is a descendant of \(L\). According to Markov
factorisation (see Appendix 2), if we were to condition on \(L\prime\),
we would also partially condition on \(L\).

Consider how conditioning on \(L\prime\) might imperil causal
estimation. Suppose there is a confounder \(L^\prime\) that is caused by
an unobserved variable \(U\), and is affected by the treatment \(A\).
Suppose further that \(U\) causes the outcome \(Y\). In this scenario,
as described in Figure~\ref{fig-dag-descendent}, conditioning on
\(L^\prime\), which is a descendant of \(A\) and \(U\), can lead to a
spurious association between \(A\) and \(Y\) through the path
\(A \to L^\prime \to U \to Y\).

\begin{figure}

\centering{

\includegraphics[width=0.8\textwidth,height=\textheight]{causal-dags_files/figure-pdf/fig-dag-descendent-1.pdf}

}

\caption{\label{fig-dag-descendent}Confounding by descent: the red path
illustrates the introduction of bias by conditioning on the descendant
of a confounder that is affected by the exposure, thus opening a path
between the exposure, A, and the outcome, Y.}

\end{figure}%

\subsubsection{Solution to the problem of augmenting bias by
conditioning on a
descendant}\label{solution-to-the-problem-of-augmenting-bias-by-conditioning-on-a-descendant}

Again, the strategy for avoiding the problem of augmented bias by
conditioning on a descendant is evident from the chronology of the
graph. If we wish to reduce confounding we must ensure that \(L^\prime\)
is measured before the exposure \(A\). This strategy is presented in
Figure~\ref{fig-dag-descendent-solution}.

\begin{figure}

\centering{

\includegraphics[width=0.8\textwidth,height=\textheight]{causal-dags_files/figure-pdf/fig-dag-descendent-solution-1.pdf}

}

\caption{\label{fig-dag-descendent-solution}Solution: measure L before
A. Note, L need not affect Y to be a confounder (i.e.~a member of a
confounder set).}

\end{figure}%

\paragraph{Case when conditioning on a descendant reduces
bias}\label{case-when-conditioning-on-a-descendant-reduces-bias}

Next consider a case in which we may use a post-treatment descendent to
reduce bias. Suppose an unmeasured confounder \(U\) affects \(A\),
\(Y\), and \(L^\prime\) in an effect of \(U\) that occurs after \(A\)
and \(Y\). In this scenario adjusting for \(L^\prime\) may help to
reduce confounding caused by the unmeasured confounder \(U\). This
strategy follows from the modified disjunctive cause criterion for
confounding control, we recommends that we ``include as a covariate any
proxy for an unmeasured variable that is a common cause of both the
exposure and the outcome'' (\citeproc{ref-vanderweele2019}{VanderWeele
2019}). As shown in Figure~\ref{fig-dag-descendent-solution-2}, although
\(L^\prime\) occurs \emph{after} the exposure, and indeed occur
\emph{after} the outcome, coniditioning on it will reduce confounding.
How might this work? Consider a genetic factor that affects the exposure
and the outcome early in life might be measured by an indicator late
that is expressed (and may be measured) later in life. Adjusting for
such an indicator would constitute an example of post-outcome
confounding control.

\textbf{This example shows that employing a rule that requires us to
condition only on pre-exposure (and indeed pre-outcome) variables would
be hasty.} More generally, fig-dag-descendent-solution-2 demonstrates
the imperative for thinking carefully about data collection. Each
problem must be approached anew.

\begin{figure}

\centering{

\includegraphics[width=0.8\textwidth,height=\textheight]{causal-dags_files/figure-pdf/fig-dag-descendent-solution-2-1.pdf}

}

\caption{\label{fig-dag-descendent-solution-2}Solution: conditioning on
a confounder that occurs after the exposure and the outcome might
address a problem of unmeasured confounding if the confounder is a
descendent of a prior common cause of the exposure and outcome. The
dotted paths denote that the effect of U on A and Y is partially blocked
by conditioning on L', even though L' occurs after the outcome. The
paths are dotted to represent a reduction of bias by conditioning on the
post-outcome descendent of an unmeasured common cause of the exposure
and outcome.}

\end{figure}%

\subsection{Part 3. Application of Causal Diagrams for Clarifying
Moderation (Interaction), Mediation, and Longitudinal
Feedback}\label{part-3.-application-of-causal-diagrams-for-clarifying-moderation-interaction-mediation-and-longitudinal-feedback}

\subsubsection{Case 1. Causal Interaction and Causal Effect
Modification: do not draw non-linear relationships such as
interactions}\label{case-1.-causal-interaction-and-causal-effect-modification-do-not-draw-non-linear-relationships-such-as-interactions}

Interactions are scientific interesting because we often wish to
understand whether causal effects operate differently in different
sub-populations, or whether the joint effect of two interventions differ
from the either taken alone, and from no intervention.

How shall we depict interactions on a graph? It is crucial to remember
the primary function of causal diagrams is to investigate confounding.
Causal diagrams are not designed to capture all facets of a phenomenon
under investigation. We should not attempt any unique visual trick to
show additive and multiplicative interaction by, for example, presenting
arrows intersecting arrows. Moreover, we should include those nodes and
paths that are necessary to evaluate structural sources of bias. Causal
graphs are meant to be human readable. They are not meant to be complete
maps of causal reality.

In my experience, misunderstandings arise about the role and function of
causal diagrams in application to interaction are not simply confusions
about graphical convention. Misunderstanding typically stems from a more
profound confusion about the concept of interaction itself.

Given this deeper problem, it is worth clarifying two distinct
conceptions of causal interaction as understood within the
counterfactual causal framework: the concept of causal interaction as a
double exposure and the concept of causal effect-modification from a
single exposure.

\paragraph{\texorpdfstring{\textbf{Causal interaction as a double
exposure}}{Causal interaction as a double exposure}}\label{causal-interaction-as-a-double-exposure}

Causal interaction refers to the combined and separate effect of two
exposures. Evidence for interaction on a given scale is present when the
effect of one exposure on an outcome depends on another exposure's
level. For instance, the impact of beliefs in Big Gods (exposure \(A\))
on social complexity (outcome \(Y\)) might depend on a culture's
monumental architecture (exposure \(B\)), which could also influence
social complexity. Evidence of causal interaction on the difference
scale would be present if:

\[\bigg(\underbrace{\mathbb{E}[Y(1,1)]}_{\text{joint exposure}} - \underbrace{\mathbb{E}[Y(0,0)]}_{\text{neither exposed}}\bigg) - \bigg[ \bigg(\underbrace{\mathbb{E}[Y(1,0)]}_{\text{only A exposed}} - \underbrace{\mathbb{E}[Y(0,0)]}_{\text{neither exposed}}\bigg) + \bigg(\underbrace{\mathbb{E}[Y(0,1)]}_{\text{only B exposed}} - \underbrace{\mathbb{E}[Y(0,0)]}_{\text{neither exposed}} \bigg)\bigg] \neq 0 \]

This equation simplifies to

\[ \underbrace{\mathbb{E}[Y(1,1)]}_{\text{joint exposure}} - \underbrace{\mathbb{E}[Y(1,0)]}_{\text{only A exposed}} - \underbrace{\mathbb{E}[Y(0,1)]}_{\text{only B exposed}} + \underbrace{\mathbb{E}[Y(0,0)]}_{\text{neither exposed}} \neq 0 \]

If the above equation were to hold, the effect of exposure \(A\) on the
outcome \(Y\) would differ across levels of \(B\) or vice versa. Such a
difference would provide evidence for interaction.

If the value is positive, we say there is evidence for an additive
effect. If the value is less than zero, we say there is evidence for a
sub-additive effect. If the value is virtually zero, there is no
reliable evidence for interaction.\footnote{Note that causal effects of
  interactions often differ when measured on the ratio scale. This
  discrepency can have significant policy implications, see:
  (\citeproc{ref-vanderweele2014}{VanderWeele and Knol 2014}). Although
  beyond the scope of this article, when evaluating evidence for
  causality we must clarify the measure of effect in which we are
  interested (\citeproc{ref-hernuxe1n2004}{Hernán \emph{et al.} 2004};
  \citeproc{ref-tripepi2007}{Tripepi \emph{et al.} 2007}).}

Remember that causal diagrams are non-parametric. They do not directly
represent interactions. They are tools for addressing the identification
problem. Although a causal diagram can indicate an interaction's
presence by displaying two exposures jointly influencing an outcome, as
in Figure~\ref{fig-dag-interaction}, it does not directly represent the
interaction's nature or scale.

\begin{figure}

\centering{

\includegraphics[width=0.6\textwidth,height=\textheight]{causal-dags_files/figure-pdf/fig-dag-interaction-1.pdf}

}

\caption{\label{fig-dag-interaction}Causal interaction: if two exposures
are causally independent of each other, we may wish to estimate their
individual and joint effects on Y, conditional on confounding control
strategy that blocks backdoor paths for bothe exposures (here, L1 and L2
are jointly required). where the counterfactual outcome is Y(a,b) and
there is evidence for additive or subadditive interaction if E{[}Y(1,1)
- Y(0,1) - Y(1,0) + Y(0,0){]} ≠ 0. If we cannot conceptualise B as a
variable upon which intervention can occur, then the interaction is
better conceived as effect modification (see next figure). Important: do
not attempt to draw a path into another path.}

\end{figure}%

\paragraph{\texorpdfstring{\textbf{Causal effect modification under a
single
exposure}}{Causal effect modification under a single exposure}}\label{causal-effect-modification-under-a-single-exposure}

With the analysis of effect modification, we aim to understand how an
exposure's effect varies, if at all, across levels of another variable,
an effect modifier.

Consider again the problem of estimating the causal effect of beliefs in
Big Gods on social complexity. Suppose this time we are interested in
the investigating whether this effect varies across early urban
civilisations in ancient China and South America. In this example
geography (China versus South America) is an ``effect modifier.'' Here,
we do not treat the effect modifier as an intervention. Rather, we wish
to investigate whether geography is a parameter that may alter the
exposure's effect on an outcome.

For clarity, consider comparing two exposure levels, represented as
\(A = a\) and \(A= a^*\). Further, assume that \(G\) represents two
levels of effect-modification, represented as \(g\) and \(g'\).

Then, the expected outcome when exposure is at level \(A=a\) among
individuals in group \(G=g\) is expressed

\[\hat{E}[Y(a)|G=g]\]

The expected outcome when exposure is at level \(A=a^*\) among
individuals in group \(G=g\) is expressed

\[\hat{E}[Y(a^*)|G=g]\]

The causal effect of shifting the exposure level from \(a^*\) to \(a\)
in group \(g\) is expressed

\[\hat{\delta}_g = \hat{\mathbb{E}}[Y(a)|G=g] - \hat{\mathbb{E}}[Y(a^*)|G=g]\]

Likewise, the causal effect of changing the exposure from \(a^*\) to
\(a\) in group \(g'\) is expressed.

\[\hat{\delta}_{g'} = \hat{\mathbb{E}}[Y(a)|G=g'] - \hat{\mathbb{E}}[Y(a^*)|G=g']\]

We compare the causal effect on the difference scale in these two groups
by computing

\[\hat{\gamma} = \hat{\delta}_g - \hat{\delta}_{g'}\]

The value of \(\hat{\gamma}\) quantifies how the effect of shifting the
exposure from \(a^*\) to \(a\) differs between groups \(g\) and \(g'\).

If \(\hat{\gamma}\neq 0\), then there is evidence for effect
modification. We may infer the exposure's effect varies by geography.

Again, remember that causal diagrams are non-parametric. More
fundamental, causal diagrams function to identify structural sources of
bias and to help researchers develop strategies for addressing such
bias. We should not draw an intersecting path or attempt other
visualisations to represent effect modification. Instead, we should draw
two edges into the exposure. This is depicted in
Figure~\ref{fig-dag-effect-modfication}.\footnote{For distinctions
  within varieties of effect modification relevant for strategies of
  confounding controul see (\citeproc{ref-vanderweele2007}{VanderWeele
  and Robins 2007}).}

\begin{figure}

\centering{

\includegraphics[width=0.8\textwidth,height=\textheight]{causal-dags_files/figure-pdf/fig-dag-effect-modfication-1.pdf}

}

\caption{\label{fig-dag-effect-modfication}A simple graph for
effect-modification in which there are no confounders. G is an effect
modifier of A on Y. We draw a box around G to indicate we are
conditioning on this variable.}

\end{figure}%

\subsubsection{Case 2: Causal mediation: causal diagrams reveal the
inadequacy of standard
approaches}\label{case-2-causal-mediation-causal-diagrams-reveal-the-inadequacy-of-standard-approaches}

The conditions necessary for causal mediation are stringent. I present
these conditions in the chronologically ordered causal diagram shown in
Figure~\ref{fig-dag-mediation-assumptions}. We will again consider
whether cultural beliefs in Big Gods affect social complexity. We now
ask whether this affect is mediated by political authority. The
assumptions required for asking causal mediation questions are as
follows

\begin{enumerate}
\def\labelenumi{\arabic{enumi}.}
\tightlist
\item
  \textbf{No unmeasured exposure-outcome confounder}
\end{enumerate}

This prerequisite is expressed: \(Y(a,m) \coprod A | L1\). Upon
controlling for the covariate set \(L1\), we must ensure that no
additional unmeasured confounders affect both the cultural beliefs in
Big Gods \(A\) and the social complexity \(Y\). For example, suppose our
study involves the effect of cultural beliefs in Big Gods (exposure) on
social complexity (outcome), and geographic location and historical
context define the covariates in \(L1\). In that case, we must assume
that accounting for \(L1\) d-separates \(A\) and \(Y\). The relevant
confounding path is depicted in brown in
Figure~\ref{fig-dag-mediation-assumptions}.

\begin{enumerate}
\def\labelenumi{\arabic{enumi}.}
\setcounter{enumi}{1}
\tightlist
\item
  \textbf{No unmeasured mediator-outcome confounder}
\end{enumerate}

This condition is expressed: \(Y(a,m) \coprod M | L2\). After
controlling for the covariate set \(L2\), we must ensure that no other
unmeasured confounders affect the political authority \(M\) and social
complexity \(Y\). For instance, if trade networks impact political
authority and social complexity, we must account for trade networks to
obstruct the unblocked path linking our mediator and outcome. Further,
we must assume the absence of any other confounders for the
mediator-outcome path. This confounding path is represented in blue in
Figure~\ref{fig-dag-mediation-assumptions}.

\begin{enumerate}
\def\labelenumi{\arabic{enumi}.}
\setcounter{enumi}{2}
\tightlist
\item
  \textbf{No unmeasured exposure-mediator confounder}
\end{enumerate}

This requirement is expressed: \(M(a) \coprod A | L3\). Upon controlling
for the covariate set \(L3\), we must ensure that no additional
unmeasured confounders affect both the cultural beliefs in Big Gods
\(A\) and political authority \(M\). For example, the capability to
construct large ritual theatres may influence the belief in Big Gods and
the level of political authority. If we have indicators for this
technology measured prior to the emergence of Big Gods (these indicators
being \(L3\)), we must assume that accounting for \(L3\) closes the
backdoor path between the exposure and the mediator. This confounding
path is shown in green in Figure~\ref{fig-dag-mediation-assumptions}.

\begin{enumerate}
\def\labelenumi{\arabic{enumi}.}
\setcounter{enumi}{3}
\tightlist
\item
  \textbf{No mediator-outcome confounder affected by the exposure}
\end{enumerate}

This requirement is expressed: \(Y(a,m) \coprod M(a^*) | L\). We must
ensure that no variables confounding the relationship between political
authority and social complexity in \(L2\) are themselves influenced by
the cultural beliefs in Big Gods (\(A\)). For instance, when studying
the effect of cultural beliefs in Big Gods (\(A\), the exposure) on
social complexity (\(Y\), the outcome) as mediated by political
authority (mediator), there can be no factors, such as trade networks
(\(L2\)), that influence both political authority and social complexity
and are affected by the belief in Big Gods. This confounding path is
shown in red in Figure~\ref{fig-dag-mediation-assumptions}. \textbf{Note
that the assumption of no exposure-induced confounding in the
mediator-outcome relationship is often a substantial obstacle for causal
mediation analysis.} If the exposure influences a confounder of the
mediator and outcome, we face a dilemma. Without accounting for this
confounder, the backdoor path between the mediator and the outcome
remains open. By accounting for it, however, we partially obstruct the
path between the exposure and the mediator, leading to bias.
Consequently, observed data cannot identify the natural direct and
indirect effects.

Notice again that the requirements for counterfactual data science are
more strict than for descriptive or predictive data science.

We have now considered how chronologically ordered causal diagrams
elucidate the conditions necessary for causal mediation
analysis.\footnote{An excellent resource both for understanding causal
  interaction and causal mediation is
  (\citeproc{ref-vanderweele2015}{VanderWeele 2015}).}

\begin{figure}

\centering{

\includegraphics[width=1\textwidth,height=\textheight]{causal-dags_files/figure-pdf/fig-dag-mediation-assumptions-1.pdf}

}

\caption{\label{fig-dag-mediation-assumptions}This causal diagram
illustrates the four fundamental assumptions needed for causal mediation
analysis. The first assumption pertains to the brown paths. It requires
the absence of an unmeasured exposure-outcome confounder, and assumes
that conditioning on L1 is sufficient for such confounding control. The
second assumption pertains to the blue paths. It requires the absence of
an unmeasured mediator-outcome confounder, and assumes that conditioning
on L2 is sufficient for such confounding control. The third assumption
pertains to the green paths. It requires the absence of an unmeasured
exposure-mediator confounder, and assumes that conditioning on L3 is
sufficient for such confounding control. The fourth and final assumption
pertains to the red paths. It requires the absence of an a
mediator-outcome confounder that is affected by the exposure, and
assumes that there is no path from the exposure to L2 to M. If the
exposure were to affect L2, then conditioning on L2 would block the
exposure's effect on the mediator, as indicated by dashed red path.
Causal diagrams not only clarify how different types of confounding bias
may converge (here mediation bias and confounder bias), but also reveal
the limitations of common methods such as structural equation models and
multilevel models for handling time-series data where the fourth
assumption fails -- that is, where there is treatment-confounder
feedback. Such feedback is common in time-series data, but not widely
understood. For example structural equation models and multi-level
models cannot address causal questions in the presence of such feedback,
but these models remain widely favoured.}

\end{figure}%

\subsubsection{Case 3: Confounder-Treatment Feedback: Longitudinal
``growth'' is not
causation}\label{case-3-confounder-treatment-feedback-longitudinal-growth-is-not-causation}

In our discussion of causal mediation, we consider how the effects of
two sequential exposures may combine to affect an outcome. We can
broaden this interest to consider the causal effects of multiple
sequential exposures. In such scenarios, causal diagrams arranged
chronologically can aid in clarifying the challenges and opportunities.

For example, consider temporally fixed multiple exposures. The
counterfactual outcomes may be denoted \(Y(a_{t1} ,a_{t2})\). There are
four counterfactual outcomes corresponding to the four fixed ``treatment
regimes:''

\begin{enumerate}
\def\labelenumi{\arabic{enumi}.}
\item
  \textbf{Always treat (Y(1,1))}
\item
  \textbf{Never treat (Y(0,0))}
\item
  \textbf{Treat once first (Y(1,0))}
\item
  \textbf{Treat once second (Y(0,1))}
\end{enumerate}

\begin{longtable}[]{@{}
  >{\raggedright\arraybackslash}p{(\columnwidth - 4\tabcolsep) * \real{0.1351}}
  >{\raggedright\arraybackslash}p{(\columnwidth - 4\tabcolsep) * \real{0.5405}}
  >{\raggedright\arraybackslash}p{(\columnwidth - 4\tabcolsep) * \real{0.3243}}@{}}
\caption{Table describes four fixed treatment regimes and six causal
contrasts in time series data where the exposure may
vary.}\label{tbl-regimes}\tabularnewline
\toprule\noalign{}
\begin{minipage}[b]{\linewidth}\raggedright
Type
\end{minipage} & \begin{minipage}[b]{\linewidth}\raggedright
Description
\end{minipage} & \begin{minipage}[b]{\linewidth}\raggedright
Counterfactual Outcome
\end{minipage} \\
\midrule\noalign{}
\endfirsthead
\toprule\noalign{}
\begin{minipage}[b]{\linewidth}\raggedright
Type
\end{minipage} & \begin{minipage}[b]{\linewidth}\raggedright
Description
\end{minipage} & \begin{minipage}[b]{\linewidth}\raggedright
Counterfactual Outcome
\end{minipage} \\
\midrule\noalign{}
\endhead
\bottomrule\noalign{}
\endlastfoot
Regime & Always treat & Y(1,1) \\
Regime & Never treat & Y(0,0) \\
Regime & Treat once first & Y(1,0) \\
Regime & Treat once second & Y(0,1) \\
Contrast & Always treat vs.~Never treat & E{[}Y(1,1) - Y(0,0){]} \\
Contrast & Always treat vs.~Treat once first & E{[}Y(1,1) - Y(1,0){]} \\
Contrast & Always treat vs.~Treat once second & E{[}Y(1,1) -
Y(0,1){]} \\
Contrast & Never treat vs.~Treat once first & E{[}Y(0,0) - Y(1,0){]} \\
Contrast & Never treat vs.~Treat once second & E{[}Y(0,0) - Y(0,1){]} \\
Contrast & Treat once first vs.~Treat once second & E{[}Y(1,0) -
Y(0,1){]} \\
\end{longtable}

There are six causal contrasts that we might compute for the four fixed
regimes, presented in Table~\ref{tbl-regimes}.\footnote{We compute the
  number of possible combinations of contrasts by
  \(C(n, r) = \frac{n!}{(n-r)! \cdot r!}\)}

Not that treatment assignments might be sensibly approached as a
function of the previous outcome. For example, we might \textbf{treat
once first} and then decide whether to treat again depending on the
outcome of the initial treatment. This aspect is known as ``time-varying
treatment regimes.''

Bear in mind that to estimate the ``effect'' of a time-varying treatment
regime, we are obligated to make comparisons between the relevant
counterfactual quantities. As mediation can introduce the possibility of
time-varying confounding (condition 4: the exposure must not impact the
confounders of the mediator/outcome path), the same holds true for all
sequential time-varying treatments. However, unlike conventional causal
mediation analysis, it might be necessary to consider the sequence of
treatment regimes over an indefinitely long period.

Chronologically organised causal diagrams are useful for highlighting
problems with traditional multi-level regression analysis and structural
equation modelling.

For example, we might be interested in whether belief in Big Gods
affects social complexity. Consider estimating a fixed treatment regime
first. Suppose we have a well-defined concept of Big Gods and social
complexity as well as excellent measurements for both over time. In that
case, we might want to assess the effects of beliefs in Big Gods on
social complexity, say, two centuries after the beliefs were introduced.

The fixed treatment strategies are: ``always believe in Big Gods''
versus ``never believe in Big Gods'' on the level of social complexity.
Refer to Figure~\ref{fig-dag-9}. Here, \(A_{tx}\) represents the
cultural belief in Big Gods at time \(tx\), and \(Y_{tx}\) is the
outcome, social complexity, at time \(x\). Imagine that economic trade,
denoted as \(L_{tx}\), is a time-varying confounder. Suppose its effect
changes over time, which in turns affects the factors that influence
economic trade. To complete our causal diagram, we might include an
unmeasured confounder \(U\), such as oral traditions, which could
influence both the belief in Big Gods and social complexity.

Consider a scenario where we can reasonably infer that the level of
economic trade at time \(0\), represented as \(L_{t0}\), impacts beliefs
in ``Big Gods'' at time \(1\), denoted as \(A_{t1}\). In this case, we
would draw an arrow from \(L_{t0}\) to \(A_{t1}\). Conversely, if we
assume that belief in ``Big Gods,'' \(A_{t1}\), influences the future
level of economic trade, \(L_{t2}\), then an arrow should be added from
\(A_{t1}\) to \(L_{t2}\). This causal diagram illustrates a feedback
process between the time-varying exposure \(A\) and the time-varying
confounder \(L\). Figure~\ref{fig-dag-9}. displays exposure-confounder
feedback. In practical settings, the diagram could contain more arrows.
However, the intention here is to use the minimal number of arrows
needed to demonstrate the issue of exposure-confounder feedback. As a
guideline, we should avoid overcomplicating our causal diagrams and aim
to include only the essential details necessary for assessing the
identification problem.

What would happen if we were to condition on the time-varying confounder
\(L_{t3}\)? Two things would occur. First, we would block all the
backdoor paths between the exposure \(A_{t2}\) and the outcome. We need
to block those paths to eliminate confounding. Therefore, conditioning
on the time-varying confounding is essential. However, paths that were
previously blocked would close. For example, the path
\(A_{t1}, L_{t2}, U, Y_{t4}\), that was previously closed would be
opened because the time-varying confounder is the common effect of
\(A_{t1}\) and \(U\). Conditioning, then, opens the path
\(A_{t1}, L_{t2}, U, Y_{t4}\). Therefore we must avoid conditioning on
the time-varying confounder. It would seem then that if we were to
condition on a confounder that is affected by the prior exposure, we are
``damned if we do'' and ``dammed if we do not.''

\begin{figure}

\centering{

\includegraphics[width=1\textwidth,height=\textheight]{causal-dags_files/figure-pdf/fig-dag-9-1.pdf}

}

\caption{\label{fig-dag-9}Exposure confounder feedback is a problem for
time-series models. If we do not condition on L\_t2, a backdoor path is
open from A\_t3 to Y\_t4. However, if conditioning on L\_t2 introduces
collider bias, opening a path, coloured in red, between A\_t2 and Y\_t4.
Here, we may not use conventional methods to estimate the effects of
multiple exposures. Instead, at best, we may obtain controlled effects
using G-methods. Multi-level models will not eliminate bias (!).
However, outside of epidemiology, G-methods are presently rarely used.}

\end{figure}%

A similar problem arises when a time-varying exposure and time-varying
confounder share a common cause. This problem arises even without the
exposure affecting the confounder. The problem is presented in
Figure~\ref{fig-dag-time-vary-common-cause-A1-l1}.

\begin{figure}

\centering{

\includegraphics[width=1\textwidth,height=\textheight]{causal-dags_files/figure-pdf/fig-dag-time-vary-common-cause-A1-l1-1.pdf}

}

\caption{\label{fig-dag-time-vary-common-cause-A1-l1}Exposure confounder
feedback is a problem for time-series models. Here, the problem arises
from an unmeasured variable (U\_2) that affects both the exposure A at
time 1 and the cofounder L at time 2. The red paths show the open
backdoor path when we condition on the L at time 2. Again, we cannot
infer causal effects in such scenarios by using regression-based
methods. In this setting, to address causal questions, we require
G-methods.}

\end{figure}%

The potential for confounding increases when the exposure \(A_{t1}\)
affects the outcome \(Y_{t4}\). For example, since \(L_{t2}\) is on the
path from \(A_{t1}\) to \(Y_{t4}\), conditioning on \(L_{t2}\) partially
blocks the relation between the exposure and the outcome, triggering
collider stratification bias and mediator bias. However, to close the
open backdoor path from \(L_{t2}\) to \(Y_{t4}\), it becomes necessary
to condition on \(L_{t2}\). Paradoxically, we have just stated that
conditioning should be avoided! This broader dilemma of
exposure-confounder feedback is thoroughly explored in
(\citeproc{ref-hernuxe1n2023}{Hernán and Robins 2023}). Treatment
confounder feedback is particularly challenging for evolutionary human
science, yet its handling is beyond the capabilities of conventional
regression-based methods, including multi-level models
(\citeproc{ref-hernuxe1n2006}{Hernán and Robins 2006};
\citeproc{ref-robins1986}{Robins 1986}; \citeproc{ref-robins1999}{Robins
\emph{et al.} 1999}). As mentioned previously, G-methods encompass
models appropriate for investigating the causal effects of both
time-fixed and time-varying exposures
(\citeproc{ref-chatton2020}{Chatton \emph{et al.} 2020};
\citeproc{ref-hernuxe1n2006}{Hernán and Robins 2006};
\citeproc{ref-naimi2017}{Naimi \emph{et al.} 2017}). Despite significant
recent advancements in the health sciences
(\citeproc{ref-breskin2020}{Breskin \emph{et al.} 2020};
\citeproc{ref-duxedaz2021}{Díaz \emph{et al.} 2021};
\citeproc{ref-williams2021}{Williams and Díaz 2021}), these methods have
not been widely embraced in the field of human evolutionary sciences
\footnote{It is worth noting that the identification of controlled
  effect estimates can be enhanced by graphical methods such as ``Single
  World Intervention Graphs'' (SWIGs), which represent counterfactual
  outcomes in the diagrams. However, SWIGs are more accurately
  considered templates rather than causal diagrams in their general
  form. The use of SWIGs extends beyond the scope of this tutorial. For
  more information, see Richardson and Robins
  (\citeproc{ref-richardson2013}{2013}).}

\subsubsection{Summary}\label{summary}

To consistently estimate causal effects, we must contrast the world as
it has been with the world as it might have been. For many questions in
evolutionary human science, we have seen that confounder-treatment
feedback leads to intractable causal identification problems. We have
also seen that causal diagrams are helpful in clarifying these problems.
Many self-inflicted injuries, such as mediator bias and
post-stratification bias, could be avoided if confounders were measured
prior to the exposures. Chronologically ordered causal diagrams aim to
make this basis transparent. They function as circuit-breakers that may
protect us from blowing up our causal inferences. More constructively,
temporal order in the graph focusses attention on imperatives for data
collection, offering guidance and hope.

\subsection{Conclusions}\label{conclusions}

Chronologically ordered causal diagrams provide significant enrichment
to causal inference endeavours. Their utility is not limited to just
modelling; they serve as valuable guides for data collection, too. When
used judiciously, within the frameworks of counterfactual data science
that support causal inference, causal diagrams can substantially enhance
the pursuit of accurate and robust causal understanding. Here is a
summary of advice.

\subsubsection{Tips}\label{tips}

\begin{enumerate}
\def\labelenumi{\arabic{enumi}.}
\item
  Clearly define all nodes on the graph. Ambiguity leads to confusion.
\item
  Simplify the graph by combining nodes where this is possible. Keep
  only those nodes and edges that are essential for clarifying the
  identification problem at hand. Avoid clutter.
\item
  Define any novel convention in your diagram explicitly. Do not assume
  familiarity.
\item
  Ensure acyclicity in the graph. This guarantees that a node cannot be
  its own ancestor, thereby eliminating circular paths.
\item
  Maintain chronological order spatially. Arrange nodes in temporal
  sequence, usually from left to right or top to bottom. Although it is
  not necessary to draw the sequence to scale, the order of events
  should be clear from the layout.
\item
  Time-stamp nodes. Causation happens over time; reflect this visually
  in the diagrams.
\item
  Be pragmatic. Use the \emph{modified disjunctive cause criterion} to
  minimise or possibly eliminate bias. As we discussed in Part 2, this
  criterion identifies a variable as part of a confounder set if it can
  reduce bias stemming from confounding, even if bias cannot be
  eliminated. Using this criterion will typically reduce your reliance
  on sensitivity analyses.
\item
  Draw nodes for unmeasured confounding where it aids confounding
  control strategies. Assume unmeasured confounding always exists,
  whether depicted on the graph or not. This assumption reveals the
  importance of sensitivity analyses when estimating causal effects.
\item
  Illustrate nodes for post-treatment selection. This facilitates
  understanding of potential sources of selection bias.
\item
  Apply a two-step strategy: Initially, isolate confounding bias and
  selection bias, then contemplate measurement bias using a secondary
  graph. This approach will foster clarity.\footnote{See Hernán and
    Robins (\citeproc{ref-hernuxe1n2023}{2023}) p.125}
\end{enumerate}

\begin{enumerate}
\def\labelenumi{\arabic{enumi}.}
\setcounter{enumi}{10}
\item
  Expand graphs to clarify relevant bias structures if mediation or
  interaction is of interest. However, do not attempt to draw non-linear
  associations between variables.
\item
  Remember, causal diagrams are qualitative tools encoding assumptions
  about causal ancestries. They are compasses, not comprehensive
  atlases.
\end{enumerate}

\subsubsection{Pitfalls}\label{pitfalls}

\begin{enumerate}
\def\labelenumi{\arabic{enumi}.}
\item
  Misunderstanding the role of causal diagrams within the framework of
  counter-factual data science.
\item
  The causal diagram contains variables without time indices. This
  omission may suggest that the researcher has not adequately considered
  the timing of events.
\item
  The graph has excessive nodes. No effort has been made to simplify the
  model by retaining only those nodes and edges essential for clarifying
  the identification problem.
\item
  The study is an experiment, but arrows are leading into the
  manipulation, revealing confusion.
\item
  Bias is incorrectly described. The exposure and outcome are
  d-separated, yet bias is claimed. This indicates a misunderstanding;
  the bias probably relates to generalisability or transportability, not
  to confounding.
\item
  Overlooking the representation of selection bias on the graph,
  particularly post-exposure selection bias from attrition or
  missingness.
\item
  Neglecting to use causal diagrams during the design phase of research
  before data collection.
\item
  Ignoring structural assumptions in classical measurement theory, such
  as in latent factor models, and blindly using construct measures
  derived from factor analysis.
\item
  Trying to represent interactions and non-linear dynamics on a causal
  diagram, which can lead to confusion about their purposes.
\item
  Failing to realise that structural equation models are not structural
  models. They are tools for statistical analysis, better termed as
  ``correlational equation models.'' Coefficients from these models
  often lack causal interpretations.
\item
  Neglecting the fact that conventional models such as multi-level (or
  mixed effects) models are unsuitable when treatment-confounder
  feedback is present. Illustrating treatment-confounder feedback on a
  graph underscores this point.\footnote{G-methods are appropriate for
    causal estimation in dynamic longitudinal settings. Their
    effectiveness notwithstanding, many evolutionary human scientists
    have not adopted them.{[}\^{}g-methods-cites{]} For good
    introductions see: Hernán and Robins
    (\citeproc{ref-hernuxe1n2023}{2023}) Díaz \emph{et al.}
    (\citeproc{ref-duxedaz2021}{2021}) VanderWeele
    (\citeproc{ref-vanderweele2015}{2015}) Hoffman \emph{et al.}
    (\citeproc{ref-hoffman2022}{2022}) Hoffman \emph{et al.}
    (\citeproc{ref-hoffman2023}{2023}) Chatton \emph{et al.}
    (\citeproc{ref-chatton2020}{2020}) Shiba and Kawahara
    (\citeproc{ref-shiba2021}{2021}) Sjölander
    (\citeproc{ref-sjuxf6lander2016}{2016}) Breskin \emph{et al.}
    (\citeproc{ref-breskin2020}{2020}) VanderWeele
    (\citeproc{ref-vanderweele2009a}{2009b}) Vansteelandt \emph{et al.}
    (\citeproc{ref-vansteelandt2012}{2012}) Shi \emph{et al.}
    (\citeproc{ref-shi2021}{2021}).)}
\end{enumerate}

\begin{enumerate}
\def\labelenumi{\arabic{enumi}.}
\setcounter{enumi}{11}
\tightlist
\item
  Failing to recognise that simple models work for time series data with
  three measurement intervals. A multi-level regression does not make
  sense for the three-wave panel design described in Part 3.
\end{enumerate}

\subsubsection{Concluding remarks}\label{concluding-remarks}

In causal analysis, the passage of time is not just another variable but
the stage on which the entire causal play unfolds. Time-ordered causal
diagrams articulate this temporal structure, revealing the necessity for
collecting time-series data in our quest to answer our causal questions.

This need places new demands on our research designs, funding
mechanisms, and the very rhythm of scientific investigation. Rather than
continuing in the high-throughput, assembly-line model of research,
where rapid publication may sometimes come at the expense of depth and
precision, we must pivot towards an approach that nurtures the careful
and extended collection of data over time.

The pace of scientific progress in the human sciences of causal
inference hinges on this transformation. Our challenge is not merely
methodological but institutional, requiring a shift in our scientific
culture towards one that values the slow but essential work of building
rich, time-resolved data sets.

\newpage{}

\subsection{Funding}\label{funding}

This work is supported by a grant from the Templeton Religion Trust
(TRT0418). JB received support from the Max Planck Institute for the
Science of Human History. The funders had no role in preparing the
manuscript or the decision to publish it.

\subsection{References}\label{references}

\newpage{}

\subsection{Appendix 1: The difficulty of satisfying the three
fundamental assumptions of causal inference when asking causal questions
of
history}\label{appendix-1-the-difficulty-of-satisfying-the-three-fundamental-assumptions-of-causal-inference-when-asking-causal-questions-of-history}

Consider the Protestant Reformation of the 16th century, which initiated
religious change throughout much of Europe. Historians have argued that
Protestantism caused social, cultural, and economic changes in those
societies where it took hold (see: (\citeproc{ref-basten2013}{Basten and
Betz 2013}; \citeproc{ref-swanson1967}{Swanson 1967};
\citeproc{ref-swanson1971}{Swanson 1971}; \citeproc{ref-weber1905}{Weber
1905}, \citeproc{ref-weber1993}{1993}), for an overview see:
(\citeproc{ref-becker2016}{Becker \emph{et al.} 2016})).

Suppose we are interested in estimating the ``Average Treatment Effect''
of the Protestant Reformation. Let \(A = a^*\) denote the adoption of
Protestantism. We compare this effect with that of remaining Catholic,
represented as \(A = a\). We assume that both the concepts of ``adopting
Protestantism'' and of ``economic development'' are well-defined
(e.g.~GDP +1 century after a country has a Protestant majority
contrasted with remaining Catholic). The causal effect for any
individual country is \(Y_i(a^*) - Y_i(a)\). Although we cannot identify
this effect, if the basic assumptions of causal inference are met, we
can estimate the average or marginal effect as

\[
\frac{1}{n} \sum_i^{n} \left[ Y_i(a^*) - Y_i(a) \right]
\]

which, conditioning the confounding effects of \(L\) gives us

\[ATE_{\textnormal{economic~development}} = \mathbb{E}[Y(\textnormal{Became~Protestant}|L) - Y(\textnormal{Remained~Catholic}|L)]\]

When asking causal questions about the economic effect of adopting
Protestantism versus remaining Catholic, there are indeed several
challenges that arise in relation to the three fundamental assumptions
required for causal inference.

\textbf{Causal Consistency}: requires the outcome under each level of
exposure is well-defined. In this context, defining what ``adopting
Protestantism'' and ``remaining Catholic'' mean may present challenges.
The practices and beliefs associated with each religion might vary
significantly across countries and time periods, and it may be difficult
to create a consistent, well-defined exposure. Furthermore, the outcome
- economic development - may also be challenging to measure consistently
across different countries and time periods.

There is undoubtedly considerable heterogeneity in the ``Protestant
exposure.'' In England, Protestantism was closely tied to the monarchy
(\citeproc{ref-collinson2007}{Collinson 2007}). In Germany, Martin
Luther's teachings emphasised individual faith in scripture, which, it
has been claimed, supported economic development by promoting literacy
(\citeproc{ref-gawthrop1984}{Gawthrop and Strauss 1984}). In England,
King Henry VIII abolished Catholicism
(\citeproc{ref-collinson2007}{Collinson 2007}). The Reformation, then,
occurred differently in different places. The exposure needs to be
better-defined.

There is also ample scope for interference: 16th century societies were
interconnected through trade, diplomacy, and warfare. Thus, the
religious decisions of one society were unlikely to have been
independent from those of other societies.

\textbf{Exchangeability}: requires that given the confounders, the
potential outcomes are independent of the treatment assignment. It might
be difficult to account for all possible confounders in this context.
For example, historical, political, social, and geographical factors
could influence both a country's religious affiliations and its economic
development. If these factors are not properly controlled, it could lead
to confounding bias.

\textbf{Positivity}: requires that there is a non-zero probability of
every level of exposure for every strata of confounders. If we consider
various confounding factors such as geographical location, historical
events, or political circumstances, some countries might only ever have
the possibility of either remaining Catholic or becoming Protestant, but
not both. For example, it is unclear under which conditions 16th century
Spain could have been randomly assigned to Protestantism
(\citeproc{ref-nalle1987}{Nalle 1987}).

Perhaps a more credible measure of effect in the region of our interests
is the Average Treatment Effect in the Treated (ATT) expressed

\[ATT_{\textnormal{economic~development}} = \mathbb{E}[(Y(a*)- Y(a))|A = a*,L]\]

Here, the ATT defines the expected difference in economic success for
cultures that became Protestant compared with the expected economic
success if those cultures had not become Protestant, conditional on
measured confounders \(L\), among the exposed (\(A = a^*\)). To estimate
this contrast, our models would need to match Protestant cultures with
comparable Catholic cultures effectively. By estimating the ATT, we
would avoid the assumption of non-deterministic positivity for the
untreated. However, whether matching is conceptually plausible remains
debatable. Ostensibly, it would seem that assigning a religion to a
culture a religion is not as easy as administering a pill
(\citeproc{ref-watts2018}{Watts \emph{et al.} 2018}).

\subsection{Appendix 2: Additional Terminology For Causal
Diagrams}\label{appendix-2-additional-terminology-for-causal-diagrams}

\begin{enumerate}
\def\labelenumi{\arabic{enumi}.}
\tightlist
\item
  \textbf{Nodes and Edges}:

  \begin{itemize}
  \tightlist
  \item
    \textbf{Nodes}: simple symbols in the diagram (such circles or dots)
    representing variables or events. For instance, in a study on social
    evolution, a node could signify a social behavior or an
    environmental factor.
  \item
    \textbf{Edges}: lines with a single arrow connecting nodes,
    indicating relationships between variables. A line between
    `enviornoment' and `social behaviour' encodes the assumption that
    environment affects social behaviour.
  \end{itemize}
\item
  \textbf{Types of Edges}:

  \begin{itemize}
  \tightlist
  \item
    \textbf{Directed Edges}: arrows showing cause-and-effect
    relationships. An arrow from `social behavior' to `population size'
    suggests social behavior influences population size.
  \item
    \textbf{Undirected Edges}: Straight lines without arrows, indicating
    an association without specifying direction or causality (these are
    of little utility for causal diagrams).
  \end{itemize}
\item
  \textbf{Ancestors and Descendants}:

  \begin{itemize}
  \tightlist
  \item
    \textbf{Ancestors}: nodes influencing others, directly or
    indirectly.
  \item
    \textbf{Descendants}: nodes influenced by others, again directly or
    indirectly.
  \end{itemize}
\end{enumerate}

For example, `historical events' might be an ancestor to `environmental
change' and `population size' might be a descendant of `social
behavior'. Causal graphs visually present these assumed relationships.

\begin{enumerate}
\def\labelenumi{\arabic{enumi}.}
\setcounter{enumi}{3}
\item
  \textbf{D-separation}: a concept to understand whether two nodes are
  independent given another variable or set of variables. If all paths
  between two nodes are `blocked', they are independent in this sense
  (\citeproc{ref-pearl2009}{Pearl 2009a}). .
\item
  \textbf{D-separation Rules}:

  \begin{itemize}
  \tightlist
  \item
    \textbf{Chain Rule}: \(A \rightarrow B \rightarrow C\): Conditioning
    on \(B\) makes \(A\) and \(C\) independent.
  \item
    \textbf{Fork Rule}:\(A \leftarrow B \rightarrow C\): Conditioning on
    \$ B \$ makes \$A \$ and \(C\)independent.
  \item
    \textbf{Collider Rule}:\(A \rightarrow B \leftarrow C\): \(A\) and
    \$ C \$ are independent unless \$ B \$ or its descendants are
    conditioned upon.
  \end{itemize}
\item
  \textbf{Adjustment set}: a collection of variables that we either
  condition upon or deliberately avoid conditioning upon to block all
  backdoor paths between the exposure and the outcome in the causal
  diagram (\citeproc{ref-pearl2009}{Pearl 2009a}).
\item
  \textbf{Confounders}: a member of an adjustment set. Importantly,
  \emph{we call a variable as a ``confounder'' in relation to a specific
  adjustment set.}
\item
  \textbf{Modified Disjunctive Cause Criterion}: VanderWeele's Modified
  Disjunctive Cause Criterion provides practical guidance for
  controlling for confounding
  (\citeproc{ref-vanderweele2019}{VanderWeele 2019}). According to this
  criterion, a member of any set of variables that can reduce or remove
  the bias caused by confounding is deemed a member of this confounder
  set. VanderWeele's strategy for defining a confounder set is as
  follows:
\end{enumerate}

\begin{enumerate}
\def\labelenumi{\alph{enumi}.}
\tightlist
\item
  Control for any variable that causes the exposure, the outcome, or
  both.
\item
  Control for any proxy for an unmeasured variable that is a shared
  cause of both exposure and outcome.
\item
  Define an instrumental variable as a variable associated with the
  exposure but does not influence the outcome independently, except
  through the exposure. Exclude any instrumental variable that is not a
  proxy for an unmeasured confounder from the confounder set.\footnote{Note
    that the concept of a ``confounder set'' is broader than the concept
    of an ``adjustment set.'' Every adjustment set is a member of a
    confounder set. So the Modified Disjunctive Cause Criterion will
    eliminate confounding when the data permit. However a confounder set
    includes variables that will reduce confounding in cases where
    confounding cannot be eliminated. Confounding can almost never be
    elimiated with certainty. For this reason we must perform
    sensitivity analyses to check the robustness of our results. These
    results will be less dependent on sensitivity analysis if we can
    reduce confounding. For this reason, I follow those who recommend
    using the Modified Disjunctive Cause Criterion for confounding
    control. Here, when focussing on strategies for attenuated
    confounding that cannot be fully controlled, I use dotted black
    directed edges to indicate attenuated confounding, and a blue
    directed edge to denote the association between the exposure and the
    outcome. Note that nearly every plausible scenario involving causal
    inference with observational data and non-random exposures presents
    a risk of unmeasured confounding. However, I refrain from
    universally applying this visualisation strategy to each graph to
    maintain focus on the specific issue each graph represents.}
\end{enumerate}

\begin{enumerate}
\def\labelenumi{\arabic{enumi}.}
\setcounter{enumi}{8}
\tightlist
\item
  \textbf{Compatibility and Faithfulness}: The idea that a dataset
  should reflect the conditional independencies suggested by a causal
  diagram and vice versa.(\citeproc{ref-pearl2009a}{Pearl 2009b};
  \citeproc{ref-pearl1995a}{Pearl and Robins 1995}).\footnote{Although
    the assumption of faithfulness or ``weak faithfulness'' allows for
    the possibility that some of the independences in the data might
    occur by coincidence (i.e., because of a cancellation of different
    effects), the assumption of strong faithfulness does not. The strong
    faithfulness condition assumes that the observed data's statistical
    relationships directly reflect the underlying causal structure, with
    no independence relationships arising purely by coincidental
    cancellations. This is a stronger assumption than (weak)
    faithfulness and is often more practical in real-world applications
    of causal inference. Note that the faithfulness assumption (whether
    weak or strong) is not testable by observed data -- it is an
    assumption about the relationship between the observed data and the
    underlying causal structure.}
\end{enumerate}

\begin{enumerate}
\def\labelenumi{\arabic{enumi}.}
\setcounter{enumi}{9}
\tightlist
\item
  \textbf{Markov Factorisation and the Causal Markov Assumption}: A
  principle that allows us to express complex relationships through
  simpler, conditional relationships.\footnote{Markov factorisation
    pertains to the connection between a causal diagram's structure and
    the distribution of the variables it depicts. It enables us to
    express the joint distribution of all variables as a product of
    simpler, conditional distributions. According to Markov
    factorisation, each variable in the diagram depends directly only on
    its parent variables and is independent of the others, thereby
    facilitating the graphical representation of complex relationships
    between multiple variables in a causal system
    (\citeproc{ref-lauritzen1990}{Lauritzen \emph{et al.} 1990};
    \citeproc{ref-pearl1988}{Pearl 1988}). The Causal Markov assumption
    states that any given variable, when conditioned on its direct
    antecedents, is rendered independent from all other variables that
    it does not cause (\citeproc{ref-hernuxe1n2023}{Hernán and Robins
    2023}). In essence, once we account for a variable's immediate
    causes, it ceases to provide additional causal information about any
    other variables in the system, except for those it directly causes.
    This assumption allows for inferring the causal effects of
    interventions in systems, as represented by causal diagrams
    (\citeproc{ref-pearl2009a}{Pearl 2009b}).}
\end{enumerate}

\begin{enumerate}
\def\labelenumi{\arabic{enumi}.}
\setcounter{enumi}{11}
\tightlist
\item
  \textbf{Backdoor Criterion}: Criteria to identify the correct set of
  variables to control for to estimate a causal effect. he backdoor
  criterion guides the selection of \textbf{adjustment sets}
  (\citeproc{ref-pearl1995}{Pearl 1995}).\footnote{There is also a
    Front-Door Criterion, which provides another way to estimate causal
    effects, even in the presence of unmeasured confounding variables.
    It relies on identifying a variable (or set of variables) that
    mediates the entire effect of the treatment on the outcome. The
    front-door criterion is rarely used in practice.}
\end{enumerate}

\begin{enumerate}
\def\labelenumi{\arabic{enumi}.}
\setcounter{enumi}{12}
\item
  \textbf{Identification Problem}:The challenge of estimating the causal
  effect of a variable using observed data. Causal diagrams were
  developed to address the identification problem.
\item
  \textbf{Diagram Acyclicity}: Causal diagrams must not contain loops;
  each variable should not be an ancestor or descendant of itself.
  \emph{Therefore, in cases where repeated measurements are taken, nodes
  must be indexed by time.}
\item
  \textbf{Effects Classification}: in the presence of mediating
  variables, it is helpful to differentiate the total effect (the
  overall effect of a variable \(A\) on an outcome \(Y\)), direct effect
  (the effect of \(A\) on \(Y\) not via any mediator), and indirect
  effect (the effect of \(A\) on \(Y\) via mediator). We consider the
  assumptions of causal mediation below
  (\citeproc{ref-vanderweele2015}{VanderWeele 2015}).
\item
  \textbf{Time-Varying Confounding:} this occurs when a confounder that
  changes over time also acts as a mediator in the causal pathway
  between exposure and outcome. Controlling for such a confounder can
  introduce bias. G-methods, a set of longitudinal methods, are
  typically utilised to address time-varying confounding. We discuss
  time-varying confounding at the end of Part 2
  (\citeproc{ref-hernuxe1n2023}{Hernán and Robins 2023}).
\item
  \textbf{Statistical vs Structural Models} a statistical model is a
  mathematical representation of the relationships between variables. It
  provides a framework to quantify how changes in one variable
  correspond with changes in others. Importantly, \textbf{statistical
  models can correspond to multiple causal structures}
  (\citeproc{ref-hernuxe1n2023}{Hernán and Robins 2023};
  \citeproc{ref-pearl2018}{Pearl and Mackenzie 2018};
  \citeproc{ref-wright1920}{Wright 1920},
  \citeproc{ref-wright1923}{1923}). Causal diagrams represent structural
  models. A structural model goes beyond a statistical model by defining
  assumptions about causal relationships. Although statistical models
  capture relationships among variables, inferring causal relationships
  necessitates additional assumptions or information. Causal diagrams
  serve to graphically encode these assumptions, effectively
  representing the structural model (\citeproc{ref-hernuxe1n2023}{Hernán
  and Robins 2023}). These assumptions should be developed in
  consultation with experts.
\item
  \textbf{A Structural Classification of Bias}:
\end{enumerate}

\begin{enumerate}
\def\labelenumi{\alph{enumi}.}
\item
  \emph{Confounding bias} occurs when the exposure and outcome share a
  common cause or condition on a common effect, distorting the true
  causal relationship between the exposure and outcome.
\item
  \emph{Selection bias} is a systematic error that arises when the
  individuals included in the study are not representative of the target
  population, leading to erroneous causal inferences from the data.
\item
  \emph{Measurement bias} occurs when the data collected inaccurately
  represents the true values of the variables being measured, distorting
  the observed relationship between the exposure and the outcome.
  (see:(\citeproc{ref-hernuxe1n2023}{Hernán and Robins 2023}))
\end{enumerate}

\subsection{Appendix 3: Review of VanderWeele's theory of causal
inference under multiple versions of
treatment}\label{appendix-3-review-of-vanderweeles-theory-of-causal-inference-under-multiple-versions-of-treatment}

We denote an average causal effect as the change in the expected
potential outcomes when all units receive one level of treatment
compared to another.

Let \(\delta\) denote the causal estimand on the difference scale
\((\mathbb{E}[Y^1 - Y^0])\). The causal effect identification can be
expressed as:

\[ \delta = \sum_l \left( \mathbb{E}[Y|A=a,l] - \mathbb{E}[Y|A=a^*,l] \right) P(l)\]

The theory of causal inference with multiple treatment versions provides
a conceptual framework for causal inference in observational studies.
Suppose we can assume that for each treatment version, the outcome under
that version equals the observed outcome when that version is
administered, conditional on baseline covariates and satisfaction of
other assumptions. In that case, we can consistently estimate causal
contrasts, even when treatments vary.

This approach interprets treatment indicator \(A\) as multiple actual
treatment versions \(K\). Furthermore, if we can assume conditional
independence, meaning there is no confounding for the effect of \(K\) on
\(Y\) given \(L\), we have: \(Y(k)\coprod A|K,L\).

This condition implies that, given \(L\), \(A\) adds no additional
information about \(Y\) after accounting for \(K\) and \(L\). If
\(Y = Y(k)\) for \(K = k\) and \(Y(k)\) is independent of \(K\),
conditional on \(L\), we can interpret \(A\) as a simplified indicator
of \(K\) (\citeproc{ref-vanderweele2013}{VanderWeele and Hernan 2013}).
This scenario is depicted in
Figure~\ref{fig-dag-multiple-version-treatment-dag}.

With the necessary assumptions in place, Vandeweele shows that can
derive consistent causal effects by proving:

\[\delta = \sum_{k,l} \left( \mathbb{E}[Y(k)|l] P(k|a,l) P(l) - \mathbb{E}[Y(k)|l] P(k|a^*,l) P(l) \right) \]

This setup is akin to a randomised trial where individuals, stratified
by covariate \(L\), are assigned a treatment version \(K\). This
assignment comes from the distribution of \(K\) for the
\((A = 1, L = l)\) subset. The control group receives a randomly
assigned \(K\) version from the \((A = 0, L = l)\) distribution.

\begin{figure}

\centering{

\includegraphics[width=1\textwidth,height=\textheight]{causal-dags_files/figure-pdf/fig-dag-multiple-version-treatment-dag-1.pdf}

}

\caption{\label{fig-dag-multiple-version-treatment-dag}Causal inference
under multiple versions of treatment. Here, (A) may be regarded as a
coarseneed indicator of (K)}

\end{figure}%

The theory of causal inference under multiple versions of treatment
reveal that consistent causal effect estimates are possible even when
treatments exhibit variability
(\citeproc{ref-vanderweele2013}{VanderWeele and Hernan 2013}). In Part
5, I explored VanderWeele's application of this theory to latent factor
models, where the presumption of a single underlying reality for the
items that constitute constructs can be challenged. VandnerWeele shows
that we may nevertheless, under assumptions of exchangeability,
consistenty estimate causal effects using a logic that parrallels the
theory of causal inference under multiple versions of treatment
(\citeproc{ref-vanderweele2022}{VanderWeele 2022}). I noted that the
possibility that directed or correlated error terms for the exposure and
outcome might nevertheless undermine inferences, and that such threats
may become more exaggerated with multiple items for our measures. I
noted that in place of general rules, researchers should be encouraged
to consider the problems of measurement in context.

\phantomsection\label{refs}
\begin{CSLReferences}{1}{0}
\bibitem[\citeproctext]{ref-barrett2021}
Barrett, M (2021) \emph{Ggdag: Analyze and create elegant directed
acyclic graphs}. Retrieved from
\url{https://CRAN.R-project.org/package=ggdag}

\bibitem[\citeproctext]{ref-basten2013}
Basten, C, and Betz, F (2013) Beyond work ethic: Religion, individual,
and political preferences. \emph{American Economic Journal: Economic
Policy}, \textbf{5}(3), 67--91.
doi:\href{https://doi.org/10.1257/pol.5.3.67}{10.1257/pol.5.3.67}.

\bibitem[\citeproctext]{ref-becker2016}
Becker, SO, Pfaff, S, and Rubin, J (2016) Causes and consequences of the
protestant reformation. \emph{Explorations in Economic History},
\textbf{62}, 125.

\bibitem[\citeproctext]{ref-breskin2020}
Breskin, A, Edmonds, A, Cole, SR, \ldots{} Adimora, AA (2020)
G-computation for policy-relevant effects of interventions on
time-to-event outcomes. \emph{International Journal of Epidemiology},
\textbf{49}(6), 2021--2029.
doi:\href{https://doi.org/10.1093/ije/dyaa156}{10.1093/ije/dyaa156}.

\bibitem[\citeproctext]{ref-bulbulia2022}
Bulbulia, JA (2022) A workflow for causal inference in cross-cultural
psychology. \emph{Religion, Brain \& Behavior}, \textbf{0}(0), 1--16.
doi:\href{https://doi.org/10.1080/2153599X.2022.2070245}{10.1080/2153599X.2022.2070245}.

\bibitem[\citeproctext]{ref-bulbulia2023}
Bulbulia, JA, Afzali, MU, Yogeeswaran, K, and Sibley, CG (2023)
Long-term causal effects of far-right terrorism in new zealand.
\emph{PNAS Nexus}, \textbf{2}(8), pgad242.

\bibitem[\citeproctext]{ref-bulbulia2021}
Bulbulia, J, Schjoedt, U, Shaver, JH, Sosis, R, and Wildman, WJ (2021)
Causal inference in regression: Advice to authors. \emph{Religion, Brain
\& Behavior}, \textbf{11}(4), 353360.

\bibitem[\citeproctext]{ref-chatton2020}
Chatton, A, Le Borgne, F, Leyrat, C, \ldots{} Foucher, Y (2020)
G-computation, propensity score-based methods, and targeted maximum
likelihood estimator for causal inference with different covariates
sets: a comparative simulation study. \emph{Scientific Reports},
\textbf{10}(1), 9219.
doi:\href{https://doi.org/10.1038/s41598-020-65917-x}{10.1038/s41598-020-65917-x}.

\bibitem[\citeproctext]{ref-cinelli2022}
Cinelli, C, Forney, A, and Pearl, J (2022) A Crash Course in Good and
Bad Controls. \emph{Sociological Methods \& Research},
00491241221099552.
doi:\href{https://doi.org/10.1177/00491241221099552}{10.1177/00491241221099552}.

\bibitem[\citeproctext]{ref-cole2010}
Cole, SR, Platt, RW, Schisterman, EF, \ldots{} Poole, C (2010)
Illustrating bias due to conditioning on a collider. \emph{International
Journal of Epidemiology}, \textbf{39}(2), 417--420.
doi:\href{https://doi.org/10.1093/ije/dyp334}{10.1093/ije/dyp334}.

\bibitem[\citeproctext]{ref-collinson2007}
Collinson, P (2007) \emph{The reformation: A history}, Vol. 19, Modern
Library.

\bibitem[\citeproctext]{ref-decoulanges1903}
De Coulanges, F (1903) \emph{La cité antique: Étude sur le culte, le
droit, les institutions de la grèce et de rome}, Hachette.

\bibitem[\citeproctext]{ref-duxedaz2021}
Díaz, I, Williams, N, Hoffman, KL, and Schenck, EJ (2021) Non-parametric
causal effects based on longitudinal modified treatment policies.
\emph{Journal of the American Statistical Association}.
doi:\href{https://doi.org/10.1080/01621459.2021.1955691}{10.1080/01621459.2021.1955691}.

\bibitem[\citeproctext]{ref-edwards2015}
Edwards, JK, Cole, SR, and Westreich, D (2015) All your data are always
missing: Incorporating bias due to measurement error into the potential
outcomes framework. \emph{International Journal of Epidemiology},
\textbf{44}(4), 14521459.

\bibitem[\citeproctext]{ref-gawthrop1984}
Gawthrop, R, and Strauss, G (1984) Protestantism and literacy in early
modern germany. \emph{Past \& Present}, (104), 3155.

\bibitem[\citeproctext]{ref-greenland1999}
Greenland, S, Pearl, J, and Robins, JM (1999) Causal diagrams for
epidemiologic research. \emph{Epidemiology (Cambridge, Mass.)},
\textbf{10}(1), 37--48.

\bibitem[\citeproctext]{ref-hernuxe1n2008}
Hernán, MA, Alonso, A, Logan, R, \ldots{} Robins, JM (2008)
Observational studies analyzed like randomized experiments: An
application to postmenopausal hormone therapy and coronary heart
disease. \emph{Epidemiology}, \textbf{19}(6), 766.
doi:\href{https://doi.org/10.1097/EDE.0b013e3181875e61}{10.1097/EDE.0b013e3181875e61}.

\bibitem[\citeproctext]{ref-hernuxe1n2004}
Hernán, MA, Hernández-Díaz, S, and Robins, JM (2004) A structural
approach to selection bias. \emph{Epidemiology}, \textbf{15}(5),
615--625. Retrieved from \url{https://www.jstor.org/stable/20485961}

\bibitem[\citeproctext]{ref-hernuxe1n2006}
Hernán, MA, and Robins, JM (2006) Estimating causal effects from
epidemiological data. \emph{Journal of Epidemiology \& Community
Health}, \textbf{60}(7), 578586.

\bibitem[\citeproctext]{ref-hernuxe1n2023}
Hernán, MA, and Robins, JM (2023) \emph{Causal inference: What if?},
Taylor \& Francis. Retrieved from
\url{https://books.google.co.nz/books?id=/_KnHIAAACAAJ}

\bibitem[\citeproctext]{ref-hernuxe1n2016}
Hernán, MA, Sauer, BC, Hernández-Díaz, S, Platt, R, and Shrier, I (2016)
Specifying a target trial prevents immortal time bias and other
self-inflicted injuries in observational analyses. \emph{Journal of
Clinical Epidemiology}, \textbf{79}, 7075.

\bibitem[\citeproctext]{ref-hernuxe1n2022a}
Hernán, MA, Wang, W, and Leaf, DE (2022) Target trial emulation: A
framework for causal inference from observational data. \emph{JAMA},
\textbf{328}(24), 2446--2447.
doi:\href{https://doi.org/10.1001/jama.2022.21383}{10.1001/jama.2022.21383}.

\bibitem[\citeproctext]{ref-hoffman2023}
Hoffman, KL, Salazar-Barreto, D, Rudolph, KE, and Díaz, I (2023)
Introducing longitudinal modified treatment policies: A unified
framework for studying complex exposures.
doi:\href{https://doi.org/10.48550/arXiv.2304.09460}{10.48550/arXiv.2304.09460}.

\bibitem[\citeproctext]{ref-hoffman2022}
Hoffman, KL, Schenck, EJ, Satlin, MJ, \ldots{} Díaz, I (2022) Comparison
of a target trial emulation framework vs cox regression to estimate the
association of corticosteroids with COVID-19 mortality. \emph{JAMA
Network Open}, \textbf{5}(10), e2234425.
doi:\href{https://doi.org/10.1001/jamanetworkopen.2022.34425}{10.1001/jamanetworkopen.2022.34425}.

\bibitem[\citeproctext]{ref-holland1986}
Holland, PW (1986) Statistics and causal inference. \emph{Journal of the
American Statistical Association}, \textbf{81}(396), 945960.

\bibitem[\citeproctext]{ref-hume1902}
Hume, D (1902) \emph{Enquiries Concerning the Human Understanding: And
Concerning the Principles of Morals}, Clarendon Press.

\bibitem[\citeproctext]{ref-lauritzen1990}
Lauritzen, SL, Dawid, AP, Larsen, BN, and Leimer, H-G (1990)
Independence properties of directed markov fields. \emph{Networks},
\textbf{20}(5), 491505.

\bibitem[\citeproctext]{ref-lewis1973}
Lewis, D (1973) Causation. \emph{The Journal of Philosophy},
\textbf{70}(17), 556--567.
doi:\href{https://doi.org/10.2307/2025310}{10.2307/2025310}.

\bibitem[\citeproctext]{ref-mcelreath2020}
McElreath, R (2020) \emph{Statistical rethinking: A bayesian course with
examples in r and stan}, CRC press.

\bibitem[\citeproctext]{ref-murray2021a}
Murray, EJ, Marshall, BDL, and Buchanan, AL (2021) Emulating target
trials to improve causal inference from agent-based models.
\emph{American Journal of Epidemiology}, \textbf{190}(8), 1652--1658.
doi:\href{https://doi.org/10.1093/aje/kwab040}{10.1093/aje/kwab040}.

\bibitem[\citeproctext]{ref-naimi2017}
Naimi, AI, Cole, SR, and Kennedy, EH (2017) An introduction to g
methods. \emph{International Journal of Epidemiology}, \textbf{46}(2),
756--762.
doi:\href{https://doi.org/10.1093/ije/dyw323}{10.1093/ije/dyw323}.

\bibitem[\citeproctext]{ref-nalle1987}
Nalle, ST (1987) Inquisitors, priests, and the people during the
catholic reformation in spain. \emph{The Sixteenth Century Journal},
557587.

\bibitem[\citeproctext]{ref-pearl1988}
Pearl, J (1988) \emph{Probabilistic reasoning in intelligent systems:
Networks of plausible inference}, Morgan kaufmann.

\bibitem[\citeproctext]{ref-pearl1995}
Pearl, J (1995) Causal diagrams for empirical research.
\emph{Biometrika}, \textbf{82}(4), 669--688.
doi:\href{https://doi.org/10.1093/biomet/82.4.669}{10.1093/biomet/82.4.669}.

\bibitem[\citeproctext]{ref-pearl2009}
Pearl, J (2009a) \emph{\href{https://doi.org/10.1214/09-SS057}{Causal
inference in statistics: An overview}}.

\bibitem[\citeproctext]{ref-pearl2009a}
Pearl, J (2009b) \emph{Causality}, Cambridge University Press.

\bibitem[\citeproctext]{ref-pearl2018}
Pearl, J, and Mackenzie, D (2018) \emph{The book of why: The new science
of cause and effect}, Basic books.

\bibitem[\citeproctext]{ref-pearl1995a}
Pearl, J, and Robins, JM (1995) Probabilistic evaluation of sequential
plans from causal models with hidden variables. In, Vol. 95, Citeseer,
444453.

\bibitem[\citeproctext]{ref-richardson2013}
Richardson, TS, and Robins, JM (2013) Single world intervention graphs
(SWIGs): A unification of the counterfactual and graphical approaches to
causality. \emph{Center for the Statistics and the Social Sciences,
University of Washington Series. Working Paper}, \textbf{128}(30), 2013.

\bibitem[\citeproctext]{ref-robins1986}
Robins, J (1986) A new approach to causal inference in mortality studies
with a sustained exposure period{\textemdash}application to control of
the healthy worker survivor effect. \emph{Mathematical Modelling},
\textbf{7}(9), 1393--1512.
doi:\href{https://doi.org/10.1016/0270-0255(86)90088-6}{10.1016/0270-0255(86)90088-6}.

\bibitem[\citeproctext]{ref-robins1999}
Robins, JM, Greenland, S, and Hu, F-C (1999) Estimation of the causal
effect of a time-varying exposure on the marginal mean of a repeated
binary outcome. \emph{Journal of the American Statistical Association},
\textbf{94}(447), 687--700.
doi:\href{https://doi.org/10.1080/01621459.1999.10474168}{10.1080/01621459.1999.10474168}.

\bibitem[\citeproctext]{ref-rohrer2018}
Rohrer, JM (2018) Thinking clearly about correlations and causation:
Graphical causal models for observational data. \emph{Advances in
Methods and Practices in Psychological Science}, \textbf{1}(1), 2742.

\bibitem[\citeproctext]{ref-rubin1976}
Rubin, DB (1976) Inference and missing data. \emph{Biometrika},
\textbf{63}(3), 581--592.
doi:\href{https://doi.org/10.1093/biomet/63.3.581}{10.1093/biomet/63.3.581}.

\bibitem[\citeproctext]{ref-shi2021}
Shi, B, Choirat, C, Coull, BA, VanderWeele, TJ, and Valeri, L (2021)
CMAverse: A suite of functions for reproducible causal mediation
analyses. \emph{Epidemiology}, \textbf{32}(5), e20e22.

\bibitem[\citeproctext]{ref-shiba2021}
Shiba, K, and Kawahara, T (2021) Using propensity scores for causal
inference: Pitfalls and tips. \emph{Journal of Epidemiology},
\textbf{31}(8), 457463.

\bibitem[\citeproctext]{ref-sjuxf6lander2016}
Sjölander, A (2016) Regression standardization with the R package
stdReg. \emph{European Journal of Epidemiology}, \textbf{31}(6),
563--574.
doi:\href{https://doi.org/10.1007/s10654-016-0157-3}{10.1007/s10654-016-0157-3}.

\bibitem[\citeproctext]{ref-suzuki2020}
Suzuki, E, Shinozaki, T, and Yamamoto, E (2020) Causal Diagrams:
Pitfalls and Tips. \emph{Journal of Epidemiology}, \textbf{30}(4),
153--162.
doi:\href{https://doi.org/10.2188/jea.JE20190192}{10.2188/jea.JE20190192}.

\bibitem[\citeproctext]{ref-swanson1967}
Swanson, GE (1967) Religion and regime: A sociological account of the
reformation.

\bibitem[\citeproctext]{ref-swanson1971}
Swanson, GE (1971) Interpreting the reformation. \emph{The Journal of
Interdisciplinary History}, \textbf{1}(3), 419446. Retrieved from
\url{http://www.jstor.org/stable/202620}

\bibitem[\citeproctext]{ref-tripepi2007}
Tripepi, G, Jager, KJ, Dekker, FW, Wanner, C, and Zoccali, C (2007)
Measures of effect: Relative risks, odds ratios, risk difference, and
{`}number needed to treat{'}. \emph{Kidney International},
\textbf{72}(7), 789--791.
doi:\href{https://doi.org/10.1038/sj.ki.5002432}{10.1038/sj.ki.5002432}.

\bibitem[\citeproctext]{ref-vanderweele2015}
VanderWeele, T (2015) \emph{Explanation in causal inference: Methods for
mediation and interaction}, Oxford University Press.

\bibitem[\citeproctext]{ref-vanderweele2009}
VanderWeele, TJ (2009a) Concerning the consistency assumption in causal
inference. \emph{Epidemiology}, \textbf{20}(6), 880.
doi:\href{https://doi.org/10.1097/EDE.0b013e3181bd5638}{10.1097/EDE.0b013e3181bd5638}.

\bibitem[\citeproctext]{ref-vanderweele2009a}
VanderWeele, TJ (2009b) Marginal structural models for the estimation of
direct and indirect effects. \emph{Epidemiology}, 1826.

\bibitem[\citeproctext]{ref-vanderweele2018}
VanderWeele, TJ (2018) On well-defined hypothetical interventions in the
potential outcomes framework. \emph{Epidemiology}, \textbf{29}(4), e24.
doi:\href{https://doi.org/10.1097/EDE.0000000000000823}{10.1097/EDE.0000000000000823}.

\bibitem[\citeproctext]{ref-vanderweele2019}
VanderWeele, TJ (2019) Principles of confounder selection.
\emph{European Journal of Epidemiology}, \textbf{34}(3), 211--219.
doi:\href{https://doi.org/10.1007/s10654-019-00494-6}{10.1007/s10654-019-00494-6}.

\bibitem[\citeproctext]{ref-vanderweele2022}
VanderWeele, TJ (2022) Constructed measures and causal inference:
Towards a new model of measurement for psychosocial constructs.
\emph{Epidemiology}, \textbf{33}(1), 141.
doi:\href{https://doi.org/10.1097/EDE.0000000000001434}{10.1097/EDE.0000000000001434}.

\bibitem[\citeproctext]{ref-vanderweele2013}
VanderWeele, TJ, and Hernan, MA (2013) Causal inference under multiple
versions of treatment. \emph{Journal of Causal Inference},
\textbf{1}(1), 120.

\bibitem[\citeproctext]{ref-vanderweele2014}
VanderWeele, TJ, and Knol, MJ (2014) A tutorial on interaction.
\emph{Epidemiologic Methods}, \textbf{3}(1), 3372.

\bibitem[\citeproctext]{ref-vanderweele2020}
VanderWeele, TJ, Mathur, MB, and Chen, Y (2020) Outcome-wide
longitudinal designs for causal inference: A new template for empirical
studies. \emph{Statistical Science}, \textbf{35}(3), 437466.

\bibitem[\citeproctext]{ref-vanderweele2007}
VanderWeele, TJ, and Robins, JM (2007) Four types of effect
modification: a classification based on directed acyclic graphs.
\emph{Epidemiology (Cambridge, Mass.)}, \textbf{18}(5), 561--568.
doi:\href{https://doi.org/10.1097/EDE.0b013e318127181b}{10.1097/EDE.0b013e318127181b}.

\bibitem[\citeproctext]{ref-vanderweele2022b}
VanderWeele, TJ, and Vansteelandt, S (2022) A statistical test to reject
the structural interpretation of a latent factor model. \emph{Journal of
the Royal Statistical Society Series B: Statistical Methodology},
\textbf{84}(5), 20322054.

\bibitem[\citeproctext]{ref-vansteelandt2012}
Vansteelandt, S, Bekaert, M, and Lange, T (2012) Imputation strategies
for the estimation of natural direct and indirect effects.
\emph{Epidemiologic Methods}, \textbf{1}(1), 131158.

\bibitem[\citeproctext]{ref-watts2016}
Watts, J, Bulbulia, J. A., Gray, RD, and Atkinson, QD (2016) Clarity and
causality needed in claims about big gods., \textbf{39}, 4142.
doi:\href{https://doi.org/d4qp}{d4qp}.

\bibitem[\citeproctext]{ref-watts2018}
Watts, J, Sheehan, O, Bulbulia, Joseph A, Gray, RD, and Atkinson, QD
(2018) Christianity spread faster in small, politically structured
societies. \emph{Nature Human Behaviour}, \textbf{2}(8), 559564.
doi:\href{https://doi.org/gdvnjn}{gdvnjn}.

\bibitem[\citeproctext]{ref-weber1905}
Weber, M (1905) \emph{The protestant ethic and the spirit of capitalism:
And other writings}, Penguin.

\bibitem[\citeproctext]{ref-weber1993}
Weber, M (1993) \emph{The sociology of religion}, Beacon Press.

\bibitem[\citeproctext]{ref-westreich2015}
Westreich, D, Edwards, JK, Cole, SR, Platt, RW, Mumford, SL, and
Schisterman, EF (2015) Imputation approaches for potential outcomes in
causal inference. \emph{International Journal of Epidemiology},
\textbf{44}(5), 17311737.

\bibitem[\citeproctext]{ref-wheatley1971}
Wheatley, P (1971) \emph{The pivot of the four quarters : A preliminary
enquiry into the origins and character of the ancient chinese city},
Edinburgh University Press. Retrieved from
\url{https://cir.nii.ac.jp/crid/1130000795717727104}

\bibitem[\citeproctext]{ref-williams2021}
Williams, NT, and Díaz, I (2021) \emph{Lmtp: Non-parametric causal
effects of feasible interventions based on modified treatment policies}.
doi:\href{https://doi.org/10.5281/zenodo.3874931}{10.5281/zenodo.3874931}.

\bibitem[\citeproctext]{ref-wright1920}
Wright, S (1920) The relative importance of heredity and environment in
determining the piebald pattern of guinea-pigs. \emph{Proceedings of the
National Academy of Sciences of the United States of America},
\textbf{6}(6), 320.

\bibitem[\citeproctext]{ref-wright1923}
Wright, S (1923) The theory of path coefficients a reply to niles's
criticism. \emph{Genetics}, \textbf{8}(3), 239.

\end{CSLReferences}



\end{document}
