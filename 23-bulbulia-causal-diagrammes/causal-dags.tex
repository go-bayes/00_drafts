% Options for packages loaded elsewhere
\PassOptionsToPackage{unicode}{hyperref}
\PassOptionsToPackage{hyphens}{url}
\PassOptionsToPackage{dvipsnames,svgnames,x11names}{xcolor}
%
\documentclass[
  singlecolumn,
  9pt]{article}

\usepackage{amsmath,amssymb}
\usepackage{iftex}
\ifPDFTeX
  \usepackage[T1]{fontenc}
  \usepackage[utf8]{inputenc}
  \usepackage{textcomp} % provide euro and other symbols
\else % if luatex or xetex
  \usepackage{unicode-math}
  \defaultfontfeatures{Scale=MatchLowercase}
  \defaultfontfeatures[\rmfamily]{Ligatures=TeX,Scale=1}
\fi
\usepackage[]{libertinus}
\ifPDFTeX\else  
    % xetex/luatex font selection
\fi
% Use upquote if available, for straight quotes in verbatim environments
\IfFileExists{upquote.sty}{\usepackage{upquote}}{}
\IfFileExists{microtype.sty}{% use microtype if available
  \usepackage[]{microtype}
  \UseMicrotypeSet[protrusion]{basicmath} % disable protrusion for tt fonts
}{}
\makeatletter
\@ifundefined{KOMAClassName}{% if non-KOMA class
  \IfFileExists{parskip.sty}{%
    \usepackage{parskip}
  }{% else
    \setlength{\parindent}{0pt}
    \setlength{\parskip}{6pt plus 2pt minus 1pt}}
}{% if KOMA class
  \KOMAoptions{parskip=half}}
\makeatother
\usepackage{xcolor}
\usepackage[top=30mm,bottom=30mm,left=20mm,heightrounded]{geometry}
\setlength{\emergencystretch}{3em} % prevent overfull lines
\setcounter{secnumdepth}{-\maxdimen} % remove section numbering
% Make \paragraph and \subparagraph free-standing
\ifx\paragraph\undefined\else
  \let\oldparagraph\paragraph
  \renewcommand{\paragraph}[1]{\oldparagraph{#1}\mbox{}}
\fi
\ifx\subparagraph\undefined\else
  \let\oldsubparagraph\subparagraph
  \renewcommand{\subparagraph}[1]{\oldsubparagraph{#1}\mbox{}}
\fi


\providecommand{\tightlist}{%
  \setlength{\itemsep}{0pt}\setlength{\parskip}{0pt}}\usepackage{longtable,booktabs,array}
\usepackage{calc} % for calculating minipage widths
% Correct order of tables after \paragraph or \subparagraph
\usepackage{etoolbox}
\makeatletter
\patchcmd\longtable{\par}{\if@noskipsec\mbox{}\fi\par}{}{}
\makeatother
% Allow footnotes in longtable head/foot
\IfFileExists{footnotehyper.sty}{\usepackage{footnotehyper}}{\usepackage{footnote}}
\makesavenoteenv{longtable}
\usepackage{graphicx}
\makeatletter
\def\maxwidth{\ifdim\Gin@nat@width>\linewidth\linewidth\else\Gin@nat@width\fi}
\def\maxheight{\ifdim\Gin@nat@height>\textheight\textheight\else\Gin@nat@height\fi}
\makeatother
% Scale images if necessary, so that they will not overflow the page
% margins by default, and it is still possible to overwrite the defaults
% using explicit options in \includegraphics[width, height, ...]{}
\setkeys{Gin}{width=\maxwidth,height=\maxheight,keepaspectratio}
% Set default figure placement to htbp
\makeatletter
\def\fps@figure{htbp}
\makeatother
% definitions for citeproc citations
\NewDocumentCommand\citeproctext{}{}
\NewDocumentCommand\citeproc{mm}{%
  \begingroup\def\citeproctext{#2}\cite{#1}\endgroup}
% avoid brackets around text for \cite:
\makeatletter
 \def\@biblabel#1{}
 \def\@cite#1#2{{#1\if@tempswa , #2\fi}}
\makeatother
\newlength{\cslhangindent}
\setlength{\cslhangindent}{1.5em}
\newlength{\csllabelwidth}
\setlength{\csllabelwidth}{3em}
\newlength{\cslentryspacing}
\setlength{\cslentryspacing}{0em}
\usepackage{enumitem}
\newlist{CSLReferences}{itemize}{1}
\setlist[CSLReferences]{label={},
  leftmargin=\cslhangindent,
  itemindent=-1\cslhangindent,
  parsep=\parskip,
  itemsep=\cslentryspacing}
\usepackage{calc}
\newcommand{\CSLBlock}[1]{#1\hfill\break}
\newcommand{\CSLLeftMargin}[1]{\parbox[t]{\csllabelwidth}{#1}}
\newcommand{\CSLRightInline}[1]{\parbox[t]{\linewidth - \csllabelwidth}{#1}\break}
\newcommand{\CSLIndent}[1]{\hspace{\cslhangindent}#1}

\usepackage{cancel}
\usepackage{xcolor}
\usepackage[noblocks]{authblk}
\renewcommand*{\Authsep}{, }
\renewcommand*{\Authand}{, }
\renewcommand*{\Authands}{, }
\renewcommand\Affilfont{\small}
\usepackage{cancel}
\usepackage{xcolor}
\makeatletter
\makeatother
\makeatletter
\makeatother
\makeatletter
\@ifpackageloaded{caption}{}{\usepackage{caption}}
\AtBeginDocument{%
\ifdefined\contentsname
  \renewcommand*\contentsname{Table of contents}
\else
  \newcommand\contentsname{Table of contents}
\fi
\ifdefined\listfigurename
  \renewcommand*\listfigurename{List of Figures}
\else
  \newcommand\listfigurename{List of Figures}
\fi
\ifdefined\listtablename
  \renewcommand*\listtablename{List of Tables}
\else
  \newcommand\listtablename{List of Tables}
\fi
\ifdefined\figurename
  \renewcommand*\figurename{Figure}
\else
  \newcommand\figurename{Figure}
\fi
\ifdefined\tablename
  \renewcommand*\tablename{Table}
\else
  \newcommand\tablename{Table}
\fi
}
\@ifpackageloaded{float}{}{\usepackage{float}}
\floatstyle{ruled}
\@ifundefined{c@chapter}{\newfloat{codelisting}{h}{lop}}{\newfloat{codelisting}{h}{lop}[chapter]}
\floatname{codelisting}{Listing}
\newcommand*\listoflistings{\listof{codelisting}{List of Listings}}
\makeatother
\makeatletter
\@ifpackageloaded{caption}{}{\usepackage{caption}}
\@ifpackageloaded{subcaption}{}{\usepackage{subcaption}}
\makeatother
\makeatletter
\makeatother
\ifLuaTeX
  \usepackage{selnolig}  % disable illegal ligatures
\fi
\IfFileExists{bookmark.sty}{\usepackage{bookmark}}{\usepackage{hyperref}}
\IfFileExists{xurl.sty}{\usepackage{xurl}}{} % add URL line breaks if available
\urlstyle{same} % disable monospaced font for URLs
\hypersetup{
  pdftitle={Chronological Ordered Causal Diagrams for the Evolutionary Human Sciences: A Practical Guide},
  pdfauthor={Joseph A. Bulbulia},
  pdfkeywords={Directed Acyclic Graph, Causal
Inference, Confounding, Feedback, Interaction, Mediation, Moderation, Panel},
  colorlinks=true,
  linkcolor={blue},
  filecolor={Maroon},
  citecolor={Blue},
  urlcolor={Blue},
  pdfcreator={LaTeX via pandoc}}

\title{Chronological Ordered Causal Diagrams for the Evolutionary Human
Sciences: A Practical Guide}


  \author{Joseph A. Bulbulia}
            \affil{%
                  Victoria University of Wellington, New Zealand, School
                  of Psychology, Centre for Applied Cross-Cultural
                  Research
              }
      
\date{2023-11-28}
\begin{document}
\maketitle
\begin{abstract}
Causation inherently unfolds in time. However, quantifying a causal
effect relies on contrasting counterfactual states of the world that
never occur. As such, causal data science relies on explicit assumptions
and careful, multi-stepped workflows. Within this framework, causal
diagrams have been developed as powerful tools for evaluating structural
assumptions necessary for obtaining consistent causal effect estimates
from data. If used without attention to this demands of this framework
causal diagrams may be easily misused. This guide offers practical
advice for creating effective causal diagrams, beginning with a review
of the causal data science framework. Next, I develop a series of
examples that illustrate the benefits of chronological order in the
spacial organisation of one's graph, both for data analysis and data
collection. I conclude by using chronologically ordered causal diagrams
to elucidate the widely misunderstood concepts of interaction
(`moderation'), mediation, and dynamic longitudinal feedback. Again we
discover the benefits of combining causal theory with a chronologically
ordered graph. I hope this guide helps those evolutionary human
scientists to develop the combination of conceptual understanding and
practical skills that will enhance the clarity of their causal
questions, and quality and enjoyment in their pursuits for answers.
\end{abstract}
\subsection{Introduction}\label{introduction}

Correlation does not imply causation. This adage is widely known.
Nevertheless, many human scientists report manifest correlations and use
hedging language that implies causation. I have been guilty. However,
such reporting typically lacks justification. Making matters worse,
widely adopted analytic strategies for confounding control, such as
indiscriminate co-variate adjustment, are known to enhance bias
(\citeproc{ref-mcelreath2020}{McElreath 2020}). Across many human
sciences, including the evolutionary human sciences, persistent
confusion in the analysis and reporting of correlations has greatly
impeded scientific progress. In short, many of the disciplines that
inform the evolutionary sciences are experiencing a `causality crisis'
(\citeproc{ref-bulbulia2022}{Bulbulia 2022})

There is hope. First, the open science movement has demonstrated that
greater attention to the `replication crisis' in the experimental human
sciences can bring considerable improvements within a relatively short
span of time. Although much remains to be done still, basic corrective
practices for open science have become normative. Second, several
decades of active development of causal data science in the health
sciences, computer sciences, economics, and several social sciences have
yielded both considerable conceptual clarifications and rigorous
analytic toolkits for inference (\citeproc{ref-hernan2023}{Hernan and
Robins 2023}; \citeproc{ref-neyman1923}{Neyman 1923};
\citeproc{ref-pearl2009a}{Pearl 2009b}; \citeproc{ref-pearl1995}{Pearl
and Robins 1995}; \citeproc{ref-robins1986}{Robins 1986};
\citeproc{ref-rubin1976}{Rubin 1976};
\citeproc{ref-vanderweele2015}{VanderWeele 2015}). Although causal data
science is still evolving (\citeproc{ref-duxedaz2021}{Díaz \emph{et al.}
2021}; \citeproc{ref-hoffman2023}{Hoffman \emph{et al.} 2023};
\citeproc{ref-vansteelandt2022}{Vansteelandt and Dukes 2022}). Thus, a
substantial foundation for causal data science already exists. Given
these precedents, we should be optimistic that rapid uptake of causal
data science in those fields where causal confusion currently holds sway
is feasible. The articles in this special issue of \emph{Evolutionary
Human Sciences} offer testimony to this hope.

Causal diagrams, also known as `directed acyclic graphs' or `DAGs,' are
powerful inferential tools for obtain valid causal inferences. These
tools rest on a robust system of formal mathematical proofs, that should
instil confidence in their appropriate use. Yet they do not require
mathematical training, and are broadly accessible, at least for the
sighted. This is a great advantage.

Yet there causal diagrams invite risks. The tool acquires its
significance only when integrated within the broader theoretical
frameworks causal data science. This framework distinguishes itself from
traditional data science by attempting to estimate pre-specified
contrasts, or `estimands', among countefactual states of the world that,
although assumed to be real, never occur. Rather, the required
counterfactual scenarios are simulated from data under explicit
assumptions about the data, and that laws of nature that generate the
data. These structural assumptions differ from the statistical
assumptions familiar to many statisticians and computer scientists and
routinely taught in these disciplines. Although because we must
eventually use models, statistical validation must also enter the
workflow. Put simply, Causal Data Science is a form of `counterfactual
data science' or `full data science' in which the data we observe
provide only partial insights into the targeted counterfactual contrasts
and their uncertainties (\citeproc{ref-bulbulia2023}{Bulbulia 2023};
\citeproc{ref-ogburn2021}{Ogburn and Shpitser 2021}). Using causal
diagrams without a understanding of their role within the framework of
theory and assumptions that underpin Causal Data Science risks
inadvertently worsen the causality crisis by fostering misguided
confidence where none is due.

In this work, I offer readers of \emph{Evolutionary Human Science}
practical guidance for creating causal diagrams that work as we intend,
while also mitigating risks of overreaching.

\textbf{Part 1} introduces core concepts and theory that underpin the
frameworks of applied Causal Data Science, emphasising fundamental
assumptions and workflows. Although this overview is brief, it it
provides an orientation to the wider workflow in which causal diagrams
possess their utility for obtaining valid causal inferences from
observational data, and outside of which the application of causal
diagrams offers no guarantees.

\textbf{Part 2} introduces \textbf{chronologically ordered causal
diagrams} and their applications for addressing confounding bias. Here,
we illustrate how maintaining `chronological hygiene' in the spatial
layout of a causal diagram is useful not only for the tasks for
developing sound data-analytic strategies but also for research planning
and data-collection. Although, chronological ordering is not strictly
essential, and indeed is not widely practised, the examples we consider
demonstrate its advantages in common scenarios.

\textbf{Part 3} uses chronologically ordered causal diagrams, again
applied within the broader framework of causal data science, to
demystify the widely misunderstood concepts of interaction (moderation),
mediation, and longitudinal data analysis. Again we find that the
frameworks of causal data science are indispensable for clarifying the
quantities researchers hope to understand when applying statistical
models in these areas. We again discover that attention to chronological
hygiene is great benefit both for data analysis and collection. Notably,
we discover that certain seemingly accessible questions, such as how
much of total effect is mediated, cannot be directly evaluated by
observational data, even at the limit of perfect data collection.
Unfortunately, question of interaction, mediation, and longitudinal
feedback remain poorly served by analytic traditions many human
scientists and statisticians were trained on, such as the structural
equation modelling tradition. These traditions continue to dominate,
thus perpetuating considerable confusions. We can do better, and should.

There are numerous excellent resources available for learning causal
diagrams, which I highly recommend to readers
(\citeproc{ref-barrett2021}{Barrett 2021};
\citeproc{ref-cinelli2022}{Cinelli \emph{et al.} 2022};
\citeproc{ref-greenland1999}{Greenland \emph{et al.} 1999a};
\citeproc{ref-hernuxe1n2023}{Hernán and Monge 2023};
\citeproc{ref-mcelreath2020}{McElreath 2020};
\citeproc{ref-pearl2009}{Pearl 2009a}; \citeproc{ref-rohrer2018}{Rohrer
2018}; \citeproc{ref-suzuki2020}{Suzuki \emph{et al.}
2020a}).\footnote{An excellent resource is Miguel Hernán's free online
  course, here:
  \url{https://pll.harvard.edu/course/causal-diagrams-draw-your-assumptions-your-conclusions}.}
My work here hopes to contribute by providing additional conceptual
orientation to the frameworks of Causal Data Science in which causal
diagrams have their place, by underscoring the benefits of chronological
hygiene for common problems, and by applying this orientation to
concepts of interaction, mediation, and longitudinal feedback, about
which there remains considerable yet easily dissipated confusions.

\subsection{Part 1. Overview of Causal Data
Science}\label{part-1.-overview-of-causal-data-science}

The critical first step in answering a causal question is formulating
it(\citeproc{ref-hernuxe1n2016}{Hernán \emph{et al.} 2016}). Causal
diagrams come later, when we consider which forms of data might enable
us to address our pre-specified causal questions. This section
introduces key concepts and broader workflows within which causal
diagrams find their purposes and utilities.

\subsubsection{The Fundamental Problem of Causal
Inference}\label{the-fundamental-problem-of-causal-inference}

Consider an intervention, \(A\), and its effect, \(Y\). We say that
\(A\) causes \(Y\) if altering \(A\) would lead to a change in \(Y\)
(\citeproc{ref-hume1902}{Hume 1902}; \citeproc{ref-lewis1973}{Lewis
1973}). If altering \(A\) would not lead to a change in \(Y\), then we
say that \(A\) has no causal effect on \(Y\).

In causal data science, our objective is to measure a contrast in a
well-defined outcome \(Y\) when subjected to different levels of a
well-defined intervention \(A\). Commonly, we refer to such
interventions as `exposures' or `treatments;' we refer to the possible
effects of interventions as `potential outcomes.'

Let us assume that \(A\) exists in two states: \(A \in \{0,1\}\). We
denote the potential outcome when \(A\) is set to \(0\) as \(Y(0)\) and
when \(A\) is set to 1 as \(Y(1)\). We call \(Y(1), Y(0)\) potential
outcomes. Suppose we have specified a well-defined exposure and outcome.
Each unit, \(i \dots, n\), can either experience \(Y_i|A_i = 1\) or
\(Y_i|A_i = 0\). However, no units can experience both simultaneously.
As a result, we cannot directly calculate a contrast between \(Y_i(1)\)
and \(Y_i(0)\) from observable data. Where \(\delta_i\) is the quantity
of interest,

\[
\delta_i = Y_i(1) - Y_i(0)
\]

\(\delta_i\) is unavailable at any given time because the realisation of
one of the potential outcomes to be contrasted prevents the realisation
of the other.

We may be tempted to ask, `What if Isaac Newton had not observed the
falling apple?' `What if Leonardo da Vinci had never pursued art?' or
`What if Archduke Ferdinand had not been assassinated?' We may
speculate, but we cannot directly observe the answers. The physics of
middle-sized dry goods entails that the joint distribution of the
alternative histories that we require for comparisons.

A distinctive and important feature of causal data science is the
assumption that, although never jointly realised, the potential outcomes
\(Y(1),Y(0)\) must nevertheless be assumed to be real, and to exist
independently of data collection.\footnote{As Hernán and Robins point
  out: ``Sometimes we abbreviate the expression individual \(i\) has
  outcome \(Y^a = 1\) by writing \(Y^a_i = 1\). Technically, when \(i\)
  refers to a specific individual, such as Zeus, \(Y^a_i\) is not a
  random variable because we are assuming that individual counterfactual
  outcomes are deterministic\ldots{} Causal effect for individual
  \(i: Y^{a=1}\neq Y^{a=0}\)'' (\citeproc{ref-hernan2023}{Hernan and
  Robins 2023 p. 6})} As such, causal data science faces a unique type
of missing data problem in which the `full data' needed to compute any
causal contrast is missing at least half of its values
(\citeproc{ref-edwards2015}{Edwards \emph{et al.} 2015};
\citeproc{ref-ogburn2021}{Ogburn and Shpitser 2021};
\citeproc{ref-westreich2015}{Westreich \emph{et al.} 2015}). This
challenge is distinct from typical missing data scenarios where data
could have been recorded but were not. The missing information crucial
for computing causal contrasts is intrinsically linked to the
irreversible nature of time.

\subsubsection{Obtaining Average Causal Effects From
Observations}\label{obtaining-average-causal-effects-from-observations}

In typical scenarios, computing individual causal effects is not
feasible. However, under certain assumptions, it is possible to credibly
calculate average causal effects. That is, we may obtain average
treatment effects by contrasting groups that have received different
levels of a treatment. On a difference scale, the average treatment
effect (\(\Delta_{ATE}\))) may be expressed,

\[
\Delta_{ATE}  = \mathbb{E}[Y(1)] - \mathbb{E}[Y(0)]
\]

Here, \(\mathbb{E}\) denotes the average response of all individuals
within an exposure group, and \(Y(1)\) and \(Y(0)\) represent the
potential outcomes under interventions \(A = 1\) and \(A = 0\),
respectively.

If individual causal effects are not observable, how might we obtain
their averages? The answer is that we do so by assumptions. To see this,
it is helpful to consider how randomised experiments obtain contrasts of
averages between treatment assignment groups.

First, let us state the problem in terms of the `full data' we would
need were we to base these contrasts on observations. Where ATE denotes
the ``Average Treatment Effect'':

\[
\Delta_{ATE} = \left(\underbrace{\underbrace{\mathbb{E}[Y(1)|A = 1]}_{\text{observed for } A = 1} + \underbrace{\mathbb{E}[Y(1)|A = 0]}_{\text{unobserved for } A = 0}}_{\text{effect among treated}}\right) - \left(\underbrace{\underbrace{\mathbb{E}[Y(0)|A = 0]}_{\text{observed for } A = 0} + \underbrace{\mathbb{E}[Y(0)|A = 1]}_{\text{unobserved for } A = 1}}_{\text{effect among untreated}}\right)
\]

In each treatment condition, we do not observe the potential outcome for
every unit that did not receive the opposite level of treatment they in
fact received. However, when researchers randomise units into treatments
conditions, then the distribution of variables that might affect each
potential outcome independently of the exposure cancel each other out
when averaging over the groups. When treatments are randomly assigned,
and randomisation is effective, the outcomes under different treatment
conditions should, in theory, be identical on average.

\[
 \mathbb{E}[Y(0) | A = 1] = \mathbb{E}[Y(0) | A = 0] 
\]

\[
\mathbb{E}[Y(1) | A = 1] = \mathbb{E}[Y(1) | A = 0] 
\]

And thus,

\[
  \widehat{\Delta_{ATE}} = \mathbb{E}[Y | A = 1] - \mathbb{E}[Y | A = 0]
\]

Where \(\widehat{ATE}\) is an unbiased estimate for expected value of
the treatment effect on the difference scale (or equivalently, for the
differences of the averages of the treatment effects).

Although randomisation can fail, it provides a means to identify
group-level causal effects by a process of elimination. The distribution
of potential outcomes must the same across treatment groups because
randomisation in a well-designed and executed experiment exhausts every
other explanation except the treatment. For this reason, we should
prefer experiments for addressing causal questions that can be addressed
by them.

Regrettably, many scientific questions, particularly those in the
evolutionary human sciences, cannot be addressed through experimental
means. This limitation is acutely felt when researchers confront `what
if?' scenarios rooted in the unidirectional nature of human history.
However, understanding how radomisation obtains missing counterfactual
outcomes clarifies the tasks of causal data science in non-experimental
settings (\citeproc{ref-hernuxe1n2008a}{Hernán \emph{et al.} 2008};
\citeproc{ref-hernuxe1n2022}{Hernán \emph{et al.} 2022a};
\citeproc{ref-hernuxe1n2006}{Hernán and Robins 2006}). We must obtain
balance across observed variables that might account for treatment-level
differences (\citeproc{ref-greifer2023}{Greifer \emph{et al.} 2023}).
This task of obtaining balance presents a significant challenge
(\citeproc{ref-stuart2015}{Stuart \emph{et al.} 2015}). Observations
typically cannot in themselves verify no-unmeasured confounding.
Moreover, we must satisfy ourselves of additional assumptions, which
although nearly automatic in randomised experimental settings, impose
strong restrictions on causal effect estimation where the exposure are
not randomised. We next discuss subset of the these assumptions and
group them into two categories: (1) Fundamental identification
assumptions; (2) Practical assumptions. And we will locate the primary
functions of causal diagrams within a workflow that must explicitly
clarify a pathway for satisfying them.

\subsubsection{Fundamental Identification
Assumptions}\label{fundamental-identification-assumptions}

There are three fundamental identification assumptions that must be
satisfied to estimate causal effects with data.

\paragraph{Assumption 1: Causal
Consistency}\label{assumption-1-causal-consistency}

The causal consistency assumption posits that for any given level of
exposure, \(A_i=a\), the observed outcome, \(Y_i|A_i=a\), is
interchangeable with the counterfactual outcome: \(Y(a)|A = a\). To
illustrate, we use the subscript \(i\) to represent individual
\(i, 1 \dots n\). The observed outcome when treatment is \(A_i = a\) is
denoted as \(Y_i^{observed}|A_i = a\). When the causal consistency
assumption is satisfied, the observed outcome for each \(i, 1 \dots n\)
corresponds to one of the counterfactual outcomes necessary for
calculating a causal contrast:

\[
Y_i^{observed}|A_i = 
\begin{cases} 
Y_i(a^*) & \text{if } A_i = a^* \\
Y_i(a) & \text{if } A_i = a
\end{cases}
\]

The consistency assumption implies that the observed outcome at a
specific exposure level mirrors the counterfactual outcome for that
individual. Although it might seem straightforward to equate an
individual's observed outcome with their counterfactual outcome, in
observational studies, treatment effects often vary, posing considerable
challenges in satisfying this assumption.

Consider a question that has been discussed in the evolutionary human
science literatures question about whether a society's beliefs in Big
Gods affects its development of Social Complexity
(\citeproc{ref-beheim2021}{Beheim \emph{et al.} 2021};
\citeproc{ref-slingerland}{Slingerland \emph{et al.}};
\citeproc{ref-watts2015}{Watts and Gray 2015};
\citeproc{ref-whitehouse2023}{Whitehouse \emph{et al.} 2023}).
Historians and anthropologists report that such beliefs vary over time
and across cultures in their intensity, interpretations, institutional
management, and ritual embodiments (\citeproc{ref-decoulanges1903}{De
Coulanges 1903}; \citeproc{ref-geertz2013}{Geertz \emph{et al.} 2013};
\citeproc{ref-wheatley1971}{Wheatley 1971}). Knowing nothing else, we
might expect that such variation in content and settings in which such
beliefs are realised could influence social complexity. Moreover the
treatments as they are realised in one society might affect the
treatments realised in other societies. Often, it may be unclear how we
can address the treatment independence assumption using a conditioning
strategy. I review additional problems in Appendix 1. For now, the
manifest variation in treatments we should assume to hold in this
setting underscores the need for careful consideration whether
\emph{treatment heterogeneity} allows us to assume conditional
exchangeability.

The theory of causal inference under multiple versions of treatment,
developed by VanderWeele and Hernán, formally addresses the challenge of
treatment heterogeneity (\citeproc{ref-vanderweele2009}{VanderWeele
2009a}, \citeproc{ref-vanderweele2018}{2018};
\citeproc{ref-vanderweele2013}{VanderWeele and Hernan 2013}).
VanderWeele proved that if the treatment variations (\(K\)) are
conditionally independent of the outcome \(Y(k)\) given covariates
\(L\), then conditioning on \(L\) allows us to consistently estimate a
causal contrast over the heterogeneous treatments
(\citeproc{ref-vanderweele2009}{VanderWeele 2009a}).

Where \(\coprod\) denotes independence, causal consistency is formally
preserved if

\[
K \coprod Y(k) | L
\]

Under the theory of causal inference under multiple versions of
treatment, we think of \(K\) as a ``coarsened indicator'' for \(A\).
That is, we obtain an average effect estimate for the multiple treatment
versions \(K\) on \(Y(k)\).

Although the theory of causal inference under multiple versions of
treatment provides a formal solution to the problem of treatment effect
heterogeneity, interpreting the resulting causal effect estimates under
this theory can be challenging. Consider the question of whether a
reduction in Body Mass Index (BMI) affects health
(\citeproc{ref-hernuxe1n2008}{Hernán and Taubman 2008}). Notably, weight
loss can occur through various methods, each with different health
implications. Some methods, such as regular exercise or a
calorie-reduced diet, are generally beneficial for health. However,
weight loss might also result from adverse conditions such as infectious
diseases, cancers, depression, famine, or even amputations, which are
not beneficial to health. Although causal effects can be consistently
estimated while adjusting for covariates \(L\), the true nature and
implications of the changes in BMI might remain unclear. This
uncertainty highlights the need for precise and well-defined causal
questions. For example, rather than stating the intervention as ``weight
loss'', we state the intervention as weight loss achieved specifically
through aerobic exercise over a period of at least five years, compared
with no weight loss. This level of specificity in our exposure and
outcomes helps ensure that the causal estimates we obtain are not only
statistically sound but also interpretable (for discussion see:
(\citeproc{ref-hernuxe1n2022a}{Hernán \emph{et al.} 2022b};
\citeproc{ref-hernuxe1n2008}{Hernán and Taubman 2008};
\citeproc{ref-murray2021a}{Murray \emph{et al.} 2021}).

Beyond interpretation, there is the additional problem that we cannot
really know whether the measured covariates \(L\) suffice to render the
multiple-versions of treatment independent of the counterfactual
outcomes. This problem is especially acute when there are spill-over
effects, such when treatment-effects are relative to the density and
distribution of treatment-effects in a population. Scope for such
interference will may often make it difficult satisfy ourselves that the
potential outcomes are truly independent of many versions of treatment
that have been realised, dependently, on the administration of previous
versions of treatments across the population
(\citeproc{ref-bulbulia2023a}{Bulbulia \emph{et al.} 2023};
\citeproc{ref-ogburn2022}{Ogburn \emph{et al.} 2022};
\citeproc{ref-vanderweele2013}{VanderWeele and Hernan 2013}).

Thus, what seemed initially to be a near truism -- that each units
observed outcome may be assumed to correspond to that unit's
counterfactual outcome -- turns out to be a strong assumption. In many
settings, causal consistency should be presumed unrealistic until proven
tenable. For now, we note that the causal consistency assumption
provides a theoretical starting point for recovering the missing
counterfactuals essential for computing causal contrasts. It achieves
this by identifying half of these counterfactuals directly from observed
data. The concept of exchangeability, which we will explore next, offers
a means to derive the remaining half of the missing counterfactuals we
require to consistently compute causal contrasts.

\paragraph{Assumption 2: Conditional Exchangeability (No Unmeasured
Confounding)}\label{assumption-2-conditional-exchangeability-no-unmeasured-confounding}

We satisfy conditional exchangeability when the treatment groups are
equivalent in variables that could affect potential outcomes. In
experimental designs, random assignment facilitates conditional
exchangeability. In observational studies, more effort is required. We
must control for covariates that could account for observed correlations
between \(A\) and \(Y\) in the absence of a causal effect of \(A\) on
\(Y\) for every observed.

Let \(L\) denote the set of covariates necessary to ensure this
conditional independence. Let \(\coprod\) again denote independence. We
satisfy conditional exchangeability when:

\[
Y(a) \coprod A | L \quad \text{or equivalently} \quad A \coprod Y(a) | L
\]

Where this, and the other fundamental assumptions hold, we may compute
the average treatment effect (ATE) on the difference scale:

\[
ATE = \mathbb{E}[Y(1) | L] - \mathbb{E}[Y(0) | L]
\]

In the disciplines of cultural evolution, where experimental control is
often impractical, causal inferences hinge on the plausibility of
satisfying this `no unmeasured confounding' assumption. Lacking
randomisation, however, causal data science must turn to sensitivity
analyses (Appendix 1 critiques such assumptions for estimating causal
effects of beliefs in Big Gods on social complexity).

Importantly, \emph{the primary purpose of a causal diagram within a
causal inference workflow is to evaluate the conditional exchangeability
assumption.} Causal diagrams enable researchers to represent crucial
structural assumptions that are necessary for achieving balance in the
confounders across levels of the exposure stated in a pre-specified
causal contrast.

Moreover, it is important to recognise that in this setting, causal
diagrams are designed to \emph{highlight only those aspects the assumed
causal order pertinent to the assessment of `no-unmeasured
confounding.'} A common mistake when creating a causal diagram is to
provide too much detail, obscuring rather than clarifying structural
sources of bias. We return to this point below.

Finally, we must realise that it will not typically be the case that we
can satisfy ourselves of the ``no-unmeasured'' confounding. For this
reason, causal data science must rely heavily on sensitivity analyses to
clarify how much unmeasured confounding would be required to compromise
a study's findings (VanderWeele (\citeproc{ref-vanderweele2019}{2019})).

\paragraph{Assumption 3: Positivity}\label{assumption-3-positivity}

We may say the positivity assumption is met when there exists a non-zero
probability for each level of exposure within every level of covariates
needed to ensure conditional exchangeability. This implies that for each
stratification of every covariate, the probability that for each level
of exposure value must exceed zero. Where \(A\) is the exposure and
\(L\) a vector of covariates, positivity is only achieved if

\[
0 < \Pr(A = a | L = l) < 1, ~ \forall a \in A, ~ \forall l \in L
\]

There are two types of positivity violation:

\begin{enumerate}
\def\labelenumi{\arabic{enumi}.}
\item
  \textbf{Random non-positivity:} which occurs when an exposure is
  theoretically possible, but specific exposure levels are not
  represented in the data. Notably, random non-positivity is the only
  identifiability assumption verifiable with data.
\item
  \textbf{Deterministic non-positivity:} which occurs when the exposure
  is implausible by nature. For instance, a hysterectomy in biological
  males would appear biologically implausible.
\end{enumerate}

Satisfying the positivity assumption may present considerable data
challenges (\citeproc{ref-westreich2010}{Westreich and Cole 2010}).
Consider estimating the effects of church attendance on charity. Suppose
our objective is to assess the one-year effect on charitable donations
following a shift from no church attendance to weekly attendance. Assume
we have access to extensive panel data that has tracked 20,000
individuals over three years. Consider, the natural transition rate from
no attendance to weekly attendance in many human populations will be
quite low. Suppose it is one in a thousand annually. In that case, the
effective sample for the treatment condition dwindles to 20. Despite
abundant data, attention to the positivity assumptions reveals the data
required for valid contrasts is sparse.

Note, where positivity is violated, causal diagrams are of limited
utility to causal inference because causal inference is not supported by
the data.

\subsubsection{Practical Assumptions and
Considerations}\label{practical-assumptions-and-considerations}

Beyond the three fundamental identification assumptions that must be
satisfied to estimate causal effects with data, there are numerous
practical consideration that enter into every causal data science
workflow.

\paragraph{1. Overly ambitious
estimands}\label{overly-ambitious-estimands}

In causal inference, the notion of assessing Average Treatment Effects
(ATE) through \(E[Y(1) - Y(0)|L]\) is often flawed. This is evident in
the context of continuous exposures, where our estimand must simplifies
the complexity of real-world phenomena into a low dimensional summary,
such as a contrast of a one-standard deviation difference in the mean,
or a comparison of a one to another quartile. The requirement that we
target a contrast leads to a stronger reliance on models, which
introduce further opportunities for bias (see below). It may also
stretch the positivity assumption to its breaking point. The real-world
simplifications required for standard causal estimands does not operate
in such neatly defined exposure levels, rendering these comparisons
artificial and potentially misleading
(\citeproc{ref-vansteelandt2022}{Vansteelandt and Dukes 2022}).

Moreover, the assumption of a monotonic relationship between treatment
and effect is equally naive (\citeproc{ref-calonico2022}{Calonico
\emph{et al.} 2022}; \citeproc{ref-ogburn2021}{Ogburn and Shpitser
2021}). Real-life treatment effects are rarely linear or
straightforward. By comparing arbitrary points on a continuous scale, we
risk drawing erroneous conclusions about the treatment's overall impact.
This approach is a shortcut that distorts the true nature of the
treatments effect.

Focusing on ATE also masks the vital heterogeneity in treatment effects
(\citeproc{ref-wager2018}{Wager and Athey 2018}). This heterogeneity is
not just a statistical nuisance; it is the essence of understanding
causal mechanisms. Average effects blur the variations and nuances that
are critical for informed decision-making. Recognising and elucidating
this heterogeneity should be a primary goal, yet current methods fall
short. The field is evolving (see: Tchetgen and VanderWeele
(\citeproc{ref-tchetgen2012}{2012}); Wager and Athey
(\citeproc{ref-wager2018}{2018}); Cui \emph{et al.}
(\citeproc{ref-cui2020}{2020}), Foster and Syrgkanis
(\citeproc{ref-foster2023}{2023});(\citeproc{ref-foster2023}{Foster and
Syrgkanis 2023}; \citeproc{ref-kennedy2023}{Kennedy 2023};
\citeproc{ref-nie2021}{Nie and Wager 2021}) ). These emerging
methodologies, while promising, are still in their infancy and are not
yet full demonstrated their capability of addressing the complexities of
heterogeneous treatment effects estimation. Such limitations must be
clarified at the outset of inquiry.

Readers should be aware that although I have presented standard causal
estimands to develop intuition and understanding that causal inference
requires consistently, contrasting specific counterfactual outcomes for
the entire population simulated at each level of the exposure to be
contrasted. However, other estimands such as modified treatment policies
(shift interventions)(\citeproc{ref-duxedaz2021}{Díaz \emph{et al.}
2021}; \citeproc{ref-hoffman2023}{Hoffman \emph{et al.} 2023};
\citeproc{ref-vanderweele2018}{VanderWeele 2018};
\citeproc{ref-williams2021}{Williams and Díaz 2021}) and optimal
treatment policies (\citeproc{ref-athey2021}{Athey and Wager 2021};
\citeproc{ref-kitagawa2018}{Kitagawa and Tetenov 2018}) are presently
active areas of interest and development. These allow researchers to
specify and obtain a broader range of causal contrasts, such population
contrast from (pseudo)-random treatments differently to specific
segments of the population. I cannot detail these developments here, but
it is important to recognise that careful thinking informed by expert
knowledge is required at the very point of specifying the hypothetical
causal intervention to be assessed. Again, if we merely stating
assumptions in a graph and proceed to modelling data there is no
guarantee that what we obtain will be interpretable or valid.

\paragraph{2. Measurement error bias}\label{measurement-error-bias}

Measurement error refers to the discrepancy between the true value of a
variable and the value that is observed or recorded. This error can
arise from a variety of sources, including instrument calibration,
respondent misreporting, or coding errors. Unfortunately, measurement
error is both commonplace, and capable of significantly distorting
causal inferences.

Briefly, measurement error confounding can be classified into two main
types: systematic (directed) and uncorrelated (undirected).

\textbf{Systematic measurement error:} occurs when the measurements
deviate from the true value in a consistent direction. Systematic errors
can lead to biased estimates of causal effects, as they consistently
overestimate or underestimate true causal magnitudes.

\textbf{Random measurement error:} occurs when fluctuations in the
measurement process and do not consistently bias the data in one
direction. While random measurement errors can increase the variability
of data and reduce statistical power, they do not typically introduce
bias in estimates of causal effects when there are no such effects.
However, random measurement error can attenuate bias when there are true
causal effects, where the estimated effect of an exposure on an outcome
is systematically weakened.

To handle measurement error the best approach is to improving data
quality; this is not always possible, and so we must perform sensitivity
analyses.

Causal diagrams may be powerful aids for evaluating structural sources
of bias that may arise from different forms of measurement error
(\citeproc{ref-hernan2023}{Hernan and Robins 2023};
\citeproc{ref-hernuxe1n2009}{Hernán and Cole 2009};
\citeproc{ref-vanderweele2012a}{VanderWeele and Hernán 2012}). We will
not consider this use here {[}CITE OTHER PAPER{]}. For now, it is
important to emphasise that the simple causal diagrams with arrows
between variables typically abstract away from biases that arise from
measurement error, and such simplicity can be a source of false
confidence.\footnote{The careful reader will note a tension. Addressing
  structural sources of bias requires simple causal diagrams. Such
  diagrams do not capture the threats to inference arising from
  measurement, which requires more complicated causal diagrams. Hernán
  and Robins advise a two step approach in which authors draft separate
  diagrams to handle separate threats to valid causal inference
  (\citeproc{ref-hernuxe1n2023}{Hernán and Monge 2023}).}

\paragraph{3. Selection bias}\label{selection-bias}

Selection bias arises when the sample that is observed is not
representative of the population for which causal inference is intended.
There are two primary forms: bias arising to initial selection and bias
resulting from attrition or non-response
(\citeproc{ref-bareinboim2022a}{Bareinboim \emph{et al.} 2022};
\citeproc{ref-hernuxe1n2004a}{Hernán \emph{et al.} 2004};
\citeproc{ref-hernuxe1n2017}{Hernán 2017}; \citeproc{ref-lu2022}{Lu
\emph{et al.} 2022}; \citeproc{ref-suzuki2020a}{Suzuki \emph{et al.}
2020b}).\footnote{It is important to note that the term ``selection on
  observables'' is used differently in economics compared to
  epidemiology and public health, as I am using the term here. In
  economics, ``selection on observables'' typically refers to the
  non-random treatment assignments in observational data. In this
  context, if all variables that influence both the selection process
  and the outcome are observed and accounted for, the resulting bias can
  be controlled for in the analysis. Terminological differences abound
  in casual data science. It is crucial to be aware of these
  distinctions and clarify the specific usage within the context of
  one's study to avoid confusion and ensure accurate communication.}

\textbf{Selection prior to observation:} the bias when the process of
selecting individuals or units for the study leads to a
non-representative sample of the target population. This bias may arise
from specific inclusion or exclusion criteria or through non-random
selection methods. Such bias can introduce systematic differences
between treatment and control groups, affecting the generalizability of
results. Consequently, the estimates obtained might not accurately
reflect the causal effects in the intended population.

\textbf{Attrition/non-response bias:} this bias occurs after selection
when participants or units drop out of the study, and their dropout
opens a path to the treatment and the outcome, or otherwise compromises
generalisation. Non-response bias occurs when certain subjects do not
respond to surveys or follow-ups, and this non-response is correlated
with the treatment and outcome. Such bias can skew results, as the final
sample may differs from the initial sample in aspects crucial to the
study's focus. Unlike common-or-garden confounding bias, selection bias
cannot be addressed by simply conditioning on a set of baseline
covariates \(L\).

Causal diagrams can be useful in diagnosing sources of selection bias
and describing its features (\citeproc{ref-hernuxe1n2017}{Hernán 2017}).
Below, we limit the application of causal diagrams for understanding the
primary structural sources of confounding bias. However, it is
imperative to recognise that, similar to measurement error bias,
selection bias can substantially distort causal inferences.

\paragraph{4. Model misspecification
bias}\label{model-misspecification-bias}

After meeting the essential and practical assumptions necessary for
valid causal inference, the next step involves deriving an estimate of
our pre-defined causal contrasts from the data. In statistical analysis,
human scientists predominantly use parametric models, which are defined
by pre-set functional forms and distributional assumptions. However,
this reliance on parametric models introduces the risk of biased
inferences due to model misspecification. The adverse impacts of model
misspecification manifest in several important ways.

\begin{enumerate}
\def\labelenumi{\alph{enumi}.}
\item
  \textbf{Regularisation bias:} parametric models will biased estimates
  of causal effects when the true inter-variable relationships are more
  complex or divergent than those assumed in the model -- and we should
  presume that parametric models are misspecified Vansteelandt and Dukes
  (\citeproc{ref-vansteelandt2022}{2022}).
\item
  \textbf{Overstated precision:} a misaligned model may falsely suggest
  a level of precision by mis-estimating standard errors, leading to
  unwarranted confidence Vansteelandt and Dukes
  (\citeproc{ref-vansteelandt2022}{2022}).
\item
  \textbf{Standard statistical tests do not establish causation:}
  because statistical models are not structural models
  (\citeproc{ref-vanderweele2022a}{VanderWeele 2022a}), even when a
  model seemingly fits the data well, it may fail to accurately capture
  causation (\citeproc{ref-mcelreath2020}{McElreath 2020}). This
  highlights the limitations of relying solely on goodness-of-fit
  metrics and underscores the need for more comprehensive evaluations
  (\citeproc{ref-vansteelandt2022}{Vansteelandt and Dukes 2022}).
\item
  \textbf{Uncertainty about model convergence:} when a model is
  misspecified, it becomes unclear what the model is converging towards.
  Again, this uncertainty raises concerns about the validity of the
  model's estimates, as we cannot be sure if the model is capturing the
  intended causal relationships or some spurious pattern in the data.
\end{enumerate}

Recent developments in of non-parametric and doubly robust estimation
that rely on machine learning to model both the exposure and outcome,
offer some promise for addressing threats to valid inference
(\citeproc{ref-athey2019}{Athey \emph{et al.} 2019};
\citeproc{ref-duxedaz2021}{Díaz \emph{et al.} 2021};
\citeproc{ref-hahn2020}{Hahn \emph{et al.} 2020};
\citeproc{ref-kuxfcnzel2019}{Künzel \emph{et al.} 2019};
\citeproc{ref-vanderlaan2011}{Van Der Laan and Rose 2011},
\citeproc{ref-vanderlaan2018}{2018}; \citeproc{ref-wager2018}{Wager and
Athey 2018}; \citeproc{ref-williams2021}{Williams and Díaz 2021}). These
methods can provide valid estimates even if only one of the models is
correctly specified. Yet, sensitivity analyses are vital for confirming
the robustness of inferences under varying model assumptions. Despite
these efforts to ensure robustness, the risk of invalid conclusions
persists, as extensively discussed in the literature
(\citeproc{ref-cui2020}{Cui \emph{et al.} 2020};
\citeproc{ref-duxedaz2021}{Díaz \emph{et al.} 2021};
\citeproc{ref-hoffman2022}{Hoffman \emph{et al.} 2022};
\citeproc{ref-muuxf1oz2012}{Muñoz and Laan 2012};
\citeproc{ref-vansteelandt2022}{Vansteelandt and Dukes 2022};
\citeproc{ref-wager2018}{Wager and Athey 2018};
\citeproc{ref-williams2021}{Williams and Díaz 2021}).

In short, causal diagrams are ``model free'' qualitative tools for
assessing structural sources of bias. They cannot address the bias of
model misspecification. Addressing the bias of model misspecification is
an active area of current research, and remains a considerable threat to
valid causal inference. It is important to keep these, and other,
threats to valid inference in mind before racing from a causal diagram
to the analysis of data.

\subsubsection{Summary of Part 1}\label{summary-of-part-1}

Causal data science is not ordinary data science. It begins with a
requirement to precisely state a causal question with reference to a
well-specified exposure and outcome, and a specific population of
interest. Classical estimands involve quantifying the effect of an
intervention (exposure or treatment) \(A\), expressed as a contrast
between potential outcomes on some scale (such as the difference scale
\(Y(1) - Y(0)\)). The central challenge arises from the inherent
limitation of observing at most only one of the potential outcomes
required to compute this contrast for each unit that is observed.

A solution to this challenge is implicit in randomised experimental
design. Randomisation allows us balance confounders in the treatment
conditions, leaving only the treatment as the best explanation for any
observed differences in the treatment averages.

We considered the three fundamental assumptions required for causal
inference and that implicit in ideally conducted randomised experiments:
causal consistency (ensuring outcomes at a specific exposure level align
with their counterfactual counterparts), conditional exchangeability
(absence of unmeasured confounding), and positivity (existence of a
non-zero probability for each exposure level across all covariate
stratifications). Fulfilling all of these assumptions is crucial for
valid causal inference. We noted that causal diagrams primarily assist
reachers in assessing the assumption of no unmeasured confounding.

Furthermore, we examined a set of practical considerations that might
undermine confidence in causal inferences, and that must also be made
explicit, such as the need for interpretable causal etimands,
inferential threats from measurement error and selection bias (problems
that overlap each other and with problems of counding bias), and
regularisation bias from model misspecification. Causal diagrams are
particularly helpful in addressing measurement error biases and
selection error biases, given their structural characteristics. However,
model misspecification can profoundly alter both the precision and
relevance of our causal conclusions. And before any of these problems
creep in, whether we have asked a sensible causal question requires
careful though, and expert Thus, it is imperative to scrutinise
measurement errors, ensure sample representativeness, manage attrition
and loss-to-follow-up effectively, utilise robust statistical models and
sensitivity analyses, among other considerations, depending on the
specific scientific context and objectives. It is evident that causal
data science imposes unique and strong demands, distinct from those of
traditional data science. Although both ultimately rely on statistical
models, causal data science requires an intricate, multi-step workflow
to consistently recover estimands and uncertainties for contrasts in the
means of unobserved counterfactuals. This process extends beyond simply
creating causal diagrams and analysing patterns in observed data. This
is not the Kansas of ordinary data science.

Having outlined the crucial aspects of the causal inferential workflow,
we are now positioned to demonstrate the application of causal diagrams
in elucidating common sources of confounding bias.

\subsection{Part 2. Applications of Chronologically Ordered Causal
Diagrams for Understanding Confounding
Bias}\label{part-2.-applications-of-chronologically-ordered-causal-diagrams-for-understanding-confounding-bias}

This section focuses on the application of chronologically ordered
causal diagrams for isolating confounding bias in causal inference,
grounded in seminal works by Pearl, Greenland, and others
(\citeproc{ref-greenland1999}{Greenland \emph{et al.} 1999a};
\citeproc{ref-pearl2009}{Pearl 2009a}; \citeproc{ref-pearl1995}{Pearl
and Robins 1995}). Causal diagrams are powerful tools for discerning
conditions under which data can yield reliable causal effect estimates.
Organizing these diagrams chronologically mitigates the risk of
inferential errors, enhances the visibility of bias sources, and
provides strategic direction for data collection (See Appendix 2 for a
more detailed explanation).

\subsubsection{Definitions}\label{definitions}

\textbf{Nodes}: in a casual diagram, each node represents a variable,
which can be observed, latent, or composite. These nodes stand for
distinct variables within the causal framework.

\textbf{Arrows}: arrows symbolise assumed pathways of cause and effect.
An arrow from \(A\) to \(Y\) (denoted as \(A \rightarrow Y\)) indicates
that \(A\) causally influences \(Y\).

\textbf{Ancestors (parents)}: are nodes with a direct or indirect
influence on others, positioned upstream in the causal chain.

\textbf{Descendants (children)}: are nodes influenced, directly or
indirectly, by other nodes, located downstream in the causal chain.

\textbf{D-separation}: a concept to understand whether two nodes are
independent given another variable or set of variables. If all paths
between two nodes are `blocked', they are independent in this sense
(\citeproc{ref-pearl2009}{Pearl 2009a}).

\textbf{Conditioning}: the process of explicitly accounting for a
variable in the analysis.In causal diagrams, we often represent
conditioning by drawing a box around a node of the conditioned variable.
We do no typically box exposures and outcomes; these are assumed to be
modelled. Depending on the setting, we may condition by regression
stratification, propensity score weighting, g-methods, or doubly robust
machine learning algorithms. In causal diagrams, we often represent
conditioning by drawing a box around a node of the conditioned variable.
We do no typically box exposures and outcomes; these are assumed to be
modelled.

\textbf{Markov Factorisation} mathematically states that the joint
probability distribution of a set of variables can be decomposed into a
product of conditional distributions. Each of these conditional
distributions depends solely on the immediate parent variables of a
given node in the causal diagram.\footnote{Formally, if
  \(X_1, X_2, \ldots, X_n\) are the nodes in a causal diagram, the joint
  distribution \(P(X_1, X_2, \ldots, X_n)\) can be expressed as: \[
  P(X_1, X_2, \ldots, X_n) = \prod_{i=1}^{n} P(X_i | \text{Parents}(X_i))
  \] Here, \(\text{Parents}(X_i)\) denotes the set of direct
  predecessors of \$ X\_i\$ in the causal graph. Markov Factorisation
  enables the simplification of complex joint distributions into more
  tractable components. This decomposition is particularly valuable in
  identifying and addressing structural sources of bias in causal
  relationships, as it aligns with the graphical structure of the causal
  model.Formally, if \(X_1, X_2, \ldots, X_n\) are the nodes in a causal
  diagram, the joint distribution \(P(X_1, X_2, \ldots, X_n)\) can be
  expressed as: \[
  P(X_1, X_2, \ldots, X_n) = \prod_{i=1}^{n} P(X_i | \text{Parents}(X_i))
  \] Here, \(\text{Parents}(X_i)\) denotes the set of direct
  predecessors of \$ X\_i\$ in the causal graph. Markov Factorisation
  enables the simplification of complex joint distributions into more
  tractable components. This decomposition is particularly valuable in
  identifying and addressing structural sources of bias in causal
  relationships, as it aligns with the graphical structure of the causal
  model.}

\textbf{Adjustment set}: a collection of variables that we must either
condition upon or deliberately avoid conditioning to avoid distorting,
or obscuring the causal association between exposure and outcome as
presented in a causal diagram (\citeproc{ref-pearl2009}{Pearl 2009a}).

\textbf{Confounder}: a member of an adjustment set. Notice, a variable
is a ``confounder'' in relation to a specific adjustment set, it is a
relative concept (\citeproc{ref-lash2020}{Lash \emph{et al.} 2020}).

\textbf{Modified Disjunctive Cause Criterion}: I adopt a \emph{Modified
Disjunctive Cause Criterion} for controlling for confounding
(\citeproc{ref-vanderweele2019}{VanderWeele 2019}). According to this
criterion, a member of any set of variables that can reduce or remove
the bias caused by confounding is deemed a member of this confounder
set. The strategy is as follows:

\begin{enumerate}
\def\labelenumi{\alph{enumi}.}
\tightlist
\item
  Control for any variable that causes the exposure, the outcome, or
  both.
\item
  Control for any proxy for an unmeasured variable that is a shared
  cause of both exposure and outcome.
\item
  Define an instrumental variable as a variable associated with the
  exposure but does not influence the outcome independently, except
  through the exposure. Exclude any instrumental variable that is not a
  proxy for an unmeasured confounder from the confounder set.\footnote{Note
    that the concept of a ``confounder set'' is broader than the concept
    of an ``adjustment set.'' Every adjustment set is a member of a
    confounder set. So the Modified Disjunctive Cause Criterion will
    eliminate confounding when the data permit. However a confounder set
    includes variables that will reduce confounding in cases where
    confounding cannot be eliminated. Confounding can almost never be
    eliminated with certainty. For this reason we must perform
    sensitivity analyses to check the robustness of our results.}
\end{enumerate}

\textbf{Instrumental variable}: an ancestor of the exposure but not of
the outcome. That is a variable that affects the outcome only through
its effect on the exposure and not otherwise. Whereas conditioning on a
variable that is causally associated with the outcome but not with the
exposure will generally increase modelling precision, we should
generally avoid conditioning on instrumental variables. The except are
when we are interested in instrumental variable analysis (see XXY this
issue) or when the instrumental variable is the descendent of an
unmeasured confounder (described below). By contrast, we typically gain
efficiency by conditioning on a variable that is associated with the
outcome, even if it is not associate with the exposure
(\citeproc{ref-cinelli2022}{Cinelli \emph{et al.} 2022}).

\textbf{Modified Disjunctive Cause Criterion}: VanderWeele's Modified
Disjunctive Cause Criterion provides practical guidance for controlling
for confounding (\citeproc{ref-vanderweele2019}{VanderWeele 2019}).
According to this criterion, a member of any set of variables that can
reduce or remove the bias caused by confounding is deemed a member of
this confounder set. VanderWeele's strategy for defining a confounder
set is as follows:

\subsubsection{The rules of
D-separation}\label{the-rules-of-d-separation}

The rules for obtaining conditional and unconditional dependencies
between nodes in a causal diagram are as follows:

\begin{enumerate}
\def\labelenumi{\arabic{enumi}.}
\item
  \textbf{Fork Rule} (\(A \leftarrow L \rightarrow Y\)): This represents
  a common cause structure. \(A\) and \(Y\) are independent by default
  in this structure, and conditioning on \(L\) maintains their
  independence. Mathematically, this is expressed as \$ A \coprod Y
  \textbar{} L \$, signifying that conditioning on \(L\) keeps \(A\) and
  \(Y\) independent.
\item
  \textbf{Chain Rule} (\(A \rightarrow L \rightarrow Y\)): In this chain
  structure, conditioning on \(L\) blocks the path between \(A\) and
  \(Y\). This can be expressed as \$ A \coprod Y \textbar{} L \$,
  indicating that \(A\) and \(Y\) are conditionally independent given
  \(L\).
\item
  \textbf{Collider Rule} (\(A \rightarrow L \leftarrow Y\)): Initially,
  \(A\) and \(Y\) are independent in this structure, as the path is
  blocked at the collider \(L\). This can be denoted as \(A \coprod Y\).
  However, conditioning on \(L\) opens the path and introduces
  dependence between \(A\) and \(Y\). This change in dependence that
  does not arise from conditioning is represented as
  \(A \cancel{\coprod} Y | L\), indicating that \(A\) and \(Y\) become
  dependent when conditioning on \(L\) or \(L'\).
\end{enumerate}

\subsubsection{Local graphical
conventions}\label{local-graphical-conventions}

It is vital that researchers describe any unique graphical conventions.
Below I adopt the following:

\textbf{Red arrows}: denote open paths between exposure and outcome due
to suboptimal conditioning strategies.

\textbf{Blue arrows}: denote open paths between the exposure and outcome
that led to bias by closing a path that should remain open (as with
mediator bias).

\textbf{Dashed red arrows}: denote paths where confounding bias has been
mitigated. Complete elimination of confounding is often unfeasible,
especially in observational studies. Hence, the focus is on bias
reduction, supplemented by sensitivity analyses to test the resilience
of findings against unmeasured confounders. It is important to note that
software tools like \texttt{Dagitty} and \texttt{ggdag}, though
beneficial, may overlook optimal strategies involving open paths
(\citeproc{ref-barrett2021}{Barrett 2021};
\citeproc{ref-textor2011}{Textor \emph{et al.} 2011}). Therefore,
reliance on these tools should be balanced with independent diagram
interpretation skills.

\textbf{Departure from conventions when describing causal mediation}:
When using causal diagrams in Part III to clarify confounding when the
interest is causal mediation, we depart from the colouring conventions
because the conditions in which there is biasing for the mediator differ
the conditions in which there is biasing for the, and we use a single
graph to represent both forms of biasing.

\textbf{Variable naming conventions}: in the context of this discussion,
I use the following notation:

\begin{itemize}
\tightlist
\item
  \(L\): denotes variables that may potentially lead to denotes bias.
\item
  \(A\): represents the treatment or intervention of interest studied.
\item
  \(Y\): denotes the outcome of interest.
\item
  \(U\): denotes an unmeasured confounder.
\item
  \(L'\); denotes the measured descendant of either a measured
  confounder \(L\) or an unmeasured confounder \(U\).
\end{itemize}

\subsubsection{Advice for drawing a chronologically ordered
graph}\label{advice-for-drawing-a-chronologically-ordered-graph}

A causal diagram is intended to succinctly depict structural sources of
bias, rather than to statistically represent data. This distinction is
fundamental because the structure suggested by a causal diagram is often
not verifiable by data, making it `structural' in nature, as distinct
from the graphs used in structural equation modelling
(\citeproc{ref-bulbulia2021}{Bulbulia \emph{et al.} 2021};
\citeproc{ref-greenland1999c}{Greenland \emph{et al.} 1999b};
\citeproc{ref-hernuxe1n2023}{Hernán and Monge 2023};
\citeproc{ref-pearl2009a}{Pearl 2009b}). Misunderstanding this
difference between structural and statistical models has led to
considerable confusion across the human sciences
(\citeproc{ref-vanderweele2015}{VanderWeele 2015};
\citeproc{ref-vanderweele2022}{VanderWeele 2022b};
\citeproc{ref-vanderweele2022b}{VanderWeele and Vansteelandt 2022}).

Although a chronologically ordered causal diagram is mathematically
identical to one that lacks such order, the following examples
reexpressing ``chronological hygiene'' in diagrams layout can
considerably enhance the understanding of causal relationships. A
chronologically hygienic graph aligns the arrangement of nodes and
arrows to reflect the assumed temporal sequence of events. The
conventions I adopt for maintaining chronological hygiene are:

\textbf{Clearly define all nodes on the graph}: Ambiguity leads to
confusion.

\textbf{Simplify the graph by combining nodes where this is possible.}
Keep only those nodes and edges that are essential for clarifying the
identification problem at hand avoids unnecessary clutter and improves
readability.

\textbf{Maintain chronological order spatially:} generally arrange nodes
in \emph{relative} temporal sequence, usually from left to right or top
to bottom. Although it is not necessary to draw the sequence to scale,
the order of events should be clear from the layout. This provides an
intuitive visual representation of how one event is assumed to precede
another in time.

\textbf{Time-index all nodes}: nodes are indexed according to their
occurrence or measurement in time. This explicit indexing helps in
demarcating the temporal relationship between variables, adding
precision to the diagram, with the organisation:

\[L_{t0} \to A_{t1} \to Y_{t2}\]

This arrangement clearly illustrates the temporal sequence of these
variables, setting the stage for effectively applying chronologically
ordered diagrams in confounding control.

\textbf{Define any novel convention in your diagram explicitly}: do not
assume familiarity.

\textbf{Ensure acyclicity in the graph}: this guarantees that a node
cannot be its own ancestor, thereby eliminating circular paths.

\textbf{Draw nodes for unmeasured confounders}: assume unmeasured
confounding always exists, whether depicted on the graph or not. This
assumption reveals the importance of sensitivity analyses when
estimating causal effects.

\textbf{Illustrate nodes for post-treatment selection.} This facilitates
understanding of potential sources of selection bias.

\textbf{Apply a two-step strategy}: initially, isolate confounding bias
and selection bias, then contemplate measurement bias using a secondary
graph. This approach will foster clarity.\footnote{See Hernán and Monge
  (\citeproc{ref-hernuxe1n2023}{2023}) p.125}

\textbf{Do not attempt to draw non-linear associations between
variables}: causal diagrams are qualitative tools encoding assumptions
about causal relationships. They are compasses, not comprehensive
atlases.

\subsubsection{The four elemental confounding
conditions}\label{the-four-elemental-confounding-conditions}

Having described key terminology, conventions, and rules, it is time to
put causal diagrams to action. I begin by reviewing what McElreath
(\citeproc{ref-mcelreath2020}{2020}) p.185 calls the ``four fundamental
confounders.'' Because we have distinguished betwen confounders and
confounding, we will call these settings as the four elemental
confounding conditions.

\subsubsection{1. The elemental confounding of an unadjusted common
cause}\label{the-elemental-confounding-of-an-unadjusted-common-cause}

The first elemental confounding condition arises when there is a common
cause \(L\) of the exposure \(A\) and outcome \(Y\). In this setting,
\(L\) may create a statistical association between \(A\) and \(Y\),
giving an illusion of causation in its absence.

Consider an example where smoking (\(L\)) is a common cause of both
yellow fingers (\(A\)) and cancer (\(Y\)). Here, \(A\) and \(Y\) may
show an association in the absence of causation. If we were to intervene
to scrub the hands of smokers this would not affect their cancer rates.
The elemental confounding condition is represented in
Figure~\ref{fig-dag-common-cause}, where the red arrow signifies the
bias from the open path connecting \(A\) and \(Y\), caused by their
common cause \(L\).

\begin{figure}

{\centering \includegraphics[width=0.8\textwidth,height=\textheight]{causal-dags_files/figure-pdf/fig-dag-common-cause-1.pdf}

}

\caption{\label{fig-dag-common-cause}Counfounding by a common cause. The
red path indicates bias arising from the open backdoor path from A to
Y.}

\end{figure}

\subsubsection{\texorpdfstring{Advice: condition on
\(L\).}{Advice: condition on L.}}\label{advice-condition-on-l.}

To address confounding by a common cause we should adjust for it,
effectively blocking the backdoor path from the exposure to the outcome,
or put differently, restoring balance across the levels of \(A\) to be
compared in the distribution of counfounders that might affect the
potential outcomes under different levels of \(Y(a)\). Again, standard
methods for this adjustment include regression, matching, inverse
probability of treatment weighting, classical G-methods
(\citeproc{ref-hernuxe1n2023}{Hernán and Monge 2023}), and more recent
targeted learning frameworks (\citeproc{ref-hoffman2023}{Hoffman
\emph{et al.} 2023}). In Figure~\ref{fig-dag-common-cause-solution}, we
draw the causal diagram chronologically from left to right, which
highlight the imperative of ensuring that the common cause of \(A\) and
\(Y\), should be measured before \(A\) has occurred, and \(A\) should be
measured before \(Y\) has occurred. That is, to manage the problem of
confounding by a common cause, it is crucial to maintain the correct
temporal order in the measurements of the variables:

\begin{enumerate}
\def\labelenumi{\arabic{enumi}.}
\tightlist
\item
  Measure all confounders \(L\) that suffice to control for the
  confounding of \(A\) and \(Y\).
\item
  Unless \(L\) is a surrogate, ensure that \(L_{t0}\) is measured before
  \(A_{t1}\) occurs.
\item
  Ensure that \(A_{t1}\) is measured before \(Y_{t2}\) occurs.
\end{enumerate}

After we have time-indexing the nodes on the graph, it becomes evident
that \textbf{control of confounding generally requires time-series data
repeatedly measured on the units for which causal inferences apply.} Our
chronologically ordered causal diagram serves as a warning for causal
inferences in settings where researchers lack accurately well-recorded
time series data. Estimates in such settings require strong and perhaps
untenable assumptions.

\begin{figure}

{\centering \includegraphics[width=0.8\textwidth,height=\textheight]{causal-dags_files/figure-pdf/fig-dag-common-cause-solution-1.pdf}

}

\caption{\label{fig-dag-common-cause-solution}Solution: adjust for
pre-exposure confounder. The implication: obtain time series data that
ensure confounders occur before the exposure.}

\end{figure}

\subsubsection{2. The elemental confounding from conditioning on a
mediator}\label{the-elemental-confounding-from-conditioning-on-a-mediator}

If we condition on \(L\) and it forms part of the causal pathway linking
the treatment and the outcome, then we may distorts the total effect of
the treatment on the outcome. This is called \emph{mediator bias}.

Take ``beliefs in Big Gods'' to be the treatment \(A_{t0}\), ``social
complexity'' to be the outcome \(Y_{t2}\), and ``economic trade'' to the
un-intentionally stratified mediator \(L_{t1}\).

In this example, beliefs in Big Gods \(A_{t0}\) directly influence
economic trade \(L_{t1}\), which then affects social complexity
\(Y_{t2}\). Conditioning on economic trade \(L_{t1}\) may biased
estimates of the overall effect of beliefs in Big Gods \(A\) on social
complexity \(Y_{t2}\) downward. How the total effect of \(A\) on \(Y\)
may be diminished by conditioning on an intervening \(L\) is presented
in Figure~\ref{fig-dag-mediator}.

\begin{figure}

{\centering \includegraphics[width=0.8\textwidth,height=\textheight]{causal-dags_files/figure-pdf/fig-dag-mediator-1.pdf}

}

\caption{\label{fig-dag-mediator}Confounding by conditioning on a
mediator. The dashed black arrow indicates bias arising from partially
blocking the path between A and Y. Here, a true effect of A on Y is
attenuated.}

\end{figure}

\subsubsection{\texorpdfstring{Advice: do not condition on the mediator.
Ensure \(L\) occurs before
\(A\)}{Advice: do not condition on the mediator. Ensure L occurs before A}}\label{advice-do-not-condition-on-the-mediator.-ensure-l-occurs-before-a}

The advice in this setting is precisely the same advice as in the
previous settings. As my Roman Catholic mother says, ``timing is
everything.'' To manage the issue of conditioning on a common effect, it
is crucial to maintain the correct temporal order.

\begin{enumerate}
\def\labelenumi{\arabic{enumi}.}
\tightlist
\item
  Measure all confounders \(L_{t0}\), which are common causes of both
  the exposure \(A_{t1}\) and the outcome \(Y_{t2}\).
\item
  Ensure that \(L_{t0}\) is measured before \(A_{t1}\) occurs.
\item
  Ensure that \(A_{t1}\) is measured before \(Y_{t2}\) occurs.
\end{enumerate}

Again we find that adhering to the temporal sequence in data
collection/organisation precludes \(L\) from being an effect of either
\(A\) or \(Y\). We have seeen this solution before. The solution is
presented in Figure~\ref{fig-dag-common-effect-solution-2}. This figure
is identical to Figure~\ref{fig-dag-common-cause-solution}. One should
generally only condition on a mediator if our interest pre-specified
causal question requires a causal mediation model (The assumptions of
causal mediation are discussed in Section 3).

\begin{figure}

{\centering \includegraphics[width=0.8\textwidth,height=\textheight]{causal-dags_files/figure-pdf/fig-dag-common-effect-solution-2-1.pdf}

}

\caption{\label{fig-dag-common-effect-solution-2}Solution: we avoid
mediator bias by ensuring the correct temporal measurement of the
confounder.}

\end{figure}

\subsubsection{3. The elemental confounding from conditioning on a
common effect (collider
stratification)}\label{the-elemental-confounding-from-conditioning-on-a-common-effect-collider-stratification}

\begin{enumerate}
\def\labelenumi{\arabic{enumi}.}
\tightlist
\item
  \textbf{Case when the collider is a common effect of the exposure and
  outcome}
\end{enumerate}

Consider a scenario in which a variable \(L\) is influenced by both a
treatment \(A\) and an outcome \(Y\) (\citeproc{ref-cole2010}{Cole
\emph{et al.} 2010}). Conditioning on a common effect, \(L\) will open a
non-causal association between \(A\) and \(Y\).

Imagine \(A\) represents the level of belief in Big Gods, and \(Y\)
denotes social complexity, with \(L\) being economic trade. Initially,
suppose there is no causal link between \(A\) and \(Y\) --- altering
belief in Big Gods does not impact social complexity directly. However,
assume both \(A\) and \(Y\) independently affect economic trade (\(L\)).
If we analyze the data, ignoring the temporal sequence, we might
mistakenly infer a causal relationship between \(A\) and \(Y\) arising
from their shared effect on \(L\). This spurious association might lead
to the false conclusion of a direct link between beliefs in Big Gods and
social complexity in cross-sectional data.\footnote{In mathematical
  terms, when \(A\) and \(Y\) are independent, their joint probability
  should equal the product of their individual probabilities:
  \(P(A, Y) = P(A)P(Y)\). But, conditioning on \(L\) alters this
  relationship. The joint probability of \(A\) and \(Y\) given \(L\),
  \(P(A, Y | L)\), does not equal the product of \(P(A | L)\) and
  \(P(Y | L)\). Thus, the common effect \(L\) creates an apparent
  association between \(A\) and \(Y\), which is not causal.} This
probelm of collider bias is presented
Figure~\ref{fig-dag-common-effect}.

\begin{figure}

{\centering \includegraphics[width=0.8\textwidth,height=\textheight]{causal-dags_files/figure-pdf/fig-dag-common-effect-1.pdf}

}

\caption{\label{fig-dag-common-effect}Confounding by conditioning on a
collider. The red path indicates bias from the open backdoor path from A
to Y.}

\end{figure}

\subsubsection{\texorpdfstring{Advice: do not condition on a common
efect. Rather, ensure \(L\) is measured before
\(A\)}{Advice: do not condition on a common efect. Rather, ensure L is measured before A}}\label{advice-do-not-condition-on-a-common-efect.-rather-ensure-l-is-measured-before-a}

Again we find that adhering to assumed causal order in data collection
and analysis precludes \(L\) from being an effect of either \(A\) or
\(Y\). We have seen this solution before, in
Figure~\ref{fig-dag-common-cause-solution} and .
Figure~\ref{fig-dag-common-effect-solution-2}. We repeat the solution in
Figure~\ref{fig-dag-common-effect-solution-3}.

\begin{figure}

{\centering \includegraphics[width=0.8\textwidth,height=\textheight]{causal-dags_files/figure-pdf/fig-dag-common-effect-solution-3-1.pdf}

}

\caption{\label{fig-dag-common-effect-solution-3}Solution: we ensure
that A and Y are d-separated by ensuring L is measured before A occurs.}

\end{figure}

\begin{enumerate}
\def\labelenumi{\arabic{enumi}.}
\setcounter{enumi}{1}
\tightlist
\item
  \textbf{Case when the collider is the effect of exposure}
\end{enumerate}

We have considered mediator bias attenutates the total effect of \(A\)
on \(Y\). However, we should not imagine that conditioning on the effect
of an exposure will only bias effect estimates downward. Consider a
scenario in which \(L\) is affected by both the exposure \(A\) and an
unmeasured variable \(U\) that is related to the outcome \(Y\) but not
to \(A\). Assume that there is no causal effect of \(A\) on \(Y\).
Neverthess, as presented in Figure~\ref{fig-dag-descendent},
conditioning on \(L\) will induce a non-causal association between \(A\)
and \(Y\)

\begin{figure}

{\centering \includegraphics[width=0.8\textwidth,height=\textheight]{causal-dags_files/figure-pdf/fig-dag-descendent-1.pdf}

}

\caption{\label{fig-dag-descendent}Confounding by descent: the red path
illustrates the biasing path introduced by conditioning on the
descendant of a confounder U that is affected by the exposure, openning
a backdoor path between A and Y.}

\end{figure}

This is setting of post-exposure \emph{collider bias}. Conditioning on
the collder \(L_{t1}\) in the analysis risks inducing a non-causal
association between \(A_{t0}\) and \(Y_{t2}\). This post-treatment
collider bias is provoked by failing to presever the causal order in
one's data collection.

\subsubsection{\texorpdfstring{Advice: do not condition on a common
efect. Rather, ensure \(L\) is measured before
\(A\)}{Advice: do not condition on a common efect. Rather, ensure L is measured before A}}\label{advice-do-not-condition-on-a-common-efect.-rather-ensure-l-is-measured-before-a-1}

Thie advice is the same as in \textbf{?@fig-dag-common-effect-solution},
Figure~\ref{fig-dag-common-effect-solution-2}, and
Figure~\ref{fig-dag-common-effect-solution-3}. There is no need to
repeat the graph.

\paragraph{Case of conditioning on a pre-exposure collider
(M-bias)}\label{case-of-conditioning-on-a-pre-exposure-collider-m-bias}

One must be cautious not to over-condition on pre-exposure variables. In
settings where we condition on a variable that is itself not associated
with the exposure or outcome, but is the descendent of an unmeasured
instrumental variable as well as of an unmeasured cause of the outcome,
we may inadvertently induce confounding known as `M-bias', illustrated
in Figure~\ref{fig-m-bias},

M-bias can arise even though a variable \(L\) that induces it occurs
before the treatment \(A\). Conditioning on \(L\) creates a spurious
association between \(A\) and \(Y\) by opening the path between the
unmeasured confounders. Here, we assume that \(A\) and \(Y\) might be
unconditionally independent (\(A \coprod Y(a)\)). However, when
stratified by \(L\), this independence is violated:
(\(A \cancel{\coprod} Y(a)| L\)). This form of bias is another
manifestation of collider stratification bias, one pertaining to
pre-exposure variables in certain structural scenarios.\footnote{When
  the path is ordered chronologically from left to right, the ``M''
  shape, giving M-bias its name, changes to an ``E'' shape. However, the
  term ``M-bias'' is retained.}

\begin{figure}

{\centering \includegraphics[width=0.8\textwidth,height=\textheight]{causal-dags_files/figure-pdf/fig-m-bias-1.pdf}

}

\caption{\label{fig-m-bias}M-bias: Confounding control by including
previous outcome measures. The dashed red path indicates bias from the
open backdoor path from A to Y by conditioning on pre-exposure variable
L. The solution: do not condition on L. The graph shows that
conditioning on variables measured before the exposure is not sufficient
to prevent confounding.}

\end{figure}

\subsubsection{Advice: take care when selecting pre-exposure
variables}\label{advice-take-care-when-selecting-pre-exposure-variables}

dopting an indiscriminate approach, what McElreath aptly calls the
``causal salad''(\citeproc{ref-mcelreath2020}{McElreath 2020}), may
induce bias, even when the confounders controlled for occur before the
exposure. Despite the utiility of chronological hygiene for our causal
diagrams, chronological hygiene in our analysis is not sufficient
strategy for reducing bias. Each problem must be considered in light of
its features, by the best-lights of subject-matter experts.

\subsubsection{4. The promise and perils of condition on a descendant
(for good or
bad).}\label{the-promise-and-perils-of-condition-on-a-descendant-for-good-or-bad.}

Recall that conditioning on a descendent functions as partially
conditioning on its parents.

\paragraph{1. Case when conditioning on a descendant amplifies
bias}\label{case-when-conditioning-on-a-descendant-amplifies-bias}

Suppose a team of anthropologists is studying the relationship between
the use of a specific social ritual \(A\) and the level of technological
advancement \(Y\) in different human societies.

Let \(U\), represent historically distant families, which influences
both the development of unique social rituals \(A\) (isolated language
families may develop distinct cultural practices). Let us suppose there
is no causal link between language family as such and technological
advancement.

Suppose \(S\) is the extent to which a society's culture is studied, and
that this is linked to both to social complexity and language families
That is, suppose technologically advanced societies are more likely to
be documented from better documentation and more linguistically
accessible documents.

\begin{figure}

{\centering \includegraphics[width=0.8\textwidth,height=\textheight]{causal-dags_files/figure-pdf/fig-dag-selection-1.pdf}

}

\caption{\label{fig-dag-selection}Confounding by descendant of the
outcome: the red path illustrates the biasing path introduced by
conditioning on the descendant of a confounder U that is affected by the
outcome Y, leading to a non-causal association of between A and Y. This
is an example of selection bias. It cannot be undone by conditioning. To
remove this bias, we must accurately measure Y.}

\end{figure}

As indicated in Figure~\ref{fig-dag-selection}, if anthropological
studies focuses only on societies that have been extensively studied and
documented \(S\), we condition on an effect of \(Y\) and an unmeasured
confounder \(U\). This conditioning opens a non-causal path between the
social ritual \(A\) and technological advancement \(Y\). Here we have an
instance of \emph{selection bias}. This bias is particularly insidious
because we cannot ``uncondition'' the dataset as they exist by
conditioning on measured variables. The threat cannot be not easily
undone because it arises after the exposure has occurred.

By conditioning on \(S\) (extent of study), we introduce a spurious
association between the social ritual and technological advancement. We
may incorrectly conclude that certain social rituals are directly linked
to higher or lower levels of technological development. In reality, the
observed correlation may arise merely because less isolated societies,
which are more likely to be studied, may independently develop specific
social ritual and but aquire technologies for unrelated reasons.

\paragraph{Advice: we cannot address this form of selection bias by
conventional
means}\label{advice-we-cannot-address-this-form-of-selection-bias-by-conventional-means}

We cannot address this form of selection bias through confentional
confounding control. Here, our causal diagram is useful because it tells
use we need to stop, and consider how to recover unbaised measurements
of Y. {[}CITE{]}

\paragraph{Case when conditioning on a descendant reduces
bias}\label{case-when-conditioning-on-a-descendant-reduces-bias}

Next consider a case in which we may use a post-treatment descendent to
reduce bias. Suppose an unmeasured confounder \(U\) affects \(A\),
\(Y\), and \(L^\prime\) in an effect of \(U\) that occurs after \(A\)
and \(Y\). In this scenario adjusting for \(L^\prime\) may help to
reduce confounding caused by the unmeasured confounder \(U\). This
strategy follows from the modified disjunctive cause criterion for
confounding control, we recommends that we ``include as a covariate any
proxy for an unmeasured variable that is a common cause of both the
exposure and the outcome'' (\citeproc{ref-vanderweele2019}{VanderWeele
2019}). As shown in \textbf{?@fig-dag-descendent-solution-2}, although
\(L^\prime\) occurs \emph{after} the exposure, and indeed occur
\emph{after} the outcome, coniditioning on it will reduce confounding.
How might this work? Consider a genetic factor that affects the exposure
and the outcome early in life but which is expressed later later in
life. Adjusting for such an expression of the genetic factor that
expresses later in life would help use to control for the unmeasured
confounding by common cause from the genetic factors influence on \(A\)
and \(Y\), which again are imagined to occur before \(L'\). Here
conditioning on \(L'\) is sensible, and provides an example of
post-outcome confounding control. This scenario is presented in
\textbf{?@fig-dag-descendent-solution-2}.

\paragraph{Advice: when developing a conditioning set, adopt the
modified disjunctive cause
criterion}\label{advice-when-developing-a-conditioning-set-adopt-the-modified-disjunctive-cause-criterion}

The prospect that we may use descendants for confounding control reveals
that even if for a causal diagram, ``timing is everything,'' when it
comes to the analysis of a problem, \textbf{structure is everything}.
The chronologically hygenic graph reveals scope for conditioning
strategies on confounders measures after the exposure or outcome. It
brings home the point that we should think of the concept of a
counfounder as meaningful only in relation to the adjustment set in
which it forms a part.

We are now in position to understand why I advocate using VanderWeele's
modified disjunctive cause criteria for selecting this confounder set in
pratice. Assuming our causal diagram is accurate, we should:

\begin{enumerate}
\def\labelenumi{\alph{enumi}.}
\tightlist
\item
  \textbf{Control for any variable that causes the exposure, the
  outcome, or both:}
\item
  \textbf{Control for any proxy for an unmeasured variable that is a
  shared cause of both exposure and outcome.}
\item
  \textbf{Exclude any instrumental variable that is not a proxy for an
  unmeasured confounder from the confounder set.}
\end{enumerate}

This approach will prevent M-bais and require than all instrumental
variables are presummed unsuitable for inclusion unless we can establish
they are descendants of unmeasured common confounders of the exposure
and outcome.

Practically speaking, however,determining which variables belong in the
confounder set can be challenging. We can only be guided by the best
lights of specialist, however, science is the practice of shifting the
standard of those lights. Therefore it is critical to do our best at
confounding control, and then perform sensitivity analyses. See:
VanderWeele \emph{et al.} (\citeproc{ref-vanderweele2020}{2020}) and
VanderWeele (\citeproc{ref-vanderweele2019}{2019}).

\subsection{Part 3. Application of Causal Diagrams for Clarifying
Moderation (Interaction), Mediation, and Longitudinal
Feedback}\label{part-3.-application-of-causal-diagrams-for-clarifying-moderation-interaction-mediation-and-longitudinal-feedback}

\subsubsection{Case 1. Causal Interaction and Causal Effect
Modification: do not draw non-linear relationships such as
interactions}\label{case-1.-causal-interaction-and-causal-effect-modification-do-not-draw-non-linear-relationships-such-as-interactions}

We often wish to understand whether causal effects operate differently
in different sub-populations, or whether the joint effect of two
interventions differ from the either taken alone, and from no
intervention. This renders questions of causal interaction scientific
interesting.

How shall we depict causal interactions on a graph? It is crucial to
remember the primary function of causal diagrams is to investigate
confounding. This task does not demand that the graph capture non-linear
relationships or interactions. Indeed, causal diagrams should not
attempt to capture all facets of a phenomenon under investigation
because doing so a distraction from the task at hand: ascertain the
conditional indepencies that might compromise causal inferences. We
therefore should not attempt any unique visual trick to show additive
and multiplicative interaction in a causal diagram. Moreover, gain, we
should include those nodes and paths that are necessary to evaluate
structural sources of bias that might compromise the pre-specifid causal
contrasts of interest.

Therefore let us consider the types of causal contrasts that questions
of interaction may direct us to specify. To do so will require that we
clarify our causal questions. Interaction takes on different meanings
depending on the question we wish to answer. Consider two very different
ways for stating questions of causal interaction.

\paragraph{Distinction 1: causal interaction as a question of double
exposure}\label{distinction-1-causal-interaction-as-a-question-of-double-exposure}

Causal interaction may refer to the combined and separate effect of two
distinct exposures. We say that evidence for causal interaction is
present a given scale when the effect of one exposure on an outcome
depends on another exposure's level. For instance, the effect of beliefs
in Big Gods (exposure \(A\)) on social complexity (outcome \(Y\)) might
depend on a culture's monumental architecture (exposure \(B\)), which
could also influence social complexity. If we are interested in the
separate effects of \(A\) and \(B\) we might say that evidence of causal
interaction on the difference scale would be present if:

\[\bigg(\underbrace{\mathbb{E}[Y(1,1)]}_{\text{joint exposure}} - \underbrace{\mathbb{E}[Y(0,0)]}_{\text{neither exposed}}\bigg) - \bigg[ \bigg(\underbrace{\mathbb{E}[Y(1,0)]}_{\text{only A exposed}} - \underbrace{\mathbb{E}[Y(0,0)]}_{\text{neither exposed}}\bigg) + \bigg(\underbrace{\mathbb{E}[Y(0,1)]}_{\text{only B exposed}} - \underbrace{\mathbb{E}[Y(0,0)]}_{\text{neither exposed}} \bigg)\bigg] \neq 0 \]

This equation simplifies to

\[ \underbrace{\mathbb{E}[Y(1,1)]}_{\text{joint exposure}} - \underbrace{\mathbb{E}[Y(1,0)]}_{\text{only A exposed}} - \underbrace{\mathbb{E}[Y(0,1)]}_{\text{only B exposed}} + \underbrace{\mathbb{E}[Y(0,0)]}_{\text{neither exposed}} \neq 0 \]

If the above equation were to hold, we would infer that the effect of
exposure \(A\) on the outcome \(Y\) would differ across levels of \(B\)
or vice versa. Such a difference would provide evidence for interaction.

More specifically, if the value of the equation were positive, we would
say there is evidence for an additive effect. If the value were less
than zero, we would say there is evidence for a sub-additive effect. If
the value were virtually zero, we would say there is no reliable
evidence for interaction.\footnote{Note that causal effects of
  interactions often differ when measured on the ratio scale. This
  discrepency can have significant policy implications, see:
  (\citeproc{ref-vanderweele2014}{VanderWeele and Knol 2014}). Although
  beyond the scope of this article, it is worth emphasising again that
  when evaluating evidence for causality, in addition to specifying the
  exposure and outcome, we must specify the measure of effect in which
  we are interested, as well as the target population for whom we wish
  to generalise (\citeproc{ref-hernuxe1n2004}{Hernán 2004};
  \citeproc{ref-tripepi2007}{Tripepi \emph{et al.} 2007}).}

Remember again that causal diagrams are non-parametric. They do not
directly represent interactions. A causal diagram can indicate an
interaction's presence by displaying two exposures jointly influencing
an outcome, while remaining independent of each other.
Figure~\ref{fig-dag-interaction} provides an example.

\begin{figure}

{\centering \includegraphics[width=0.6\textwidth,height=\textheight]{causal-dags_files/figure-pdf/fig-dag-interaction-1.pdf}

}

\caption{\label{fig-dag-interaction}Causal interaction: if two exposures
are causally independent of each other, we may wish to estimate their
individual and joint effects on Y, conditional on confounding control
strategy that blocks backdoor paths for bothe exposures (here, L1 and L2
are jointly required). where the counterfactual outcome is Y(a,b) and
there is evidence for additive or subadditive interaction if E{[}Y(1,1)
- Y(0,1) - Y(1,0) + Y(0,0){]} ≠ 0. If we cannot conceptualise B as a
variable upon which intervention can occur, then the interaction is
better conceived as effect modification (see next figure). Important: do
not attempt to draw a path into another path.}

\end{figure}

Note that the chronological order in Figure~\ref{fig-dag-interaction}
reveals demands on data collection To ensure that \(A\) and \(B\) do not
effect each other requires a combination of expert knowledge and
measurements of \(A\) and \(B\) and intervals in which there can be no
reciprocal effects.

\paragraph{Distinction 2. Causal effect modification under a single
exposure}\label{distinction-2.-causal-effect-modification-under-a-single-exposure}

With the analysis of effect modification, we aim to understand how an
exposure's effect varies, if at all, across stratums of another variable
or in more complex cases across stratums of other variables. We call a
stratum of a variable in which an exposure may operate differently, an
`effect modifier.'

Consider again the problem of estimating a causal effect of beliefs in
Big Gods on social complexity. Suppose this time we are interested in
the investigating whether this effect varies by region. In this example,
perhaps we are interested in comparing regions North American societies
with Continental Societies. In this setting, geography is conceived as
an ``effect modifier.'' Notice that we do not wish to treat the
effect-modifier as an intervention, and indeed it is conceptually
implausible to do so. Rather, we wish to investigate whether geography
is a parameter that may alter a well-defined exposure's effect on a
well-defined outcome.

FFor clarity, consider two exposure levels, denoted as \(A = a\) and
\(A = a^*\). Additionally, assume that \(G\) represents two distinct
groups, such as \(g\) and \(g'\), where these groups could be based on
different geographical characteristics.

The expected outcome when the exposure is at level \(A = a\) among
individuals in group \(G = g\) is expressed:

\[\hat{E}[Y(a)|G=g]\]

This represents the average outcome under exposure \(a\) for group
\(g\).

Similarly, the expected outcome for exposure level \(A = a^*\) among
individuals in the same group (\(G = g\)) is expressed:

\[\hat{E}[Y(a^*)|G=g]\]

The causal effect of shifting the exposure level from \(a^*\) to \(a\)
within group \(g\) is thus expressed:

\[\hat{\delta}_g = \hat{E}[Y(a)|G=g] - \hat{E}[Y(a^*)|G=g]\]

The quantity on the left computes the change in the expected outcome
from altering the exposure from \(a^*\) to \(a\) within group \(g\).

Likewise, the causal effect of changing the exposure from \(a^*\) to
\(a\) within group \(g'\) is expressed:

\[\hat{\delta}_{g'} = \hat{E}[Y(a)|G=g'] - \hat{E}[Y(a^*)|G=g']\]

Here, \(\hat{\delta}_{g'}\) captures the analogous effect of the
exposure in group \(g'\).

To understand effect modification, we compare the conditional causal
effect estimate on a difference scale between these two groups, which we
calculate as:

\[\hat{\gamma} = \hat{\delta}_g - \hat{\delta}_{g'}\]

The value of \(\hat{\gamma}\) quantifies the differential effect of
shifting the exposure from \(a^*\) to \(a\) between groups \(g\) and
\(g'\). A non-zero \(\hat{\gamma}\) indicates evidence of effect
modification, suggesting that the impact of changing the exposure indeed
varies based on group characteristics. If \(G\) represents a geographic
distinction, then \(\hat{\gamma} \neq 0\) would suggest geographical
variation in the exposure effect.\footnote{For distinctions within
  varieties of effect modification relevant for strategies of
  confounding controul see (\citeproc{ref-vanderweele2007}{VanderWeele
  and Robins 2007}).}

\begin{figure}

{\centering \includegraphics[width=0.8\textwidth,height=\textheight]{causal-dags_files/figure-pdf/fig-dag-effect-modfication-1.pdf}

}

\caption{\label{fig-dag-effect-modfication}A simple graph for
effect-modification in which there are no confounders. G is an effect
modifier of A on Y. We draw a box around G to indicate we are
conditioning on this variable.}

\end{figure}

\subsubsection{Causal mediation: causal diagrams reveal the inadequacy
of standard
approaches}\label{causal-mediation-causal-diagrams-reveal-the-inadequacy-of-standard-approaches}

Mediation analysis within the human sciences is notably plagued by
confusion. This stems partly from the inherent complexity of the causal
relationships that mediation seeks to clarify. Simply put, causal
mediation analysis is inherently challenging. However, much of the
current confusion results from the misguided practice of using so-called
of constructing (so-called) structural equation models directly from
(so-called) `structural equation graphs'. Typically, the models this
tradition produces are stastical models that lack any systematic
relationship to the counterfactual contrasts that stand are the basis of
causal mediation analysis. Lacking a conceptually clear framework for
assessing structural relationships, the prevailing traditions lead us to
ambiguous results without any guarantees. However, by combining careful
causal reasoning with a judicious use of causal diagrams, we can diagnos
the flaws in the current practices and develop alternatives that may
lead to clearer, scientifically relevant understanding..

\paragraph{Defining the estimands}\label{defining-the-estimands}

To better understand what we are getting with causal mediation, it is
useful decompose the the total effect into the natural direct and
indirect effects.

We define the total effect of the treatment \(A\) on the outcome \(Y\)
as the overall difference between the potential outcomes when the
treatment is applied compared to when it is not. We have seen this
effect before, however, because are presenting the problem by referring
to the `full data' that include all counterfactual outcomes, we will not
include the expected values. Our estimand for the total effect is:

\[
\Delta_{ATE} = TE = Y(1) - Y(0)
\]

We may decomposed this total effect estimand into the direct and
indirect effects of a mediated effect as follows:

We decompose the potential outcome Y(1) as:

\[ 
Y(1) = Y(1, M(1))
\]

This describes the effect of the exposure, here set to \(A = 1\) with
the mediator taking the value it would naturally take in the presence of
the exposure when it is set to 1.

We decompose the potential outcome Y(0) as:

\[ 
Y(0) = Y(0, M(0))
\]

This quantity describes the effect of the exposure, here set to 0, with
the mediator taking the value that it would naturally take in the
presence of the exposure when the exposure is set to 0.

Next, we offer the following definitions:

\textbf{Natural Direct Effect (NDE):} this is the effect of the
treatment on the outcome, keeping the mediator at the level it would
have been if the treatment had not been applied. I outline the unusual
portion of this counterfactual quantity in blue.

This quantity is expressed:

\[
 NDE = \textcolor{blue}{Y(1, M(0))} - Y(0, M(0))
 \]

\textbf{Natural Indirect Effect (NIE):} this is the effect of the
treatment on the outcome that operates through the mediator. It compares
the potential outcome under treatment where the mediator assumes its
natural level under treatment with the potential outcome under treatment
where the mediator assumes its natural value under no-treatment. Again I
outline the unusual portion of this counterfactual quantity in blue.

This quantity is expressed:

\[
 NIE = Y(1, M(1)) - \textcolor{blue}{Y(1, M(0))}
 \]

This decomposition can be rearranged to show that the total effect is
the sum of the natural direct and indirect effects. We simply add
subtract and add the term \$ textcolor\{blue\}\{Y(1, M(0))\}\$ to the
equation for the Recover the Total Effect. These terms are highlighted
in blue:

\[
TE = NDE + NIE = [Y(1, M(1)) - \textcolor{blue}{Y(1, M(0))}] + [\textcolor{blue}{Y(1, M(0))} - Y(0, M(0))]
\]

This decomposition of the total effect into the natural direct and
indirect effect clarifies the generic form of the pre-specified
estimands that we recover in causal mediation when interested in natural
direct and indirect effects. Without explicitly conceptualising this
targets as counterfactual contrasts, however, we do not know what we are
getting when applying statistical models to data -- as happens in the
structural equation modelling traditions. Again, structural equations
models are a misnomer. The models produced in this tradition are rather
statistical models that lack an automatic structural basis for
interpretation. More acturately we would describe them as un-structural
equation models. On the other hand, approaching mediation from the
structural perspective afforded by causal data sciences allows us to
decompose the Total Effect into the part that is mediated by changes in
the mediator due to the treatment (NIE) and the part that is not
mediated by the mediator (NDE). It is only with these targetted
counterfactua contrasts in mind that we can begin to address questions
of causal mediation to obtain valid inferences -- or in cases where the
stringent requirements remain elusive -- to understand why it is would
be inappropriate to excercise anything more than restraint.

\paragraph{Chronological causal diagrams in causal mediation
analysis}\label{chronological-causal-diagrams-in-causal-mediation-analysis}

The conditions necessary for causal mediation are stringent. I next
clarify these conditions in the chronologically ordered causal diagram
,shown in Figure~\ref{fig-dag-mediation-assumptions}. We will again pose
whether cultural beliefs in Big Gods affect social complexity, mediated
by political authority. Assume for the sake of argument these poorly
defined questions are well defined. What would we need to answer them?

\begin{enumerate}
\def\labelenumi{\arabic{enumi}.}
\tightlist
\item
  \textbf{No unmeasured exposure-outcome confounder}
\end{enumerate}

This prerequisite is expressed: \(Y(a,m) \coprod A | L1\). Upon
controlling for the covariate set \(L1\), we must ensure that no
additional unmeasured confounders affect both the cultural beliefs in
Big Gods \(A\) and the social complexity \(Y\). For example, suppose our
study involves the effect of cultural beliefs in Big Gods (exposure) on
social complexity (outcome), and geographic location and historical
context define the covariates in \(L1\). In that case, we must assume
that accounting for \(L1\) d-separates \(A\) and \(Y\). The relevant
confounding path is depicted in brown in
Figure~\ref{fig-dag-mediation-assumptions}.

\begin{enumerate}
\def\labelenumi{\arabic{enumi}.}
\setcounter{enumi}{1}
\tightlist
\item
  \textbf{No unmeasured mediator-outcome confounder}
\end{enumerate}

This condition is expressed: \(Y(a,m) \coprod M | L2\). After
controlling for the covariate set \(L2\), we must ensure that no other
unmeasured confounders affect the political authority \(M\) and social
complexity \(Y\). For instance, if trade networks impact political
authority and social complexity, we must account for trade networks to
obstruct the unblocked path linking our mediator and outcome. Further,
we must assume the absence of any other confounders for the
mediator-outcome path. This confounding path is represented in blue in
Figure~\ref{fig-dag-mediation-assumptions}.

\begin{enumerate}
\def\labelenumi{\arabic{enumi}.}
\setcounter{enumi}{2}
\tightlist
\item
  \textbf{No unmeasured exposure-mediator confounder}
\end{enumerate}

This requirement is expressed: \(M(a) \coprod A | L3\). Upon controlling
for the covariate set \(L3\), we must ensure that no additional
unmeasured confounders affect both the cultural beliefs in Big Gods
\(A\) and political authority \(M\). For example, the capability to
construct large ritual theatres may influence the belief in Big Gods and
the level of political authority. If we have indicators for this
technology measured prior to the emergence of Big Gods (these indicators
being \(L3\)), we must assume that accounting for \(L3\) closes the
backdoor path between the exposure and the mediator. This confounding
path is shown in green in Figure~\ref{fig-dag-mediation-assumptions}.

\begin{enumerate}
\def\labelenumi{\arabic{enumi}.}
\setcounter{enumi}{3}
\tightlist
\item
  \textbf{No mediator-outcome confounder affected by the exposure}
\end{enumerate}

This requirement is expressed: \(Y(a,m) \coprod M(a^*) | L\). We must
ensure that no variables confounding the relationship between political
authority and social complexity in \(L2\) are themselves influenced by
the cultural beliefs in Big Gods (\(A\)). For instance, when studying
the effect of cultural beliefs in Big Gods (\(A\), the exposure) on
social complexity (\(Y\), the outcome) as mediated by political
authority (mediator), there can be no factors, such as trade networks
(\(L2\)), that influence both political authority and social complexity
and are affected by the belief in Big Gods. This confounding path is
shown in red in Figure~\ref{fig-dag-mediation-assumptions}. \textbf{Note
that the assumption of no exposure-induced confounding in the
mediator-outcome relationship is often a substantial obstacle for causal
mediation analysis.} If the exposure influences a confounder of the
mediator and outcome, we face a dilemma. Without accounting for this
confounder, the backdoor path between the mediator and the outcome
remains open. By accounting for it, however, we partially obstruct the
path between the exposure and the mediator, leading to bias. In this
setting, we cannot recover the natural direct and indirect effects from
the data (\citeproc{ref-vanderweele2015}{VanderWeele 2015}).

Notice again that the requirements for counterfactual data science are
considerably more strict than has been appreciated in the human
sciences, particularly those in which the structural equation modelling
traditions have exerted influence. An entire generation of researchers
must unlearn the habit of leaping from a description of a statistical
process as embodied in a structural equation diagram into the analysis
of the data. As Robins and Greeland have shown, we simply do not know
what quantities we are estimating without first specifying the estimands
of interest in terms of the targeted counterfactuals of interest
(\citeproc{ref-robins1992}{Robins and Greenland 1992}). Moreover, where
the Natural Direct and Indirect Effects are of interest, such estimands
require conceptualising a rather unusual counterfactual that is
\emph{never} directly observed from the data, namely:
\(\textcolor{blue}{Y(1, M(0))}\), and simulating it from data only when
stringent assumptions are satisfied (an outstanding resource on this
topic is VanderWeele (\citeproc{ref-vanderweele2015}{2015})).\footnote{I
  encourage readers interested in causal interaction and causal
  mediation to study VanderWeele (\citeproc{ref-vanderweele2015}{2015}).}

\begin{figure}

{\centering \includegraphics[width=1\textwidth,height=\textheight]{causal-dags_files/figure-pdf/fig-dag-mediation-assumptions-1.pdf}

}

\caption{\label{fig-dag-mediation-assumptions}This causal diagram
illustrates the four fundamental assumptions needed for causal mediation
analysis. The first assumption pertains to the brown paths. It requires
the absence of an unmeasured exposure-outcome confounder, and assumes
that conditioning on L1 is sufficient for such confounding control. The
second assumption pertains to the blue paths. It requires the absence of
an unmeasured mediator-outcome confounder, and assumes that conditioning
on L2 is sufficient for such confounding control. The third assumption
pertains to the green paths. It requires the absence of an unmeasured
exposure-mediator confounder, and assumes that conditioning on L3 is
sufficient for such confounding control. The fourth and final assumption
pertains to the red paths. It requires the absence of an a
mediator-outcome confounder that is affected by the exposure, and
assumes that there is no path from the exposure to L2 to M. If the
exposure were to affect L2, then conditioning on L2 would block the
exposure's effect on the mediator, as indicated by dashed red path.
Causal diagrams not only clarify how different types of confounding bias
may converge (here mediation bias and confounder bias), but also reveal
the limitations of common methods such as structural equation models and
multilevel models for handling time-series data where the fourth
assumption fails -- that is, where there is treatment-confounder
feedback. Such feedback is common in time-series data, but not widely
understood. For example structural equation models and multi-level
models cannot address causal questions in the presence of such feedback,
but these models remain widely favoured.}

\end{figure}

\subsubsection{Case 3: Confounder-treatment feedback: longitudinal
``growth'' is not
causation}\label{case-3-confounder-treatment-feedback-longitudinal-growth-is-not-causation}

In our discussion of causal mediation, we consider how the effects of
two sequential exposures may combine to affect an outcome. We can
broaden this interest to consider the causal effects of multiple
sequential exposures. In such scenarios, causal diagrams arranged
chronologically can aid in clarifying the challenges and opportunities.

For example, consider temporally fixed multiple exposures. The
counterfactual outcomes may be denoted \(Y(a_{t1} ,a_{t2})\). There are
four counterfactual outcomes corresponding to the four fixed ``treatment
regimes:''

\begin{enumerate}
\def\labelenumi{\arabic{enumi}.}
\item
  \textbf{Always treat (Y(1,1))}
\item
  \textbf{Never treat (Y(0,0))}
\item
  \textbf{Treat once first (Y(1,0))}
\item
  \textbf{Treat once second (Y(0,1))}
\end{enumerate}

\phantomsection\label{tbl-regimes}
\begin{longtable}[]{@{}
  >{\raggedright\arraybackslash}p{(\columnwidth - 4\tabcolsep) * \real{0.1351}}
  >{\raggedright\arraybackslash}p{(\columnwidth - 4\tabcolsep) * \real{0.5405}}
  >{\raggedright\arraybackslash}p{(\columnwidth - 4\tabcolsep) * \real{0.3243}}@{}}
\caption{\label{tbl-regimes}Table describes four fixed treatment regimes
and six causal contrasts in time series data where the exposure may
vary.}\tabularnewline
\toprule\noalign{}
\begin{minipage}[b]{\linewidth}\raggedright
Type
\end{minipage} & \begin{minipage}[b]{\linewidth}\raggedright
Description
\end{minipage} & \begin{minipage}[b]{\linewidth}\raggedright
Counterfactual Outcome
\end{minipage} \\
\midrule\noalign{}
\endfirsthead
\toprule\noalign{}
\begin{minipage}[b]{\linewidth}\raggedright
Type
\end{minipage} & \begin{minipage}[b]{\linewidth}\raggedright
Description
\end{minipage} & \begin{minipage}[b]{\linewidth}\raggedright
Counterfactual Outcome
\end{minipage} \\
\midrule\noalign{}
\endhead
\bottomrule\noalign{}
\endlastfoot
Regime & Always treat & Y(1,1) \\
Regime & Never treat & Y(0,0) \\
Regime & Treat once first & Y(1,0) \\
Regime & Treat once second & Y(0,1) \\
Contrast & Always treat vs.~Never treat & E{[}Y(1,1) - Y(0,0){]} \\
Contrast & Always treat vs.~Treat once first & E{[}Y(1,1) - Y(1,0){]} \\
Contrast & Always treat vs.~Treat once second & E{[}Y(1,1) -
Y(0,1){]} \\
Contrast & Never treat vs.~Treat once first & E{[}Y(0,0) - Y(1,0){]} \\
Contrast & Never treat vs.~Treat once second & E{[}Y(0,0) - Y(0,1){]} \\
Contrast & Treat once first vs.~Treat once second & E{[}Y(1,0) -
Y(0,1){]} \\
\end{longtable}

There are six causal contrasts that we might compute for the four fixed
regimes, presented in Table~\ref{tbl-regimes}.\footnote{We compute the
  number of possible combinations of contrasts by
  \(C(n, r) = \frac{n!}{(n-r)! \cdot r!}\)}

Not that treatment assignments might be sensibly approached as a
function of the previous outcome. For example, we might \textbf{treat
once first} and then decide whether to treat again depending on the
outcome of the initial treatment. This aspect is known as ``time-varying
treatment regimes.''

Bear in mind that to estimate the ``effect'' of a time-varying treatment
regime, we are obligated to make comparisons between the relevant
counterfactual quantities. As mediation can introduce the possibility of
time-varying confounding (condition 4: the exposure must not impact the
confounders of the mediator/outcome path), the same holds true for all
sequential time-varying treatments. However, unlike conventional causal
mediation analysis, it might be necessary to consider the sequence of
treatment regimes over an indefinitely long period.

Chronologically organised causal diagrams are useful for highlighting
problems with traditional multi-level regression analysis and structural
equation modelling.

For example, we might be interested in whether belief in Big Gods
affects social complexity. Consider estimating a fixed treatment regime
first. Suppose we have a well-defined concept of Big Gods and social
complexity as well as excellent measurements for both over time. In that
case, we might want to assess the effects of beliefs in Big Gods on
social complexity, say, two centuries after the beliefs were introduced.

The fixed treatment strategies are: ``always believe in Big Gods''
versus ``never believe in Big Gods'' on the level of social complexity.
Refer to Figure~\ref{fig-dag-9}. Here, \(A_{tx}\) represents the
cultural belief in Big Gods at time \(tx\), and \(Y_{tx}\) is the
outcome, social complexity, at time \(x\). Imagine that economic trade,
denoted as \(L_{tx}\), is a time-varying confounder. Suppose its effect
changes over time, which in turns affects the factors that influence
economic trade. To complete our causal diagram, we might include an
unmeasured confounder \(U\), such as oral traditions, which could
influence both the belief in Big Gods and social complexity.

Consider a scenario where we can reasonably infer that the level of
economic trade at time \(0\), represented as \(L_{t0}\), impacts beliefs
in ``Big Gods'' at time \(1\), denoted as \(A_{t1}\). In this case, we
would draw an arrow from \(L_{t0}\) to \(A_{t1}\). Conversely, if we
assume that belief in ``Big Gods,'' \(A_{t1}\), influences the future
level of economic trade, \(L_{t2}\), then an arrow should be added from
\(A_{t1}\) to \(L_{t2}\). This causal diagram illustrates a feedback
process between the time-varying exposure \(A\) and the time-varying
confounder \(L\). Figure~\ref{fig-dag-9}. displays exposure-confounder
feedback. In practical settings, the diagram could contain more arrows.
However, the intention here is to use the minimal number of arrows
needed to demonstrate the issue of exposure-confounder feedback. As a
guideline, we should avoid overcomplicating our causal diagrams and aim
to include only the essential details necessary for assessing the
identification problem.

What would happen if we were to condition on the time-varying confounder
\(L_{t3}\)? Two things would occur. First, we would block all the
backdoor paths between the exposure \(A_{t2}\) and the outcome. We need
to block those paths to eliminate confounding. Therefore, conditioning
on the time-varying confounding is essential. However, paths that were
previously blocked would close. For example, the path
\(A_{t1}, L_{t2}, U, Y_{t4}\), that was previously closed would be
opened because the time-varying confounder is the common effect of
\(A_{t1}\) and \(U\). Conditioning, then, opens the path
\(A_{t1}, L_{t2}, U, Y_{t4}\). Therefore we must avoid conditioning on
the time-varying confounder. It would seem then that if we were to
condition on a confounder that is affected by the prior exposure, we are
``damned if we do'' and ``dammed if we do not.''

\begin{figure}

{\centering \includegraphics[width=1\textwidth,height=\textheight]{causal-dags_files/figure-pdf/fig-dag-9-1.pdf}

}

\caption{\label{fig-dag-9}Exposure confounder feedback is a problem for
time-series models. If we do not condition on L\_t2, a backdoor path is
open from A\_t3 to Y\_t4. However, if conditioning on L\_t2 introduces
collider bias, opening a path, coloured in red, between A\_t2 and Y\_t4.
Here, we may not use conventional methods to estimate the effects of
multiple exposures. Instead, at best, we may obtain controlled effects
using G-methods. Multi-level models will not eliminate bias (!).
However, outside of epidemiology, G-methods are presently rarely used.}

\end{figure}

A similar problem arises when a time-varying exposure and time-varying
confounder share a common cause. This problem arises even without the
exposure affecting the confounder. The problem is presented in
Figure~\ref{fig-dag-time-vary-common-cause-A1-l1}.

\begin{figure}

{\centering \includegraphics[width=1\textwidth,height=\textheight]{causal-dags_files/figure-pdf/fig-dag-time-vary-common-cause-A1-l1-1.pdf}

}

\caption{\label{fig-dag-time-vary-common-cause-A1-l1}Exposure confounder
feedback is a problem for time-series models. Here, the problem arises
from an unmeasured variable (U\_2) that affects both the exposure A at
time 1 and the cofounder L at time 2. The red paths show the open
backdoor path when we condition on the L at time 2. Again, we cannot
infer causal effects in such scenarios by using regression-based
methods. In this setting, to address causal questions, we require
G-methods.}

\end{figure}

The potential for confounding increases when the exposure \(A_{t1}\)
affects the outcome \(Y_{t4}\). For example, since \(L_{t2}\) is on the
path from \(A_{t1}\) to \(Y_{t4}\), conditioning on \(L_{t2}\) partially
blocks the relation between the exposure and the outcome, triggering
collider stratification bias and mediator bias. However, to close the
open backdoor path from \(L_{t2}\) to \(Y_{t4}\), it becomes necessary
to condition on \(L_{t2}\). Paradoxically, we have just stated that
conditioning should be avoided! This broader dilemma of
exposure-confounder feedback is thoroughly explored in
(\citeproc{ref-hernuxe1n2023}{Hernán and Monge 2023}). Treatment
confounder feedback is particularly challenging for evolutionary human
science, yet its handling is beyond the capabilities of conventional
regression-based methods, including multi-level models
(\citeproc{ref-hernuxe1n2006}{Hernán and Robins 2006};
\citeproc{ref-robins1986}{Robins 1986}; \citeproc{ref-robins1999}{Robins
\emph{et al.} 1999}). As mentioned previously, G-methods encompass
models appropriate for investigating the causal effects of both
time-fixed and time-varying exposures
(\citeproc{ref-chatton2020}{Chatton \emph{et al.} 2020};
\citeproc{ref-hernuxe1n2006}{Hernán and Robins 2006};
\citeproc{ref-naimi2017}{Naimi \emph{et al.} 2017}). Despite significant
recent advancements in the health sciences
(\citeproc{ref-breskin2021}{Breskin \emph{et al.} 2021};
\citeproc{ref-duxedaz2021}{Díaz \emph{et al.} 2021};
\citeproc{ref-williams2021}{Williams and Díaz 2021}), these methods have
not been widely embraced in the field of human evolutionary sciences
\footnote{It is worth noting that the identification of controlled
  effect estimates can be enhanced by graphical methods such as ``Single
  World Intervention Graphs'' (SWIGs), which represent counterfactual
  outcomes in the diagrams. However, SWIGs are more accurately
  considered templates rather than causal diagrams in their general
  form. The use of SWIGs extends beyond the scope of this tutorial. For
  more information, see Richardson and Robins
  (\citeproc{ref-richardson2013}{2013}).}

\subsubsection{Summary}\label{summary}

To consistently estimate causal effects, we must contrast the world as
it has been with the world as it might have been. For many questions in
evolutionary human science, we have seen that confounder-treatment
feedback leads to intractable causal identification problems. We have
also seen that causal diagrams are helpful in clarifying these problems.
Many self-inflicted injuries, such as mediator bias and
post-stratification bias, could be avoided if confounders were measured
prior to the exposures. Chronologically ordered causal diagrams aim to
make this basis transparent. They function as circuit-breakers that may
protect us from blowing up our causal inferences. More constructively,
temporal order in the graph focusses attention on imperatives for data
collection, offering guidance and hope.

\subsection{Conclusions}\label{conclusions}

Chronologically ordered causal diagrams provide significant enrichment
to causal inference endeavours. Their utility is not limited to just
modelling; they serve as valuable guides for data collection. When used
judiciously, within the frameworks of counterfactual data science that
support causal inference, causal diagrams can substantially enhance the
pursuit of accurate and robust causal understanding.

\subsubsection{Pitfalls}\label{pitfalls}

\begin{enumerate}
\def\labelenumi{\arabic{enumi}.}
\item
  Misunderstanding the role of causal diagrams within the framework of
  counter-factual data science.
\item
  The causal diagram contains variables without time indices. This
  omission may suggest that the researcher has not adequately considered
  the timing of events.
\item
  The graph has excessive nodes. No effort has been made to simplify the
  model by retaining only those nodes and edges essential for clarifying
  the identification problem.
\item
  The study is an experiment, but arrows are leading into the
  manipulation, revealing confusion.
\item
  Bias is incorrectly described. The exposure and outcome are
  d-separated, yet bias is claimed. This indicates a misunderstanding;
  the bias probably relates to generalisability or transportability, not
  to confounding.
\item
  Overlooking the representation of selection bias on the graph,
  particularly post-exposure selection bias from attrition or
  missingness.
\item
  Neglecting to use causal diagrams during the design phase of research
  before data collection.
\item
  Ignoring structural assumptions in classical measurement theory, such
  as in latent factor models, and blindly using construct measures
  derived from factor analysis.
\item
  Trying to represent interactions and non-linear dynamics on a causal
  diagram, which can lead to confusion about their purposes.
\item
  Failing to realise that structural equation models are not structural
  models. They are tools for statistical analysis, better termed as
  ``correlational equation models.'' Coefficients from these models
  often lack causal interpretations.
\item
  Neglecting the fact that conventional models such as multi-level (or
  mixed effects) models are unsuitable when treatment-confounder
  feedback is present. Illustrating treatment-confounder feedback on a
  graph underscores this point.\footnote{G-methods are appropriate for
    causal estimation in dynamic longitudinal settings. Their
    effectiveness notwithstanding, many evolutionary human scientists
    have not adopted them.{[}\^{}g-methods-cites{]}}
\end{enumerate}

For good introductions see: Hernán and Monge
(\citeproc{ref-hernuxe1n2023}{2023}) Díaz \emph{et al.}
(\citeproc{ref-duxedaz2021}{2021}) VanderWeele
(\citeproc{ref-vanderweele2015}{2015}) Hoffman \emph{et al.}
(\citeproc{ref-hoffman2022}{2022}) Hoffman \emph{et al.}
(\citeproc{ref-hoffman2023}{2023}) Chatton \emph{et al.}
(\citeproc{ref-chatton2020}{2020}) Shiba and Kawahara
(\citeproc{ref-shiba2021}{2021}) Sjölander
(\citeproc{ref-sjuxf6lander2016}{2016})
(\citeproc{ref-breskin2020}{\textbf{breskin2020?}}) VanderWeele
(\citeproc{ref-vanderweele2009a}{2009b}) Vansteelandt \emph{et al.}
(\citeproc{ref-vansteelandt2012}{2012}) Shi \emph{et al.}
(\citeproc{ref-shi2021}{2021}).)

\subsubsection{Concluding remarks}\label{concluding-remarks}

In causal analysis, the passage of time is not just another variable but
the stage on which the entire causal play unfolds. Time-ordered causal
diagrams articulate this temporal structure, revealing the necessity for
collecting time-series data in our quest to answer our causal questions.

This need places new demands on our research designs, funding
mechanisms, and the very rhythm of scientific investigation. Rather than
continuing in the high-throughput, assembly-line model of research,
where rapid publication may sometimes come at the expense of depth and
precision, we must pivot towards an approach that nurtures the careful
and extended collection of data over time.

The pace of scientific progress in the human sciences of causal
inference hinges on this transformation. Our challenge is not merely
methodological but institutional, requiring a shift in our scientific
culture towards one that values the slow but essential work of building
rich, time-resolved data sets.

\newpage{}

\subsection{Funding}\label{funding}

This work is supported by a grant from the Templeton Religion Trust
(TRT0418). JB received support from the Max Planck Institute for the
Science of Human History. The funders had no role in preparing the
manuscript or the decision to publish it.

\subsection{References}\label{references}

\phantomsection\label{refs}
\setlength{\cslentryspacing}{0em}
\begin{CSLReferences}
\bibitem[\citeproctext]{ref-athey2019}
Athey, S, Tibshirani, J, and Wager, S (2019) Generalized random forests.
\emph{The Annals of Statistics}, \textbf{47}(2), 1148--1178.
doi:\href{https://doi.org/10.1214/18-AOS1709}{10.1214/18-AOS1709}.

\bibitem[\citeproctext]{ref-athey2021}
Athey, S, and Wager, S (2021) Policy Learning With Observational Data.
\emph{Econometrica}, \textbf{89}(1), 133--161.
doi:\href{https://doi.org/10.3982/ECTA15732}{10.3982/ECTA15732}.

\bibitem[\citeproctext]{ref-bareinboim2022a}
Bareinboim, E, Tian, J, and Pearl, J (2022) Recovering from selection
bias in causal and statistical inference. In, 1st edn, Vol. 36, New
York, NY, USA: Association for Computing Machinery, 433450. Retrieved
from \url{https://doi.org/10.1145/3501714.3501740}

\bibitem[\citeproctext]{ref-barrett2021}
Barrett, M (2021) \emph{Ggdag: Analyze and create elegant directed
acyclic graphs}. Retrieved from
\url{https://CRAN.R-project.org/package=ggdag}

\bibitem[\citeproctext]{ref-basten2013}
Basten, C, and Betz, F (2013) Beyond work ethic: Religion, individual,
and political preferences. \emph{American Economic Journal: Economic
Policy}, \textbf{5}(3), 67--91.
doi:\href{https://doi.org/10.1257/pol.5.3.67}{10.1257/pol.5.3.67}.

\bibitem[\citeproctext]{ref-becker2016}
Becker, SO, Pfaff, S, and Rubin, J (2016) Causes and consequences of the
protestant reformation. \emph{Explorations in Economic History},
\textbf{62}, 125.

\bibitem[\citeproctext]{ref-beheim2021}
Beheim, B, Atkinson, QD, Bulbulia, J, \ldots{} Willard, AK (2021)
Treatment of missing data determined conclusions regarding moralizing
gods. \emph{Nature}, \textbf{595}(7866), E29--E34.
doi:\href{https://doi.org/10.1038/s41586-021-03655-4}{10.1038/s41586-021-03655-4}.

\bibitem[\citeproctext]{ref-breskin2021}
Breskin, A, Edmonds, A, Cole, SR, \ldots{} Adimora, AA (2021)
G-computation for policy-relevant effects of interventions on
time-to-event outcomes. \emph{International Journal of Epidemiology},
\textbf{49}(6), 2021--2029.
doi:\href{https://doi.org/10.1093/ije/dyaa156}{10.1093/ije/dyaa156}.

\bibitem[\citeproctext]{ref-bulbulia2022}
Bulbulia, JA (2022) A workflow for causal inference in cross-cultural
psychology. \emph{Religion, Brain \& Behavior}, \textbf{0}(0), 1--16.
doi:\href{https://doi.org/10.1080/2153599X.2022.2070245}{10.1080/2153599X.2022.2070245}.

\bibitem[\citeproctext]{ref-bulbulia2023}
Bulbulia, JA (2023) Causal diagrams (DAGS) for evolutionary human
science: A practical guide.

\bibitem[\citeproctext]{ref-bulbulia2023a}
Bulbulia, JA, Afzali, MU, Yogeeswaran, K, and Sibley, CG (2023)
Long-term causal effects of far-right terrorism in new zealand.
\emph{PNAS Nexus}, \textbf{2}(8), pgad242.

\bibitem[\citeproctext]{ref-bulbulia2021}
Bulbulia, J, Schjoedt, U, Shaver, JH, Sosis, R, and Wildman, WJ (2021)
Causal inference in regression: Advice to authors. \emph{Religion, Brain
\& Behavior}, \textbf{11}(4), 353360.

\bibitem[\citeproctext]{ref-calonico2022}
Calonico, S, Cattaneo, MD, Farrell, MH, and Titiunik, R (2022)
\emph{Rdrobust: Robust data-driven statistical inference in
regression-discontinuity designs}. Retrieved from
\url{https://CRAN.R-project.org/package=rdrobust}

\bibitem[\citeproctext]{ref-chatton2020}
Chatton, A, Le Borgne, F, Leyrat, C, \ldots{} Foucher, Y (2020)
G-computation, propensity score-based methods, and targeted maximum
likelihood estimator for causal inference with different covariates
sets: a comparative simulation study. \emph{Scientific Reports},
\textbf{10}(1), 9219.
doi:\href{https://doi.org/10.1038/s41598-020-65917-x}{10.1038/s41598-020-65917-x}.

\bibitem[\citeproctext]{ref-cinelli2022}
Cinelli, C, Forney, A, and Pearl, J (2022) A Crash Course in Good and
Bad Controls. \emph{Sociological Methods \& Research},
00491241221099552.
doi:\href{https://doi.org/10.1177/00491241221099552}{10.1177/00491241221099552}.

\bibitem[\citeproctext]{ref-cole2010}
Cole, SR, Platt, RW, Schisterman, EF, \ldots{} Poole, C (2010)
Illustrating bias due to conditioning on a collider. \emph{International
Journal of Epidemiology}, \textbf{39}(2), 417--420.
doi:\href{https://doi.org/10.1093/ije/dyp334}{10.1093/ije/dyp334}.

\bibitem[\citeproctext]{ref-collinson2007}
Collinson, P (2007) \emph{The reformation: A history}, Vol. 19, Modern
Library.

\bibitem[\citeproctext]{ref-cui2020}
Cui, Y, Kosorok, MR, Sverdrup, E, Wager, S, and Zhu, R (2020) Estimating
heterogeneous treatment effects with right-censored data via causal
survival forests. Retrieved from
\url{https://arxiv.org/abs/2001.09887v5}

\bibitem[\citeproctext]{ref-decoulanges1903}
De Coulanges, F (1903) \emph{La cité antique: Étude sur le culte, le
droit, les institutions de la grèce et de rome}, Hachette.

\bibitem[\citeproctext]{ref-duxedaz2021}
Díaz, I, Williams, N, Hoffman, KL, and Schenck, EJ (2021) Non-parametric
causal effects based on longitudinal modified treatment policies.
\emph{Journal of the American Statistical Association}.
doi:\href{https://doi.org/10.1080/01621459.2021.1955691}{10.1080/01621459.2021.1955691}.

\bibitem[\citeproctext]{ref-edwards2015}
Edwards, JK, Cole, SR, and Westreich, D (2015) All your data are always
missing: Incorporating bias due to measurement error into the potential
outcomes framework. \emph{International Journal of Epidemiology},
\textbf{44}(4), 14521459.

\bibitem[\citeproctext]{ref-foster2023}
Foster, DJ, and Syrgkanis, V (2023) Orthogonal statistical learning.
\emph{The Annals of Statistics}, \textbf{51}(3), 879--908.
doi:\href{https://doi.org/10.1214/23-AOS2258}{10.1214/23-AOS2258}.

\bibitem[\citeproctext]{ref-gawthrop1984}
Gawthrop, R, and Strauss, G (1984) Protestantism and literacy in early
modern germany. \emph{Past \& Present}, (104), 3155.

\bibitem[\citeproctext]{ref-geertz2013}
Geertz, AW, Atkinson, QD, Cohen, E, \ldots{} Wilson, DS (2013) The
cultural evolution of religion. In P. J. Richerson and M. Christiansen,
eds., Cambridge, MA: MIT press, 381404.

\bibitem[\citeproctext]{ref-greenland1999}
Greenland, S, Pearl, J, and Robins, JM (1999a) Causal diagrams for
epidemiologic research. \emph{Epidemiology (Cambridge, Mass.)},
\textbf{10}(1), 37--48.

\bibitem[\citeproctext]{ref-greenland1999c}
Greenland, S, Pearl, J, and Robins, JM (1999b) Causal diagrams for
epidemiologic research. \emph{Epidemiology (Cambridge, Mass.)},
\textbf{10}(1), 37--48.

\bibitem[\citeproctext]{ref-greifer2023}
Greifer, N, Worthington, S, Iacus, S, and King, G (2023) \emph{Clarify:
Simulation-based inference for regression models}. Retrieved from
\url{https://iqss.github.io/clarify/}

\bibitem[\citeproctext]{ref-hahn2020}
Hahn, PR, Murray, JS, and Carvalho, CM (2020) Bayesian regression tree
models for causal inference: Regularization, confounding, and
heterogeneous effects (with discussion). \emph{Bayesian Analysis},
\textbf{15}(3), 965--1056.
doi:\href{https://doi.org/10.1214/19-BA1195}{10.1214/19-BA1195}.

\bibitem[\citeproctext]{ref-hernan2023}
Hernan, MA, and Robins, JM (2023) \emph{Causal inference}, Taylor \&
Francis. Retrieved from
\url{https://books.google.co.nz/books?id=/_KnHIAAACAAJ}

\bibitem[\citeproctext]{ref-hernuxe1n2004}
Hernán, MA (2004) A definition of causal effect for epidemiological
research. \emph{Journal of Epidemiology \& Community Health},
\textbf{58}(4), 265--271.
doi:\href{https://doi.org/10.1136/jech.2002.006361}{10.1136/jech.2002.006361}.

\bibitem[\citeproctext]{ref-hernuxe1n2017}
Hernán, MA (2017) Invited commentary: Selection bias without colliders
\textbar{} american journal of epidemiology \textbar{} oxford academic.
\emph{American Journal of Epidemiology}, \textbf{185}(11), 10481050.
Retrieved from \url{https://doi.org/10.1093/aje/kwx077}

\bibitem[\citeproctext]{ref-hernuxe1n2008a}
Hernán, MA, Alonso, A, Logan, R, \ldots{} Robins, JM (2008)
Observational studies analyzed like randomized experiments: An
application to postmenopausal hormone therapy and coronary heart
disease. \emph{Epidemiology}, \textbf{19}(6), 766.
doi:\href{https://doi.org/10.1097/EDE.0b013e3181875e61}{10.1097/EDE.0b013e3181875e61}.

\bibitem[\citeproctext]{ref-hernuxe1n2009}
Hernán, MA, and Cole, SR (2009) Invited commentary: Causal diagrams and
measurement bias. \emph{American Journal of Epidemiology},
\textbf{170}(8), 959--962.
doi:\href{https://doi.org/10.1093/aje/kwp293}{10.1093/aje/kwp293}.

\bibitem[\citeproctext]{ref-hernuxe1n2004a}
Hernán, MA, Hernández-Díaz, S, and Robins, JM (2004) A structural
approach to selection bias. \emph{Epidemiology}, \textbf{15}(5),
615--625. Retrieved from \url{https://www.jstor.org/stable/20485961}

\bibitem[\citeproctext]{ref-hernuxe1n2023}
Hernán, MA, and Monge, S (2023) Selection bias due to conditioning on a
collider. \emph{BMJ}, \textbf{381}, p1135.
doi:\href{https://doi.org/10.1136/bmj.p1135}{10.1136/bmj.p1135}.

\bibitem[\citeproctext]{ref-hernuxe1n2006}
Hernán, MA, and Robins, JM (2006) Estimating causal effects from
epidemiological data. \emph{Journal of Epidemiology \& Community
Health}, \textbf{60}(7), 578--586.
doi:\href{https://doi.org/10.1136/jech.2004.029496}{10.1136/jech.2004.029496}.

\bibitem[\citeproctext]{ref-hernuxe1n2016}
Hernán, MA, Sauer, BC, Hernández-Díaz, S, Platt, R, and Shrier, I (2016)
Specifying a target trial prevents immortal time bias and other
self-inflicted injuries in observational analyses. \emph{Journal of
Clinical Epidemiology}, \textbf{79}, 7075.

\bibitem[\citeproctext]{ref-hernuxe1n2008}
Hernán, MA, and Taubman, SL (2008) Does obesity shorten life? The
importance of well-defined interventions to answer causal questions.
\emph{International Journal of Obesity (2005)}, \textbf{32 Suppl 3},
S8--14.
doi:\href{https://doi.org/10.1038/ijo.2008.82}{10.1038/ijo.2008.82}.

\bibitem[\citeproctext]{ref-hernuxe1n2022}
Hernán, MA, Wang, W, and Leaf, DE (2022a) Target trial emulation: A
framework for causal inference from observational data. \emph{JAMA},
\textbf{328}(24), 2446--2447.
doi:\href{https://doi.org/10.1001/jama.2022.21383}{10.1001/jama.2022.21383}.

\bibitem[\citeproctext]{ref-hernuxe1n2022a}
Hernán, MA, Wang, W, and Leaf, DE (2022b) Target trial emulation: A
framework for causal inference from observational data. \emph{JAMA},
\textbf{328}(24), 2446--2447.
doi:\href{https://doi.org/10.1001/jama.2022.21383}{10.1001/jama.2022.21383}.

\bibitem[\citeproctext]{ref-hoffman2023}
Hoffman, KL, Salazar-Barreto, D, Rudolph, KE, and Díaz, I (2023)
Introducing longitudinal modified treatment policies: A unified
framework for studying complex exposures.
doi:\href{https://doi.org/10.48550/arXiv.2304.09460}{10.48550/arXiv.2304.09460}.

\bibitem[\citeproctext]{ref-hoffman2022}
Hoffman, KL, Schenck, EJ, Satlin, MJ, \ldots{} Díaz, I (2022) Comparison
of a target trial emulation framework vs cox regression to estimate the
association of corticosteroids with COVID-19 mortality. \emph{JAMA
Network Open}, \textbf{5}(10), e2234425.
doi:\href{https://doi.org/10.1001/jamanetworkopen.2022.34425}{10.1001/jamanetworkopen.2022.34425}.

\bibitem[\citeproctext]{ref-hume1902}
Hume, D (1902) \emph{Enquiries Concerning the Human Understanding: And
Concerning the Principles of Morals}, Clarendon Press.

\bibitem[\citeproctext]{ref-kennedy2023}
Kennedy, EH (2023) Towards optimal doubly robust estimation of
heterogeneous causal effects. \emph{Electronic Journal of Statistics},
\textbf{17}(2), 3008--3049.
doi:\href{https://doi.org/10.1214/23-EJS2157}{10.1214/23-EJS2157}.

\bibitem[\citeproctext]{ref-kitagawa2018}
Kitagawa, T, and Tetenov, A (2018) Who should be treated? Empirical
welfare maximization methods for treatment choice. \emph{Econometrica},
\textbf{86}(2), 591--616. Retrieved from
\url{https://www.jstor.org/stable/44955978}

\bibitem[\citeproctext]{ref-kuxfcnzel2019}
Künzel, SR, Sekhon, JS, Bickel, PJ, and Yu, B (2019) Metalearners for
estimating heterogeneous treatment effects using machine learning.
\emph{Proceedings of the National Academy of Sciences},
\textbf{116}(10), 4156--4165.
doi:\href{https://doi.org/10.1073/pnas.1804597116}{10.1073/pnas.1804597116}.

\bibitem[\citeproctext]{ref-lash2020}
Lash, TL, Rothman, KJ, VanderWeele, TJ, and Haneuse, S (2020)
\emph{Modern epidemiology}, Wolters Kluwer. Retrieved from
\url{https://books.google.co.nz/books?id=SiTSnQEACAAJ}

\bibitem[\citeproctext]{ref-lewis1973}
Lewis, D (1973) Causation. \emph{The Journal of Philosophy},
\textbf{70}(17), 556--567.
doi:\href{https://doi.org/10.2307/2025310}{10.2307/2025310}.

\bibitem[\citeproctext]{ref-lu2022}
Lu, H, Cole, SR, Howe, CJ, and Westreich, D (2022) Toward a Clearer
Definition of Selection Bias When Estimating Causal Effects.
\emph{Epidemiology (Cambridge, Mass.)}, \textbf{33}(5), 699--706.
doi:\href{https://doi.org/10.1097/EDE.0000000000001516}{10.1097/EDE.0000000000001516}.

\bibitem[\citeproctext]{ref-mcelreath2020}
McElreath, R (2020) \emph{Statistical rethinking: A bayesian course with
examples in r and stan}, CRC press.

\bibitem[\citeproctext]{ref-muuxf1oz2012}
Muñoz, ID, and Laan, M van der (2012) Population intervention causal
effects based on stochastic interventions. \emph{Biometrics},
\textbf{68}(2), 541--549.
doi:\href{https://doi.org/10.1111/j.1541-0420.2011.01685.x}{10.1111/j.1541-0420.2011.01685.x}.

\bibitem[\citeproctext]{ref-murray2021a}
Murray, EJ, Marshall, BDL, and Buchanan, AL (2021) Emulating target
trials to improve causal inference from agent-based models.
\emph{American Journal of Epidemiology}, \textbf{190}(8), 1652--1658.
doi:\href{https://doi.org/10.1093/aje/kwab040}{10.1093/aje/kwab040}.

\bibitem[\citeproctext]{ref-naimi2017}
Naimi, AI, Cole, SR, and Kennedy, EH (2017) An introduction to g
methods. \emph{International Journal of Epidemiology}, \textbf{46}(2),
756--762.
doi:\href{https://doi.org/10.1093/ije/dyw323}{10.1093/ije/dyw323}.

\bibitem[\citeproctext]{ref-nalle1987}
Nalle, ST (1987) Inquisitors, priests, and the people during the
catholic reformation in spain. \emph{The Sixteenth Century Journal},
557587.

\bibitem[\citeproctext]{ref-neyman1923}
Neyman, JS (1923) On the application of probability theory to
agricultural experiments. Essay on principles. Section 9.(tlanslated and
edited by dm dabrowska and tp speed, statistical science (1990), 5,
465-480). \emph{Annals of Agricultural Sciences}, \textbf{10}, 151.

\bibitem[\citeproctext]{ref-nie2021}
Nie, X, and Wager, S (2021) Quasi-oracle estimation of heterogeneous
treatment effects. \emph{Biometrika}, \textbf{108}(2), 299--319.
doi:\href{https://doi.org/10.1093/biomet/asaa076}{10.1093/biomet/asaa076}.

\bibitem[\citeproctext]{ref-ogburn2021}
Ogburn, EL, and Shpitser, I (2021) Causal modelling: The two cultures.
\emph{Observational Studies}, \textbf{7}(1), 179--183.
doi:\href{https://doi.org/10.1353/obs.2021.0006}{10.1353/obs.2021.0006}.

\bibitem[\citeproctext]{ref-ogburn2022}
Ogburn, EL, Sofrygin, O, Díaz, I, and Laan, MJ van der (2022) Causal
inference for social network data. \emph{Journal of the American
Statistical Association}, \textbf{0}(0), 1--15.
doi:\href{https://doi.org/10.1080/01621459.2022.2131557}{10.1080/01621459.2022.2131557}.

\bibitem[\citeproctext]{ref-pearl2009}
Pearl, J (2009a) \emph{\href{https://doi.org/10.1214/09-SS057}{Causal
inference in statistics: An overview}}.

\bibitem[\citeproctext]{ref-pearl2009a}
Pearl, J (2009b) \emph{Causality}, Cambridge University Press.

\bibitem[\citeproctext]{ref-pearl1995}
Pearl, J, and Robins, JM (1995) Probabilistic evaluation of sequential
plans from causal models with hidden variables. In, Vol. 95, Citeseer,
444453.

\bibitem[\citeproctext]{ref-richardson2013}
Richardson, TS, and Robins, JM (2013) Single world intervention graphs:
A primer. In, Citeseer.

\bibitem[\citeproctext]{ref-robins1986}
Robins, J (1986) A new approach to causal inference in mortality studies
with a sustained exposure period{\textemdash}application to control of
the healthy worker survivor effect. \emph{Mathematical Modelling},
\textbf{7}(9), 1393--1512.
doi:\href{https://doi.org/10.1016/0270-0255(86)90088-6}{10.1016/0270-0255(86)90088-6}.

\bibitem[\citeproctext]{ref-robins1992}
Robins, JM, and Greenland, S (1992) Identifiability and exchangeability
for direct and indirect effects. \emph{Epidemiology}, \textbf{3}(2),
143155.

\bibitem[\citeproctext]{ref-robins1999}
Robins, JM, Greenland, S, and Hu, F-C (1999) Estimation of the causal
effect of a time-varying exposure on the marginal mean of a repeated
binary outcome. \emph{Journal of the American Statistical Association},
\textbf{94}(447), 687--700.
doi:\href{https://doi.org/10.1080/01621459.1999.10474168}{10.1080/01621459.1999.10474168}.

\bibitem[\citeproctext]{ref-rohrer2018}
Rohrer, JM (2018) Thinking clearly about correlations and causation:
Graphical causal models for observational data. \emph{Advances in
Methods and Practices in Psychological Science}, \textbf{1}(1), 2742.

\bibitem[\citeproctext]{ref-rubin1976}
Rubin, DB (1976) Inference and missing data. \emph{Biometrika},
\textbf{63}(3), 581--592.
doi:\href{https://doi.org/10.1093/biomet/63.3.581}{10.1093/biomet/63.3.581}.

\bibitem[\citeproctext]{ref-shi2021}
Shi, B, Choirat, C, Coull, BA, VanderWeele, TJ, and Valeri, L (2021)
CMAverse: A suite of functions for reproducible causal mediation
analyses. \emph{Epidemiology}, \textbf{32}(5), e20e22.

\bibitem[\citeproctext]{ref-shiba2021}
Shiba, K, and Kawahara, T (2021) Using propensity scores for causal
inference: Pitfalls and tips. \emph{Journal of Epidemiology},
\textbf{31}(8), 457463.

\bibitem[\citeproctext]{ref-sjuxf6lander2016}
Sjölander, A (2016) Regression standardization with the R package
stdReg. \emph{European Journal of Epidemiology}, \textbf{31}(6),
563--574.
doi:\href{https://doi.org/10.1007/s10654-016-0157-3}{10.1007/s10654-016-0157-3}.

\bibitem[\citeproctext]{ref-slingerland}
Slingerland, E, Atkinson, QD, Ember, CR, Sheehan, O, Muthukrishna, M,
and Gray, RD Coding culture: Challenges and recommendations for
comparative cultural databases. \emph{Evolutionary Human Sciences}.

\bibitem[\citeproctext]{ref-stuart2015}
Stuart, EA, Bradshaw, CP, and Leaf, PJ (2015) Assessing the
Generalizability of Randomized Trial Results to Target Populations.
\emph{Prevention Science}, \textbf{16}(3), 475--485.
doi:\href{https://doi.org/10.1007/s11121-014-0513-z}{10.1007/s11121-014-0513-z}.

\bibitem[\citeproctext]{ref-suzuki2020}
Suzuki, E, Shinozaki, T, and Yamamoto, E (2020a) Causal Diagrams:
Pitfalls and Tips. \emph{Journal of Epidemiology}, \textbf{30}(4),
153--162.
doi:\href{https://doi.org/10.2188/jea.JE20190192}{10.2188/jea.JE20190192}.

\bibitem[\citeproctext]{ref-suzuki2020a}
Suzuki, E, Shinozaki, T, and Yamamoto, E (2020b) Causal Diagrams:
Pitfalls and Tips. \emph{Journal of Epidemiology}, \textbf{30}(4),
153--162.
doi:\href{https://doi.org/10.2188/jea.JE20190192}{10.2188/jea.JE20190192}.

\bibitem[\citeproctext]{ref-swanson1967}
Swanson, GE (1967) Religion and regime: A sociological account of the
reformation.

\bibitem[\citeproctext]{ref-swanson1971}
Swanson, GE (1971) Interpreting the reformation. \emph{The Journal of
Interdisciplinary History}, \textbf{1}(3), 419446. Retrieved from
\url{http://www.jstor.org/stable/202620}

\bibitem[\citeproctext]{ref-tchetgen2012}
Tchetgen, EJT, and VanderWeele, TJ (2012) On causal inference in the
presence of interference. \emph{Statistical Methods in Medical
Research}, \textbf{21}(1), 5575.

\bibitem[\citeproctext]{ref-textor2011}
Textor, J, Hardt, J, and Knüppel, S (2011) DAGitty: A graphical tool for
analyzing causal diagrams. \emph{Epidemiology}, \textbf{22}(5), 745.

\bibitem[\citeproctext]{ref-tripepi2007}
Tripepi, G, Jager, KJ, Dekker, FW, Wanner, C, and Zoccali, C (2007)
Measures of effect: Relative risks, odds ratios, risk difference, and
{`}number needed to treat{'}. \emph{Kidney International},
\textbf{72}(7), 789--791.
doi:\href{https://doi.org/10.1038/sj.ki.5002432}{10.1038/sj.ki.5002432}.

\bibitem[\citeproctext]{ref-vanderlaan2011}
Van Der Laan, MJ, and Rose, S (2011) \emph{Targeted Learning: Causal
Inference for Observational and Experimental Data}, New York, NY:
Springer. Retrieved from
\url{https://link.springer.com/10.1007/978-1-4419-9782-1}

\bibitem[\citeproctext]{ref-vanderlaan2018}
Van Der Laan, MJ, and Rose, S (2018) \emph{Targeted Learning in Data
Science: Causal Inference for Complex Longitudinal Studies}, Cham:
Springer International Publishing. Retrieved from
\url{http://link.springer.com/10.1007/978-3-319-65304-4}

\bibitem[\citeproctext]{ref-vanderweele2015}
VanderWeele, T (2015) \emph{Explanation in causal inference: Methods for
mediation and interaction}, Oxford University Press.

\bibitem[\citeproctext]{ref-vanderweele2009}
VanderWeele, TJ (2009a) Concerning the consistency assumption in causal
inference. \emph{Epidemiology}, \textbf{20}(6), 880.
doi:\href{https://doi.org/10.1097/EDE.0b013e3181bd5638}{10.1097/EDE.0b013e3181bd5638}.

\bibitem[\citeproctext]{ref-vanderweele2009a}
VanderWeele, TJ (2009b) Marginal structural models for the estimation of
direct and indirect effects. \emph{Epidemiology}, 1826.

\bibitem[\citeproctext]{ref-vanderweele2018}
VanderWeele, TJ (2018) On well-defined hypothetical interventions in the
potential outcomes framework. \emph{Epidemiology}, \textbf{29}(4), e24.
doi:\href{https://doi.org/10.1097/EDE.0000000000000823}{10.1097/EDE.0000000000000823}.

\bibitem[\citeproctext]{ref-vanderweele2019}
VanderWeele, TJ (2019) Principles of confounder selection.
\emph{European Journal of Epidemiology}, \textbf{34}(3), 211219.

\bibitem[\citeproctext]{ref-vanderweele2022}
VanderWeele, TJ (2022b) Constructed measures and causal inference:
Towards a new model of measurement for psychosocial constructs.
\emph{Epidemiology}, \textbf{33}(1), 141.
doi:\href{https://doi.org/10.1097/EDE.0000000000001434}{10.1097/EDE.0000000000001434}.

\bibitem[\citeproctext]{ref-vanderweele2022a}
VanderWeele, TJ (2022a) Constructed measures and causal inference:
Towards a new model of measurement for psychosocial constructs.
\emph{Epidemiology}, \textbf{33}(1), 141.
doi:\href{https://doi.org/10.1097/EDE.0000000000001434}{10.1097/EDE.0000000000001434}.

\bibitem[\citeproctext]{ref-vanderweele2013}
VanderWeele, TJ, and Hernan, MA (2013) Causal inference under multiple
versions of treatment. \emph{Journal of Causal Inference},
\textbf{1}(1), 120.

\bibitem[\citeproctext]{ref-vanderweele2012a}
VanderWeele, TJ, and Hernán, MA (2012) Results on differential and
dependent measurement error of the exposure and the outcome using signed
directed acyclic graphs. \emph{American Journal of Epidemiology},
\textbf{175}(12), 1303--1310.
doi:\href{https://doi.org/10.1093/aje/kwr458}{10.1093/aje/kwr458}.

\bibitem[\citeproctext]{ref-vanderweele2014}
VanderWeele, TJ, and Knol, MJ (2014) A tutorial on interaction.
\emph{Epidemiologic Methods}, \textbf{3}(1), 3372.

\bibitem[\citeproctext]{ref-vanderweele2020}
VanderWeele, TJ, Mathur, MB, and Chen, Y (2020) Outcome-wide
longitudinal designs for causal inference: A new template for empirical
studies. \emph{Statistical Science}, \textbf{35}(3), 437466.

\bibitem[\citeproctext]{ref-vanderweele2007}
VanderWeele, TJ, and Robins, JM (2007) Four types of effect
modification: a classification based on directed acyclic graphs.
\emph{Epidemiology (Cambridge, Mass.)}, \textbf{18}(5), 561--568.
doi:\href{https://doi.org/10.1097/EDE.0b013e318127181b}{10.1097/EDE.0b013e318127181b}.

\bibitem[\citeproctext]{ref-vanderweele2022b}
VanderWeele, TJ, and Vansteelandt, S (2022) A statistical test to reject
the structural interpretation of a latent factor model. \emph{Journal of
the Royal Statistical Society Series B: Statistical Methodology},
\textbf{84}(5), 20322054.

\bibitem[\citeproctext]{ref-vansteelandt2012}
Vansteelandt, S, Bekaert, M, and Lange, T (2012) Imputation strategies
for the estimation of natural direct and indirect effects.
\emph{Epidemiologic Methods}, \textbf{1}(1), 131158.

\bibitem[\citeproctext]{ref-vansteelandt2022}
Vansteelandt, S, and Dukes, O (2022) Assumption-lean inference for
generalised linear model parameters. \emph{Journal of the Royal
Statistical Society Series B: Statistical Methodology}, \textbf{84}(3),
657685.

\bibitem[\citeproctext]{ref-wager2018}
Wager, S, and Athey, S (2018) Estimation and inference of heterogeneous
treatment effects using random forests. \emph{Journal of the American
Statistical Association}, \textbf{113}(523), 1228--1242.
doi:\href{https://doi.org/10.1080/01621459.2017.1319839}{10.1080/01621459.2017.1319839}.

\bibitem[\citeproctext]{ref-watts2015}
Watts, J, and Gray, R (2015) \emph{Broad supernatural punishment but not
moralising high gods precede the evolution of political complexity in
austronesia}, Victoria University Empirical Philosophy Workshop.
Wellington, New Zealand.

\bibitem[\citeproctext]{ref-watts2018}
Watts, J, Sheehan, O, Bulbulia, Joseph A, Gray, RD, and Atkinson, QD
(2018) Christianity spread faster in small, politically structured
societies. \emph{Nature Human Behaviour}, \textbf{2}(8), 559564.
doi:\href{https://doi.org/gdvnjn}{gdvnjn}.

\bibitem[\citeproctext]{ref-weber1905}
Weber, M (1905) \emph{The protestant ethic and the spirit of capitalism:
And other writings}, Penguin.

\bibitem[\citeproctext]{ref-weber1993}
Weber, M (1993) \emph{The sociology of religion}, Beacon Press.

\bibitem[\citeproctext]{ref-westreich2010}
Westreich, D, and Cole, SR (2010) Invited commentary: positivity in
practice. \emph{American Journal of Epidemiology}, \textbf{171}(6).
doi:\href{https://doi.org/10.1093/aje/kwp436}{10.1093/aje/kwp436}.

\bibitem[\citeproctext]{ref-westreich2015}
Westreich, D, Edwards, JK, Cole, SR, Platt, RW, Mumford, SL, and
Schisterman, EF (2015) Imputation approaches for potential outcomes in
causal inference. \emph{International Journal of Epidemiology},
\textbf{44}(5), 17311737.

\bibitem[\citeproctext]{ref-wheatley1971}
Wheatley, P (1971) \emph{The pivot of the four quarters : A preliminary
enquiry into the origins and character of the ancient chinese city},
Edinburgh University Press. Retrieved from
\url{https://cir.nii.ac.jp/crid/1130000795717727104}

\bibitem[\citeproctext]{ref-whitehouse2023}
Whitehouse, H, François, P, Savage, PE, \ldots{} Haar, B ter (2023)
Testing the big gods hypothesis with global historical data: A review
and {``}retake{''}. \emph{Religion, Brain \& Behavior}, \textbf{13}(2),
124166.

\bibitem[\citeproctext]{ref-williams2021}
Williams, NT, and Díaz, I (2021) \emph{Lmtp: Non-parametric causal
effects of feasible interventions based on modified treatment policies}.
doi:\href{https://doi.org/10.5281/zenodo.3874931}{10.5281/zenodo.3874931}.

\end{CSLReferences}

\newpage{}

\subsection{Appendix 1: The difficulty of satisfying the three
fundamental assumptions of causal inference when asking causal questions
of
history}\label{appendix-1-the-difficulty-of-satisfying-the-three-fundamental-assumptions-of-causal-inference-when-asking-causal-questions-of-history}

Consider the Protestant Reformation of the 16th century, which initiated
religious change throughout much of Europe. Historians have argued that
Protestantism caused social, cultural, and economic changes in those
societies where it took hold (see: (\citeproc{ref-basten2013}{Basten and
Betz 2013}; \citeproc{ref-swanson1967}{Swanson 1967};
\citeproc{ref-swanson1971}{Swanson 1971}; \citeproc{ref-weber1905}{Weber
1905}, \citeproc{ref-weber1993}{1993}), for an overview see:
(\citeproc{ref-becker2016}{Becker \emph{et al.} 2016})).

Suppose we are interested in estimating the ``Average Treatment Effect''
of the Protestant Reformation. Let \(A = a^*\) denote the adoption of
Protestantism. We compare this effect with that of remaining Catholic,
represented as \(A = a\). We assume that both the concepts of ``adopting
Protestantism'' and of ``economic development'' are well-defined
(e.g.~GDP +1 century after a country has a Protestant majority
contrasted with remaining Catholic). The causal effect for any
individual country is \(Y_i(a^*) - Y_i(a)\). Although we cannot identify
this effect, if the basic assumptions of causal inference are met, we
can estimate the average or marginal effect as

\[
\frac{1}{n} \sum_i^{n} \left[ Y_i(a^*) - Y_i(a) \right]
\]

which, conditioning the confounding effects of \(L\) gives us

\[ATE_{\textnormal{economic~development}} = \mathbb{E}[Y(\textnormal{Became~Protestant}|L) - Y(\textnormal{Remained~Catholic}|L)]\]

When asking causal questions about the economic effect of adopting
Protestantism versus remaining Catholic, there are indeed several
challenges that arise in relation to the three fundamental assumptions
required for causal inference.

\textbf{Causal Consistency}: requires the outcome under each level of
exposure is well-defined. In this context, defining what ``adopting
Protestantism'' and ``remaining Catholic'' mean may present challenges.
The practices and beliefs associated with each religion might vary
significantly across countries and time periods, and it may be difficult
to create a consistent, well-defined exposure. Furthermore, the outcome
- economic development - may also be challenging to measure consistently
across different countries and time periods.

There is undoubtedly considerable heterogeneity in the ``Protestant
exposure.'' In England, Protestantism was closely tied to the monarchy
(\citeproc{ref-collinson2007}{Collinson 2007}). In Germany, Martin
Luther's teachings emphasised individual faith in scripture, which, it
has been claimed, supported economic development by promoting literacy
(\citeproc{ref-gawthrop1984}{Gawthrop and Strauss 1984}). In England,
King Henry VIII abolished Catholicism
(\citeproc{ref-collinson2007}{Collinson 2007}). The Reformation, then,
occurred differently in different places. The exposure needs to be
better-defined.

There is also ample scope for interference: 16th century societies were
interconnected through trade, diplomacy, and warfare. Thus, the
religious decisions of one society were unlikely to have been
independent from those of other societies.

\textbf{Exchangeability}: requires that given the confounders, the
potential outcomes are independent of the treatment assignment. It might
be difficult to account for all possible confounders in this context.
For example, historical, political, social, and geographical factors
could influence both a country's religious affiliations and its economic
development. If these factors are not properly controlled, it could lead
to confounding bias.

\textbf{Positivity}: requires that there is a non-zero probability of
every level of exposure for every strata of confounders. If we consider
various confounding factors such as geographical location, historical
events, or political circumstances, some countries might only ever have
the possibility of either remaining Catholic or becoming Protestant, but
not both. For example, it is unclear under which conditions 16th century
Spain could have been randomly assigned to Protestantism
(\citeproc{ref-nalle1987}{Nalle 1987}).

Perhaps a more credible measure of effect in the region of our interests
is the Average Treatment Effect in the Treated (ATT) expressed

\[ATT_{\textnormal{economic~development}} = \mathbb{E}[(Y(a*)- Y(a))|A = a*,L]\]

Here, the ATT defines the expected difference in economic success for
cultures that became Protestant compared with the expected economic
success if those cultures had not become Protestant, conditional on
measured confounders \(L\), among the exposed (\(A = a^*\)). To estimate
this contrast, our models would need to match Protestant cultures with
comparable Catholic cultures effectively. By estimating the ATT, we
would avoid the assumption of non-deterministic positivity for the
untreated. However, whether matching is conceptually plausible remains
debatable. Ostensibly, it would seem that assigning a religion to a
culture a religion is not as easy as administering a pill
(\citeproc{ref-watts2018}{Watts \emph{et al.} 2018}).

\subsection{Appendix 3: Review of VanderWeele's theory of causal
inference under multiple versions of
treatment}\label{appendix-3-review-of-vanderweeles-theory-of-causal-inference-under-multiple-versions-of-treatment}

We denote an average causal effect as the change in the expected
potential outcomes when all units receive one level of treatment
compared to another.

Let \(\delta\) denote the causal estimand on the difference scale
\((\mathbb{E}[Y^1 - Y^0])\). The causal effect identification can be
expressed as:

\[ \delta = \sum_l \left( \mathbb{E}[Y|A=a,l] - \mathbb{E}[Y|A=a^*,l] \right) P(l)\]

The theory of causal inference with multiple treatment versions provides
a conceptual framework for causal inference in observational studies.
Suppose we can assume that for each treatment version, the outcome under
that version equals the observed outcome when that version is
administered, conditional on baseline covariates and satisfaction of
other assumptions. In that case, we can consistently estimate causal
contrasts, even when treatments vary.

This approach interprets treatment indicator \(A\) as multiple actual
treatment versions \(K\). Furthermore, if we can assume conditional
independence, meaning there is no confounding for the effect of \(K\) on
\(Y\) given \(L\), we have: \(Y(k)\coprod A|K,L\).

This condition implies that, given \(L\), \(A\) adds no additional
information about \(Y\) after accounting for \(K\) and \(L\). If
\(Y = Y(k)\) for \(K = k\) and \(Y(k)\) is independent of \(K\),
conditional on \(L\), we can interpret \(A\) as a simplified indicator
of \(K\) (\citeproc{ref-vanderweele2013}{VanderWeele and Hernan 2013}).
This scenario is depicted in
Figure~\ref{fig-dag-multiple-version-treatment-dag}.

With the necessary assumptions in place, Vandeweele shows that can
derive consistent causal effects by proving:

\[\delta = \sum_{k,l} \left( \mathbb{E}[Y(k)|l] P(k|a,l) P(l) - \mathbb{E}[Y(k)|l] P(k|a^*,l) P(l) \right) \]

This setup is akin to a randomised trial where individuals, stratified
by covariate \(L\), are assigned a treatment version \(K\). This
assignment comes from the distribution of \(K\) for the
\((A = 1, L = l)\) subset. The control group receives a randomly
assigned \(K\) version from the \((A = 0, L = l)\) distribution.

\begin{figure}

{\centering \includegraphics[width=1\textwidth,height=\textheight]{causal-dags_files/figure-pdf/fig-dag-multiple-version-treatment-dag-1.pdf}

}

\caption{\label{fig-dag-multiple-version-treatment-dag}Causal inference
under multiple versions of treatment. Here, (A) may be regarded as a
coarseneed indicator of (K)}

\end{figure}

The theory of causal inference under multiple versions of treatment
reveal that consistent causal effect estimates are possible even when
treatments exhibit variability
(\citeproc{ref-vanderweele2013}{VanderWeele and Hernan 2013}). In Part
5, I explored VanderWeele's application of this theory to latent factor
models, where the presumption of a single underlying reality for the
items that constitute constructs can be challenged. VandnerWeele shows
that we may nevertheless, under assumptions of exchangeability,
consistenty estimate causal effects using a logic that parrallels the
theory of causal inference under multiple versions of treatment
(\citeproc{ref-vanderweele2022}{VanderWeele 2022b}). I noted that the
possibility that directed or correlated error terms for the exposure and
outcome might nevertheless undermine inferences, and that such threats
may become more exaggerated with multiple items for our measures. I
noted that in place of general rules, researchers should be encouraged
to consider the problems of measurement in context.

\textless!--



\end{document}
