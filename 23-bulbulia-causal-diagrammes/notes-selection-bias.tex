% Options for packages loaded elsewhere
\PassOptionsToPackage{unicode}{hyperref}
\PassOptionsToPackage{hyphens}{url}
\PassOptionsToPackage{dvipsnames,svgnames,x11names}{xcolor}
%
\documentclass[
  singlecolumn,
  9pt]{article}

\usepackage{amsmath,amssymb}
\usepackage{iftex}
\ifPDFTeX
  \usepackage[T1]{fontenc}
  \usepackage[utf8]{inputenc}
  \usepackage{textcomp} % provide euro and other symbols
\else % if luatex or xetex
  \usepackage{unicode-math}
  \defaultfontfeatures{Scale=MatchLowercase}
  \defaultfontfeatures[\rmfamily]{Ligatures=TeX,Scale=1}
\fi
\usepackage[]{libertinus}
\ifPDFTeX\else  
    % xetex/luatex font selection
\fi
% Use upquote if available, for straight quotes in verbatim environments
\IfFileExists{upquote.sty}{\usepackage{upquote}}{}
\IfFileExists{microtype.sty}{% use microtype if available
  \usepackage[]{microtype}
  \UseMicrotypeSet[protrusion]{basicmath} % disable protrusion for tt fonts
}{}
\makeatletter
\@ifundefined{KOMAClassName}{% if non-KOMA class
  \IfFileExists{parskip.sty}{%
    \usepackage{parskip}
  }{% else
    \setlength{\parindent}{0pt}
    \setlength{\parskip}{6pt plus 2pt minus 1pt}}
}{% if KOMA class
  \KOMAoptions{parskip=half}}
\makeatother
\usepackage{xcolor}
\usepackage[top=30mm,bottom=30mm,left=20mm,heightrounded]{geometry}
\setlength{\emergencystretch}{3em} % prevent overfull lines
\setcounter{secnumdepth}{-\maxdimen} % remove section numbering
% Make \paragraph and \subparagraph free-standing
\ifx\paragraph\undefined\else
  \let\oldparagraph\paragraph
  \renewcommand{\paragraph}[1]{\oldparagraph{#1}\mbox{}}
\fi
\ifx\subparagraph\undefined\else
  \let\oldsubparagraph\subparagraph
  \renewcommand{\subparagraph}[1]{\oldsubparagraph{#1}\mbox{}}
\fi


\providecommand{\tightlist}{%
  \setlength{\itemsep}{0pt}\setlength{\parskip}{0pt}}\usepackage{longtable,booktabs,array}
\usepackage{calc} % for calculating minipage widths
% Correct order of tables after \paragraph or \subparagraph
\usepackage{etoolbox}
\makeatletter
\patchcmd\longtable{\par}{\if@noskipsec\mbox{}\fi\par}{}{}
\makeatother
% Allow footnotes in longtable head/foot
\IfFileExists{footnotehyper.sty}{\usepackage{footnotehyper}}{\usepackage{footnote}}
\makesavenoteenv{longtable}
\usepackage{graphicx}
\makeatletter
\def\maxwidth{\ifdim\Gin@nat@width>\linewidth\linewidth\else\Gin@nat@width\fi}
\def\maxheight{\ifdim\Gin@nat@height>\textheight\textheight\else\Gin@nat@height\fi}
\makeatother
% Scale images if necessary, so that they will not overflow the page
% margins by default, and it is still possible to overwrite the defaults
% using explicit options in \includegraphics[width, height, ...]{}
\setkeys{Gin}{width=\maxwidth,height=\maxheight,keepaspectratio}
% Set default figure placement to htbp
\makeatletter
\def\fps@figure{htbp}
\makeatother
% definitions for citeproc citations
\NewDocumentCommand\citeproctext{}{}
\NewDocumentCommand\citeproc{mm}{%
  \begingroup\def\citeproctext{#2}\cite{#1}\endgroup}
% avoid brackets around text for \cite:
\makeatletter
 \def\@biblabel#1{}
 \def\@cite#1#2{{#1\if@tempswa , #2\fi}}
\makeatother
\newlength{\cslhangindent}
\setlength{\cslhangindent}{1.5em}
\newlength{\csllabelwidth}
\setlength{\csllabelwidth}{3em}
\newlength{\cslentryspacing}
\setlength{\cslentryspacing}{0em}
\usepackage{enumitem}
\newlist{CSLReferences}{itemize}{1}
\setlist[CSLReferences]{label={},
  leftmargin=\cslhangindent,
  itemindent=-1\cslhangindent,
  parsep=\parskip,
  itemsep=\cslentryspacing}
\usepackage{calc}
\newcommand{\CSLBlock}[1]{#1\hfill\break}
\newcommand{\CSLLeftMargin}[1]{\parbox[t]{\csllabelwidth}{#1}}
\newcommand{\CSLRightInline}[1]{\parbox[t]{\linewidth - \csllabelwidth}{#1}\break}
\newcommand{\CSLIndent}[1]{\hspace{\cslhangindent}#1}

\usepackage{cancel}
\usepackage{xcolor}
\usepackage[noblocks]{authblk}
\renewcommand*{\Authsep}{, }
\renewcommand*{\Authand}{, }
\renewcommand*{\Authands}{, }
\renewcommand\Affilfont{\small}
\usepackage{cancel}
\makeatletter
\makeatother
\makeatletter
\makeatother
\makeatletter
\@ifpackageloaded{caption}{}{\usepackage{caption}}
\AtBeginDocument{%
\ifdefined\contentsname
  \renewcommand*\contentsname{Table of contents}
\else
  \newcommand\contentsname{Table of contents}
\fi
\ifdefined\listfigurename
  \renewcommand*\listfigurename{List of Figures}
\else
  \newcommand\listfigurename{List of Figures}
\fi
\ifdefined\listtablename
  \renewcommand*\listtablename{List of Tables}
\else
  \newcommand\listtablename{List of Tables}
\fi
\ifdefined\figurename
  \renewcommand*\figurename{Figure}
\else
  \newcommand\figurename{Figure}
\fi
\ifdefined\tablename
  \renewcommand*\tablename{Table}
\else
  \newcommand\tablename{Table}
\fi
}
\@ifpackageloaded{float}{}{\usepackage{float}}
\floatstyle{ruled}
\@ifundefined{c@chapter}{\newfloat{codelisting}{h}{lop}}{\newfloat{codelisting}{h}{lop}[chapter]}
\floatname{codelisting}{Listing}
\newcommand*\listoflistings{\listof{codelisting}{List of Listings}}
\makeatother
\makeatletter
\@ifpackageloaded{caption}{}{\usepackage{caption}}
\@ifpackageloaded{subcaption}{}{\usepackage{subcaption}}
\makeatother
\makeatletter
\makeatother
\ifLuaTeX
  \usepackage{selnolig}  % disable illegal ligatures
\fi
\IfFileExists{bookmark.sty}{\usepackage{bookmark}}{\usepackage{hyperref}}
\IfFileExists{xurl.sty}{\usepackage{xurl}}{} % add URL line breaks if available
\urlstyle{same} % disable monospaced font for URLs
\hypersetup{
  pdftitle={Selection Bias: A Brief Guide for the Perplexed},
  pdfauthor={Joseph A. Bulbulia},
  pdfkeywords={DAGS (Directed Acyclic Graphs), Causal
Inference, Confounding, History, Psychology, Panel},
  colorlinks=true,
  linkcolor={blue},
  filecolor={Maroon},
  citecolor={Blue},
  urlcolor={Blue},
  pdfcreator={LaTeX via pandoc}}

\title{Selection Bias: A Brief Guide for the Perplexed}


  \author{Joseph A. Bulbulia}
            \affil{%
                  Victoria University of Wellington, New Zealand
              }
      
\date{2023-11-30}
\begin{document}
\maketitle
\begin{abstract}
Selection bias arises when the sample used in a study does not
accurately represent the target population, skewing causal inferences. I
briefly review key concept of selection bias as it is used in the
potential outcomes framework in causal data science, and explain their
importance for evolutionary human scientists. Unlike confounding bias,
which results from unaccounted variables that influence both the
treatment and the outcome, selection bias stems from the process of
choosing the units in a study sample, or from differential attrition of
these units. I explain how these distinctions are useful for applied
researchers in the evolutionary human scientists, focussing on research
design, data collection, and data analysis in longitudinal studies.
\end{abstract}
\subsubsection{Introduction to selection
bias}\label{introduction-to-selection-bias}

Terminology in causal data science is a dog's breakfast. Researchers are
confronted with a confusing array of meanings and distinctions. This is
particularly apparent in the concept of `selection bias.' In this
research note, I examine selection bias as it is considered in the
potential outcomes framework of causal data science. Understanding and
addressing selection bias so conceived is essential for ensuring the
validity and applicability of research findings beyond the immediate
study context (\citeproc{ref-hernuxe1n2017}{Hernán 2017}). Before we
define selection bias we require the following definitions:

\begin{enumerate}
\def\labelenumi{\arabic{enumi}.}
\item
  \textbf{Source Population:} the population from which individuals are
  sampled for a study.
\item
  \textbf{Target Population:} the population intended to be understood
  through the study. The similarity of the source population to the
  target population in study-relevant characteristics enhances the
  applicability of the findings to the target population.
\item
  \textbf{Generalisability:} the extent to which findings from the
  source population apply to the target population. Findings are
  considered generalisable when the effects observed in the study group
  are also valid in the larger target group. The term \(PATE\) denotes
  the Population Average Treatment Effect in the target population, and
  \(ATE_{\text{source}}\) signifies the Average Treatment Effect in the
  source population. If there are characteristics \(W\) by which the
  source and target populations differ, the relationship between the two
  populations can be modeled as:

  \[PATE = f(ATE_{\text{source}}, W)\]
\item
  \textbf{Transportability:} An extension of generalisability, this
  concept relates to the applicability of study findings to different
  settings or groups not included in the original study. The findings
  are transportable if there is a function such that:

  \[ATE_{\text{target}} \approx f(ATE_{\text{source}}, T)\]

  Here, \(T\) represents variables differentiating the source and target
  populations.
\item
  \textbf{Structural Approach}: causation follows the arrow of time.
  Causal data science adopts a structural approach in which the exposure
  \(A\), outcome \(Y\), and pre-exposure confounders \(L\) are ordered
  in time such that:
\end{enumerate}

\[
L_{t0}\to A_{t1} \to Y_{t2}
\]

We are now ready to define selection bias.

\textbf{Selection bias} occurs when the study group (source population)
does not accurately represent the group of interest (target population).
There are two intervals in the temporal process that selection bias can
affect the quality and bias of causal inferences:

\begin{enumerate}
\def\labelenumi{(\arabic{enumi})}
\tightlist
\item
  Selection bias during pre-randomisation;
\item
  Selection bias post-randomisation.
\end{enumerate}

By `randomisation', I include the pseudo-randomisation of observational
studies. {[}Cite other paper.{]}

\subsubsection{Selection Bias Post
Randomisation}\label{selection-bias-post-randomisation}

Consider Figure~\ref{fig-1}, which presents a scenario in which there is
no marginal causal effect of exposure on the outcome, yet a degree of
selection into the study. We will assume randomisation succeeded so
there are no arrows into \(A\). As Figure~\ref{fig-1} shows, randomising
from a selected group does not result in confounding bias. Despite
selection, the null effect in the source population is not biased for
the target population (\citeproc{ref-greenland1977}{GREENLAND 1977};
\citeproc{ref-hernuxe1n2004}{Hernán 2004}).\footnote{Note we use the
  term ``null effect'' as a structural concept. There are no statistical
  ``null effects.'' Instead there are reliable or unreliable statistical
  effect estimates according to some measure of evidence and arbitrary
  threshold (\citeproc{ref-bulbulia2021}{Bulbulia \emph{et al.} 2021}).}
Under the null, selection does not induce ``confounding in expectation''
(\citeproc{ref-suzuki2016}{Suzuki \emph{et al.} 2016};
\citeproc{ref-suzuki2020}{Suzuki \emph{et al.} 2020};
\citeproc{ref-suzuki2014}{Suzuki and Yamamoto 2014}).

\begin{figure}

{\centering \includegraphics[width=0.6\textwidth,height=\textheight]{notes-selection-bias_files/figure-pdf/fig-1-1.pdf}

}

\caption{\label{fig-1}Selection under the null. An unmeasured variable
affects the selection for the study and the outcome. D-separation is
preserved; there is no confounding in expectation.}

\end{figure}

Figure~\ref{fig-2} presents a different scenario in which there is
selection bias for the population parameter: the association in the
population of selected individuals differs from the causal association
for the target population. Hernán calls this scenario ``selection bias
off the null'' (\citeproc{ref-hernuxe1n2017}{Hernán 2017}). Lu et
al.~call this scenario ``Type 2 selection bias''
(\citeproc{ref-lu2022}{Lu \emph{et al.} 2022}). Such bias occurs because
the selection into the study occurs on an effect modifier for the effect
of the exposure on the outcome. Note that although the causal effect of
\(A\to Y\) is unbiased for the exposed and unexposed in the source
population, the effect estimate does not generalise to the exposed and
unexposed in the target population:
\(PATE \cancel{\approx} ATE_{\text{selected sample}}\). Although there
is no ``confounding in expectation'' causal inferences in this scenario
do not generalise as we might hope (\citeproc{ref-suzuki2016}{Suzuki
\emph{et al.} 2016}; \citeproc{ref-suzuki2020}{Suzuki \emph{et al.}
2020}; \citeproc{ref-suzuki2014}{Suzuki and Yamamoto 2014}).

\begin{figure}

{\centering \includegraphics[width=0.6\textwidth,height=\textheight]{notes-selection-bias_files/figure-pdf/fig-2-1.pdf}

}

\caption{\label{fig-2}Selection off the null: an unmeasured variable
affects selection into the study and the outcome. Here the exposure
affects the outcome. Selection, then, is an effect modifier. Although
d-separation is preserved.}

\end{figure}

There has been considerable technical research investigating the
conditions under which causal estimates for a target population may be
identified when the source population differs from the target population
(see: (\citeproc{ref-lu2022}{Lu \emph{et al.} 2022})). There has also
been considerable technical research investigating the conditions in
which results might transport to populations that systematically differ
from the source population (see:
(\citeproc{ref-bareinboim2022}{Bareinboim \emph{et al.} 2022};
\citeproc{ref-deffner2022}{Deffner \emph{et al.} 2022};
\citeproc{ref-pearl2022}{\textbf{pearl2022?}})). I briefly review key
results.

\subsubsection{Selection Bias Post
Randomisation}\label{selection-bias-post-randomisation-1}

\subsubsection{Selection bias in which both the exposure and outcome
affect
censoring}\label{selection-bias-in-which-both-the-exposure-and-outcome-affect-censoring}

In panel designs, there is a constant threat of selection occurring
\emph{after} enrolment into the study. Panel attrition can be viewed as
a special case of selection bias because the participants who continue
in a longitudinal study may differ from those who drop out in ways that
generate structural biases. We next consider how selection may arise in
a three-wave longitudinal panel design. To address Type 2 selection bias
in a three-wave panel, we must accurately measure and adjust for a
sufficient set of covariates that affect selection \(\framebox{S}\)
(\citeproc{ref-lu2022}{Lu \emph{et al.} 2022}).\footnote{Note that when
  drawing causal diagrams, it is vital to present confounding as it is
  assumed to exist in the target population, not the source population
  (see Suzuki \emph{et al.} (\citeproc{ref-suzuki2020}{2020}),
  especially their examples in the supplement.) Practically speaking,
  where census data are available these should be collected for
  constructing survey weights (see: (\citeproc{ref-pishgar2021}{Pishgar
  \emph{et al.} 2021}; \citeproc{ref-stuart2015}{Stuart \emph{et al.}
  2015})).}

Figure~\ref{fig-3} describes a scenario in which both the exposure and
the true outcome affect panel attrition, biasing the observed
association between the exposure and the measured outcome in the
remaining sample. The problem of selection here is a problem of collider
stratification bias. We can equivalently view the problem as one of
directed measurement error, described in in Figure~\ref{fig-3}. Either
way, restricting the analysis to the retained sample introduces bias in
the causal effect estimate by opening a backdoor path from the exposure
to the outcome. Lu \emph{et al.} (\citeproc{ref-lu2022}{2022}) call this
form of bias: ``Type 1 selection bias'' and distinguishes between
scenarios when causal effects that generalise are recoverable (Type 1a
selection bias) and not recoverable (Type 1b selection bias). In both
cases, we must develop strategies to evaluate whether we may recover
from the subset of the source population that has been censored, causal
effect estimates that generalise to a clearly defined target population.

\begin{figure}

{\centering \includegraphics[width=0.8\textwidth,height=\textheight]{notes-selection-bias_files/figure-pdf/fig-3-1.pdf}

}

\caption{\label{fig-3}Causal diagram in which outcome and exposure
affect attrition. Dashed red path shows correlation of A an Y in the
absence of causation.}

\end{figure}

\subsubsection{Example of selection bias in a three-wave panel were
there are systematic differences between the source population at
baseline and at follow
up.}\label{example-of-selection-bias-in-a-three-wave-panel-were-there-are-systematic-differences-between-the-source-population-at-baseline-and-at-follow-up.}

Figure~\ref{fig-4} shows selection bias manifest in a three-wave panel
design when loss-to-follow-up results in a systematic disparity between
the baseline and follow-up source populations. The red dashed lines in
the diagram represent an open backdoor path, revealing a potential
indirect association between the exposure and the outcome. Upon
considering only the selected sample (i.e., when we condition on the
selected sample \(\framebox{S}\)), we may create or obscure associations
not evident in the source population at baseline.

\begin{figure}

{\centering \includegraphics[width=0.8\textwidth,height=\textheight]{notes-selection-bias_files/figure-pdf/fig-4-1.pdf}

}

\caption{\label{fig-4}Causal diagram of a three-wave panel design with
selection bias. Red paths reveal the open backdoor path induced by
conditioning on the selected sample.}

\end{figure}

\subsubsection{Unmeasured confounder affects outcome and variable that
affects
attrition}\label{unmeasured-confounder-affects-outcome-and-variable-that-affects-attrition}

Figure~\ref{fig-5} presents another problem of selection bias in a
three-wave panel design. This diagram shows how an unmeasured
confounder, \(U_S\), can simultaneously influence the outcome variable
\(Y_{t2}\) and another variable, \(L_{t2}\), responsible for attrition
(i.e., the drop-out rate, denoted as \(\framebox{S}\)). Here we present
a scenario in which the exposure variable \(A_{t1}\) can impact a
post-treatment confounder \(L_{t2}\), which subsequently affects
attrition, \(\framebox{S}\). If the study's selected sample descends
from L\({t2}\), the selection effectively conditions on \(L_{t2}\),
potentially introducing bias into the analysis. Figure~\ref{fig-5} marks
this biasing pathway with red-dashed lines.

\begin{figure}

{\centering \includegraphics[width=0.8\textwidth,height=\textheight]{notes-selection-bias_files/figure-pdf/fig-5-1.pdf}

}

\caption{\label{fig-5}Causal diagram of a three-wave panel design with
selection bias: Unmeasured confounder U\_S, is a cause of both of the
outcome Y\_t2 and of a variable, L\_t2 that affects attrition, S. The
exposure A affects this cause L\_t2 of attrition, S. The selected sample
is a descendent of Lt2. Hence selection is a form of conditioning on
L\_t2. Such conditioning opens a biasing path, indicated by the
red-dashed lines.}

\end{figure}

\subsubsection{The outcome affects
selection}\label{the-outcome-affects-selection}

\begin{figure}

{\centering \includegraphics[width=0.8\textwidth,height=\textheight]{notes-selection-bias_files/figure-pdf/fig-6-1.pdf}

}

\caption{\label{fig-6}Y affects attrition, as does an unmeasured
variable that also effects treatment.}

\end{figure}

\subsubsection{Why regression adjustment fails to address select bias
from
attrition}\label{why-regression-adjustment-fails-to-address-select-bias-from-attrition}

We cannot address selection bias from attrition using regressions.
Suppose that sicker individuals or those with less successful outcomes
are more likely to drop out of the study. If this occurs, the remaining
sample will not represent the original population; it will
over-represent healthier individuals or those with more successful
outcomes.

What to do?

\begin{enumerate}
\def\labelenumi{\arabic{enumi}.}
\item
  \textbf{Retain sample}: the best way to deal with missing data is to
  prevent it in the first place. Maintain regular contact with study
  participants, using incentives for continued participation, making
  follow-ups as convenient as possible, and tracking participant details
  from multiple sources (email, phone, secondary contacts).
\item
  \textbf{Missing data imputation}: this requires predicting the missing
  values based on our data, assuming that the missingness is random
  conditional on baseline measures. Note that this missingness should be
  predicted separately within strata of the exposed and unexposed in the
  previous wave (see: (\citeproc{ref-westreich2015}{Westreich \emph{et
  al.} 2015}; \citeproc{ref-zhang2023}{Zhang \emph{et al.} 2023})).
  Imputation methods typically assume that data are missing conditional
  on a correctly specified model and information obtained at baseline.
\item
  \textbf{Inverse probability weighting with censoring weights}: this
  requires weighting the values of each participant in the study by the
  inverse of the probability of their observed pattern of missingness
  (censoring weights)(\citeproc{ref-leyrat2019}{\textbf{leyrat2019?}};
  \citeproc{ref-cole2008}{\textbf{cole2008?}}). In this approach, the
  sample gives more weight to under-represented individuals owing to
  drop-out. As with missing data imputation, IPW with censoring weights
  also assumes that we can correctly model the missingness from the
  observed data (\citeproc{ref-shiba2021}{Shiba and Kawahara 2021}).
\item
  \textbf{Sensitivity analysis}: as with nearly all causal inference, we
  should quantitatively evaluate how sensitive results are to different
  assumptions and methods for handling censoring events
  (\citeproc{ref-shi2021}{Shi \emph{et al.} 2021}).
\end{enumerate}

\subsubsection{Integrating Strategies for Addressing Selection Bias in
Evolutionary Human
Sciences}\label{integrating-strategies-for-addressing-selection-bias-in-evolutionary-human-sciences}

In addressing the challenge of selection bias in evolutionary human
sciences, particularly arising from attrition in longitudinal studies,
it is crucial to adopt a multifaceted and vigilent approach. This
approach should not only incorporate strategies to mitigate selection
bias but also ensure these strategies are seamlessly integrated into
both the design and analysis phases of research. The key lies in
harmonizing proactive measures with analytical techniques to create a
robust research methodology.

\paragraph{Recommended Approach to Study Design and
Analysis}\label{recommended-approach-to-study-design-and-analysis}

\begin{enumerate}
\def\labelenumi{\arabic{enumi}.}
\item
  \textbf{Enhancing sample representation}: begin with a broad and
  diverse sampling strategy. This ensures the representativeness of the
  target population in the study, addressing potential selection biases
  from the onset.
\item
  \textbf{Precise measurement and covariate adjustment}: use precise
  measurements and adjust for covariates (CITE). This step is crucial in
  controlling for Type 2 selection bias and ensuring that the results
  are reflective of the true relationships within the target population.
\item
  \textbf{Causal diagrams reflecting target population structure}:
  develop causal diagrams that accurately represent the confounding
  structure as it exists in the target population. This helps in
  identifying potential sources of bias and in designing appropriate
  strategies to address them.
\item
  \textbf{Focus on participant retention}: Implement strategies to
  maximize participant retention, such as regular communication,
  incentives, and multiple contact methods. High retention rates reduce
  the risk of attrition-related biases significantly.
\item
  \textbf{Adopt appropriate analytical methods}: in cases where
  attrition is unavoidable, employ advanced analytical methods like
  multiple imputation or inverse probability weighting with censoring
  weights. These methods help in adjusting for the biases introduced by
  missing data.
\item
  \textbf{Conduct sensitivity analysis for robustness}: conduct
  sensitivity analyses to assess how different assumptions about missing
  data and other factors impact the study's findings. This is a critical
  step in ensuring the validity and reliability of the research
  conclusions.
\end{enumerate}

\subsubsection{Summary}\label{summary}

By integrating these strategies into both the design and analysis phases
of research, we can effectively address the challenges posed by
selection bias, particularly in the context of evolutionary human
sciences. This approach will not only enhance the accuracy and
applicability of the findings but also contributes to the overall
robustness and credibility of the research.

\subsubsection{Acknowledgements}\label{acknowledgements}

\subsection*{References}\label{references}
\addcontentsline{toc}{subsection}{References}

\phantomsection\label{refs}
\setlength{\cslentryspacing}{0em}
\begin{CSLReferences}
\bibitem[\citeproctext]{ref-bareinboim2022}
Bareinboim, E, Tian, J, and Pearl, J (2022) Recovering from selection
bias in causal and statistical inference. In, 1st edn, Vol. 36, New
York, NY, USA: Association for Computing Machinery, 433450. Retrieved
from \url{https://doi.org/10.1145/3501714.3501740}

\bibitem[\citeproctext]{ref-bulbulia2021}
Bulbulia, J, Schjoedt, U, Shaver, JH, Sosis, R, and Wildman, WJ (2021)
Causal inference in regression: Advice to authors. \emph{Religion, Brain
\& Behavior}, \textbf{11}(4), 353360.

\bibitem[\citeproctext]{ref-deffner2022}
Deffner, D, Rohrer, JM, and McElreath, R (2022) A Causal Framework for
Cross-Cultural Generalizability. \emph{Advances in Methods and Practices
in Psychological Science}, \textbf{5}(3), 25152459221106366.
doi:\href{https://doi.org/10.1177/25152459221106366}{10.1177/25152459221106366}.

\bibitem[\citeproctext]{ref-greenland1977}
GREENLAND, S (1977) Response and follow-up bias in cohort studies.
\emph{American Journal of Epidemiology}, \textbf{106}(3), 184--187.
doi:\href{https://doi.org/10.1093/oxfordjournals.aje.a112451}{10.1093/oxfordjournals.aje.a112451}.

\bibitem[\citeproctext]{ref-hernuxe1n2004}
Hernán, MA (2004) A definition of causal effect for epidemiological
research. \emph{Journal of Epidemiology \& Community Health},
\textbf{58}(4), 265--271.
doi:\href{https://doi.org/10.1136/jech.2002.006361}{10.1136/jech.2002.006361}.

\bibitem[\citeproctext]{ref-hernuxe1n2017}
Hernán, MA (2017) Invited commentary: Selection bias without colliders
\textbar{} american journal of epidemiology \textbar{} oxford academic.
\emph{American Journal of Epidemiology}, \textbf{185}(11), 10481050.
Retrieved from \url{https://doi.org/10.1093/aje/kwx077}

\bibitem[\citeproctext]{ref-lu2022}
Lu, H, Cole, SR, Howe, CJ, and Westreich, D (2022) Toward a Clearer
Definition of Selection Bias When Estimating Causal Effects.
\emph{Epidemiology (Cambridge, Mass.)}, \textbf{33}(5), 699--706.
doi:\href{https://doi.org/10.1097/EDE.0000000000001516}{10.1097/EDE.0000000000001516}.

\bibitem[\citeproctext]{ref-pishgar2021}
Pishgar, F, Greifer, N, Leyrat, C, and Stuart, E (2021) MatchThem::
Matching and weighting after multiple imputation. \emph{The R Journal},
\textbf{13}(2), 292305.
doi:\href{https://doi.org/10.32614/RJ-2021-073}{10.32614/RJ-2021-073}.

\bibitem[\citeproctext]{ref-shi2021}
Shi, B, Choirat, C, Coull, BA, VanderWeele, TJ, and Valeri, L (2021)
CMAverse: A suite of functions for reproducible causal mediation
analyses. \emph{Epidemiology}, \textbf{32}(5), e20e22.

\bibitem[\citeproctext]{ref-shiba2021}
Shiba, K, and Kawahara, T (2021) Using propensity scores for causal
inference: Pitfalls and tips. \emph{Journal of Epidemiology},
\textbf{31}(8), 457463.

\bibitem[\citeproctext]{ref-stuart2015}
Stuart, EA, Bradshaw, CP, and Leaf, PJ (2015) Assessing the
Generalizability of Randomized Trial Results to Target Populations.
\emph{Prevention Science}, \textbf{16}(3), 475--485.
doi:\href{https://doi.org/10.1007/s11121-014-0513-z}{10.1007/s11121-014-0513-z}.

\bibitem[\citeproctext]{ref-suzuki2016}
Suzuki, E, Mitsuhashi, T, Tsuda, T, and Yamamoto, E (2016) A typology of
four notions of confounding in epidemiology. \emph{Journal of
Epidemiology}, \textbf{27}(2), 49--55.
doi:\href{https://doi.org/10.1016/j.je.2016.09.003}{10.1016/j.je.2016.09.003}.

\bibitem[\citeproctext]{ref-suzuki2020}
Suzuki, E, Shinozaki, T, and Yamamoto, E (2020) Causal Diagrams:
Pitfalls and Tips. \emph{Journal of Epidemiology}, \textbf{30}(4),
153--162.
doi:\href{https://doi.org/10.2188/jea.JE20190192}{10.2188/jea.JE20190192}.

\bibitem[\citeproctext]{ref-suzuki2014}
Suzuki, E, and Yamamoto, E (2014) Further refinements to the
organizational schema for causal effects. \emph{Epidemiology},
\textbf{25}(4), 618.
doi:\href{https://doi.org/10.1097/EDE.0000000000000114}{10.1097/EDE.0000000000000114}.

\bibitem[\citeproctext]{ref-westreich2015}
Westreich, D, Edwards, JK, Cole, SR, Platt, RW, Mumford, SL, and
Schisterman, EF (2015) Imputation approaches for potential outcomes in
causal inference. \emph{International Journal of Epidemiology},
\textbf{44}(5), 17311737.

\bibitem[\citeproctext]{ref-zhang2023}
Zhang, J, Dashti, SG, Carlin, JB, Lee, KJ, and Moreno-Betancur, M (2023)
Should multiple imputation be stratified by exposure group when
estimating causal effects via outcome regression in observational
studies? \emph{BMC Medical Research Methodology}, \textbf{23}(1), 42.
doi:\href{https://doi.org/10.1186/s12874-023-01843-6}{10.1186/s12874-023-01843-6}.

\end{CSLReferences}



\end{document}
